%!TEX program = arara
% arara: xelatex: {shell: yes, synctex: yes}
% arara: xelatex: {shell: yes, synctex: yes}
\documentclass[a4paper, 10pt, twoside, calcwidth]{oblivoir}
\usepackage[left=3.5cm,right=3.5cm,top=3.5cm,bottom=3.5cm, a4paper]{geometry}
\usepackage{subfiles}
\usepackage{amsmath, amssymb, amsfonts}
\usepackage{upgreek}
\usepackage{titlesec}
\usepackage{titletoc}
\usepackage{tocloft}
\usepackage{stmaryrd}
\usepackage{float}
\usepackage{mathpazo}
\usepackage{mathtools}
\usepackage[datesep={-}]{datetime2}
\usepackage{titlesec}
\usepackage{xcolor}
\usepackage{minted}
\usepackage{fancyhdr}
\usepackage{textcomp}
\usepackage{chngcntr}
\counterwithin{footnote}{section}
\usepackage[colorinlistoftodos,prependcaption,textsize=tiny]{todonotes}
\usepackage{subcaption}
\usepackage[normalem]{ulem}
\usepackage{graphicx}
\usepackage{physics}

\ifPDFTeX
  \usepackage{newpxtext}
  \usepackage{tgadventor}
  \usepackage{inconsolata}
\else\ifLuaOrXeTeX
  \setmainfont{TeX Gyre Pagella}
  \setsansfont{Noto Sans}
  \setmonofont{Ubuntu Mono}
  \setkomainfont(Noto Serif CJK KR)(* Bold)(Noto Sans CJK KR)
  \setkosansfont[Noto Sans CJK KR]()( Bold)( Medium)
  \setkomonofont[Noto Sans Mono CJK KR]()( Bold)()
\fi\fi

\renewcommand{\vec}[1]{\mathbf{#1}}

\pagestyle{fancy}
\fancyhf{}
\fancyhead[LO]{\fontsize{9}{11}\selectfont\textsf{Python 및 알고리즘 입문 교재}}
\fancyhead[LE]{\fontsize{9}{11}\selectfont\textsf\rightmark}
\fancyhead[R]{\fontsize{9}{11}\selectfont\textsf\thepage}

\renewcommand{\contentsname}{\normalfont\Large\bfseries\sffamily\color{black!60}{목차}}

\newlength\myseclen
\setlength\myseclen{\dimexpr\oddsidemargin+\hoffset\relax-0.9em}
\newlength\mysubseclen
\setlength\mysubseclen{\dimexpr\oddsidemargin+\hoffset\relax}

\titleformat{\section}
{\normalfont\Large\bfseries\sffamily\color{black!60}}{\llap{\hspace*{-\myseclen}\thesection\hfill}}{0em}{}

\titleformat{\subsection}
{\normalfont\large\bfseries\sffamily\color{black!60}}{\llap{\hspace*{-\mysubseclen}\thesubsection\hfill}}{0em}{}

\begin{document}

\begin{center}\sffamily
  \text{}\\[1cm]
  \huge Python 및 정보과학 입문 교재

  \vspace{.1cm}
  \Large Python 3.7 및 기본적인 알고리즘에 대한 이해\\[.6cm]
  \begin{tabular} {c c}
    교재 제작 & v3.3\\
    이재호 & \today
  \end{tabular}
\end{center}
\vspace{.5cm}
\noindent\rule[0.5ex]{\linewidth}{.5pt}
\tableofcontents*

\pagebreak
\mbox{}
\vfill
\noindent\rule[0.5ex]{\linewidth}{.5pt}
{\scriptsize
  \noindent Copyright (c)  2018, 2019  Jaeho Lee.\\
  Permission is granted to copy, distribute and/or modify this document
  under the terms of the GNU Free Documentation License, Version 1.3
  or any later version published by the Free Software Foundation;
  with no Invariant Sections, no Front-Cover Texts, and no Back-Cover Texts.
  A copy of the license is included in the section entitled "GNU
  Free Documentation License".
}

\pagebreak

\section{Python 소개 및 기본 요소}

\subfile{lectures/lecture1}

\section{Functions함수와 Conditionals조건문}

\subfile{lectures/lecture2}

\section{Boolean Functions불리언 함수와 Loops반복문 기본}

\subfile{lectures/lecture3}

\section{Lists리스트, Strings문자열, Counters카운터}

\subfile{lectures/lecture4}

\section{Quantifiers한정자와 While 문}

% \subfile{lectures/lecture5}

\section{Loops반복문 응용과 파일 입출력}

% \subfile{lectures/lecture6}

% \section{토이 로봇 활용 (선택)}
% \subfile{lectures/lecture7}

\section{Recursion재귀법, Python의 다양한 객체, 그리고 Lambda람다 함수}
% \subfile{lectures/lecture8}

\section{Object-Oriented Programming객체 지향 프로그래밍}

\section{정렬 알고리즘}

\section{Divide-and-Conquer분할 정복법}

\section{Dynamic Programming동적 계획법}

\end{document}
