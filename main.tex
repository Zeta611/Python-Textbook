\documentclass[a4paper,10pt]{memoir}

%%% Overall document settings
\usepackage{microtype}

\usepackage{subfiles}

\usepackage{caption}
\usepackage{subcaption}

\renewcommand*{\figurename}{그림}

\usepackage{makeidx}
\renewcommand*{\indexname}{찾아보기}
\makeindex

\usepackage{minted}
\setminted{
  mathescape,breaklines,numbersep=5pt,frame=lines,framesep=2mm,autogobble
}

\usepackage{relsize} % \smaller

% For madsen chapter style:
\usepackage{graphicx}

\usepackage[no-math]{fontspec}
% Oblivoir does the following automatically:
\usepackage{kotex}
\hangingpunctuations
\usepackage[pdfencoding=auto,bookmarks]{hyperref}
\ifpdf
  \input glyphtounicode
  \pdfgentounicode=1
\fi

%%% Math settings
\usepackage{amssymb,amsmath} % Before unicode-math
\usepackage[math-style=TeX,bold-style=TeX]{unicode-math}

%%% Font settings
\setmainfont{Libertinus Serif}[Scale=1.11]
\setsansfont{Libertinus Sans}[Scale=1.11]
\setmonofont{Inconsolata}

\setmathfont{Libertinus Math}[Scale=1.11] % Before set*hangulfont

\setmainhangulfont{Noto Serif CJK KR}
\setsanshangulfont[BoldFont={* Bold}]{KoPubWorldDotum_Pro}
\setmonohangulfont{D2Coding}

\newfontfamily{\titlefont}{Alegreya}
\newhangulfontfamily{\authorfont}{MaruBuriOTF}
\newfontfamily{\oldnumfont}{Libertinus Serif}[Ligatures=TeX,Numbers=OldStyle]

%%% Memoir settings
\OnehalfSpacing
\headstyles{komalike}
\chapterstyle{madsen}

\copypagestyle{customruled}{ruled}
\makeevenfoot{customruled}{\oldnumfont\thepage}{}{}
\makeoddfoot{customruled}{}{}{\oldnumfont\thepage}
\pagestyle{customruled}

\makepagestyle{customchapter}
\makeoddfoot{customchapter}{}{\oldnumfont\thepage}{}
\aliaspagestyle{chapter}{customchapter}

\pretitle{%
  \begin{flushright}%
    \fontsize{1.67cm}{2cm}\selectfont\titlefont}
  \title{Programming with Python 101}
  \posttitle{%
    \par\end{flushright}\vskip2.1cm}

\preauthor{\begin{flushright}\huge\authorfont}
  \author{이재호}
  \postauthor{\par\end{flushright}\vskip0.9cm}

\predate{\begin{flushright}\huge\sffamily\scshape}
  \date{}
  \postdate{\par\end{flushright}}

\setsecnumdepth{subsection}

\settocdepth{subsection}
\renewcommand*{\contentsname}{목차}
\renewcommand*{\cftchapterfont}{\sffamily}
\renewcommand*{\cftsectionfont}{\sffamily}
\renewcommand*{\cftsubsectionfont}{\sffamily\small}
\renewcommand*{\cftchapterpagefont}{\oldnumfont\bfseries}
\renewcommand*{\cftsectionpagefont}{\oldnumfont}
\renewcommand*{\cftsubsectionpagefont}{\oldnumfont}

\epigraphfontsize{\small\itshape}
\setlength\epigraphwidth{8cm}
\setlength\epigraphrule{0pt}

%%% Custom commands
% Math
\renewcommand*{\vec}[1]{\symbf{#1}}
\newcommand*{\norm}[1]{\lVert#1\rVert}

% Typesetting
% Korean-English, e.g., \alt{메모리 재활용}{garbage collection}
\ExplSyntaxOn

\NewDocumentCommand { \myindex } { s m m }
  {
    \IfBooleanTF {#1}
      { \index{#2!#3|textbf}\index{#3!#2|textbf} }
      { \index{#2!#3}\index{#3!#2} }
  }

\NewDocumentCommand { \alt } { s m O{#2} d<> m O{#5} d<> }
  {
    #2\,{\color{gray!50!darkgray}\smaller\textit{#5}}\hspace{.06667em}
    \IfBooleanT {#1}
      {
        \myindex{#3}{#6}

        \IfNoValueTF {#4}
          {
            \IfNoValueF {#7}
              { % See English
                \index{#6|see{#7}}
              }
          }
          {
             \IfNoValueTF {#7}
              { % See Korean
                \index{#3|see{#4}}
              }
              { % See both
                \index{#3|see{#4}}\index{#6|see{#7}}
              }
          }
      }
  }
\ExplSyntaxOff

\newcommand*{\booktitle}[1]{\textit{#1}}

\ExplSyntaxOn
\NewDocumentEnvironment { epi } { m m +b }
  { \epigraph{#3}{---~\textup{#1},\ #2} }
  { }
\ExplSyntaxOff

\newcommand*{\progname}[1]{\textsf{#1}}

\newcommand*{\VSC}{VS\,Code}

\begin{document}
\frontmatter
\begin{titlingpage}
  \setlength{\droptitle}{10.5cm}
  \calccentering{\unitlength}
  \begin{adjustwidth*}{\unitlength}{-\unitlength}
    \maketitle
  \end{adjustwidth*}
\end{titlingpage}

\tableofcontents

\chapter{머리말}
\begin{epi}{Lewis Carroll}{Alice in Wonderland}
  ``Begin at the beginning," the King said gravely, ``and go on till you come to the end: then stop."
\end{epi}
\subfile{lectures/lecture0}

\mainmatter
\chapter{시작하기}
\begin{epi}{Brian Kernighan}{The C Programming Language}
  \texttt{printf("hello, world\textbackslash n");}
\end{epi}
\subfile{lectures/lecture1}

\chapter{Functions함수와 Conditionals조건문}
\begin{epi}{Donald Knuth}{The Art of Computer Programming}
  When a certain task is to be performed at several different places in a program, it is usually undesirable to repeat the coding in each place.
  To avoid this situation, the coding (called a subroutine) can be put into one place only, and a few extra instructions can be added to restart the outer program properly after
the subroutine is finished.
\end{epi}
\subfile{lectures/lecture2}

\chapter{Boolean Functions불리언 함수와 Loops반복문 기본}
\begin{epi}{Friedrich Nietzsche}{Twilight of the Idols}
  Formula of my happiness: a Yes, a No, a straight line, a goal.
\end{epi}
\subfile{lectures/lecture3}

\chapter{Lists리스트, Strings문자열, Counters카운터}
\begin{epi}{Aristotle}{Ethica Nicomachea II}
  What we have to learn to do we learn by doing.
\end{epi}
\subfile{lectures/lecture4}

\chapter{[DRAFT] Quantifiers한정자와 While 문}
\begin{epi}{Philip K. Dick}{VALIS}
  There exists, for everyone, a sentence---a series of words---that has the power to destroy you.
  Another sentence exists, another series of words, that could heal you.
  If you're lucky you will get the second, but you can be certain of getting the first.
\end{epi}
\subfile{lectures/lecture5}

\chapter{[DRAFT] Loops반복문 응용과 파일 입출력}
\begin{epi}{Alan Perlis}{Epigrams in Programming}
  A program without a loop and a structured variable isn't worth writing.
\end{epi}
\subfile{lectures/lecture6}

\chapter{토이 로봇 활용}
\begin{epi}{Isaac Asimov}{I, Robot}
  A robot must obey the orders given it by human beings except where such orders would conflict with the First Law.
\end{epi}
\subfile{lectures/lecture7}

\chapter{[DRAFT] Recursion재귀법, Python의 다양한 객체, 그리고 Lambda람다 함수}
\begin{epi}{Douglas Hofstadter}{Gödel, Escher, Bach}
  Sometimes recursion seems to brush paradox very closely.
  For example, there are \emph{recursive definitions}.
  Such a definition may give the casual viewer the impression that something is being defined in terms of \emph{itself}.
  That would be circular and lead to infinite regress, if not to paradox proper.
  Actually, a recursive definition (when properly formulated) never leads to infinite regress or paradox.
  This is because a recursive definition never defines something in terms of itself, but always in terms of \emph{simpler versions} of itself.
\end{epi}
\subfile{lectures/lecture8}

% \chapter{[TBD] Object-Oriented Programming객체 지향 프로그래밍}
% \begin{epigraph}{Steve Jobs}{The Guts of a New Machine}
%   Design is not just what it looks like and feels like. Design is how it works.
% \end{epigraph}

% \chapter{[TBD] 정렬 알고리즘}

% \chapter{[TBD] Divide-and-Conquer분할 정복법}

% \chapter{[TBD] Dynamic Programming동적 계획법}

\backmatter
\renewcommand{\preindexhook}{\begin{epi}{Albert H. Highton}{Practical Proofreading}
Important works such as histories, biographies, scientific and technical text-books, etc., should contain indexes. Indeed, such works are scarcely to be considered complete without indexes.
An index is almost invariably placed at the end of a volume and is set in smaller type than the text-matter. Its subjects should be thoroughly alphabetized.
The compiling of an index is interesting work, though some authors are apt to find it tedious and delegate the work to others. The proofreader who undertakes it will find that it is splendid mental exercise and brings out his latent editorial capability.
\end{epi}\vskip\baselineskip}
\printindex

\end{document}
%%% Local Variables:
%%% TeX-command-extra-options: "-shell-escape"
%%% coding: utf-8
%%% mode: latex
%%% TeX-engine: xetex
%%% End: