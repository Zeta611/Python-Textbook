\documentclass[../main.tex]{subfiles}

\begin{document}
\alt{파이썬}{Python}은 문법이 단순하지만 활용 가능성이 무궁무진한 언어입니다.
단순히 학교에서 필수적으로 가르치기 때문에 파이썬을 배운다기보다는 앞으로 다양한 분야에서 오랫동안 활용할 수 있는 유용한 도구를 배운다는 마음가짐을 가지면 좋겠습니다.
파이썬은 R이나 \textsc{Matlab} 등과 함께 학문적인 용도로도 쓰임이 많은 언어입니다.
특히 NumPy, SciPy, Matplotlib, Pandas 등의 패키지를 활용한다면 \textsc{Matlab}이나 Mathematica와 같은 유료 프로그래밍 언어가 할 수 있는 다양한 작업들을 대등하게 할 수 있습니다.
나아가 파이썬은 오픈 소스에 무료인데다가, 여러 분야에서 활용할 수 있는 범용 프로그래밍 언어입니다.

저는 물리학, 천문학, 화학, 생물학 등의 학문 분야에서 데이터 분석 및 시각화를 위해 파이썬을 사용하기도 하고, Raspberry Pi에 연결된 다양한 센서와 카메라 모듈 등을 제어하기 위해 파이썬을 사용한 프로젝트를 진행하는데에도 사용했습니다.
이 외에도 사이트에서 여러 자료를 동시에 다운로드 받는 파이썬 스크립트를 작성하거나, 간단한 계산을 수행하기 위해서 사용하는 등 파이썬은 제 일상에 자연스럽게 녹아든 든든한 도구가 되었습니다.

웹 서버 개발에 관심이 있다면, 파이썬을 사용해 손쉽게 서버를 만들 수 있습니다.
특히 Django, Flask, FastAPI 등의 프레임워크를 사용한다면 큰 규모의 서버도 무리 없이 개발할 수 있습니다.
YouTube, Dropbox, Facebook, Netflix, Google, Instagram, Spotify 등의 대형 서비스들은 이미 여러 부품에 파이썬을 사용하고 있습니다.

딥러닝에 관심이 있다면, 파이썬은 필수로 배워야 하는 언어입니다.
2022년 현재 PyTorch, Tensorflow, Hugging Face 등 거의 모든 딥러닝 프레임워크는 파이썬을 통한 API를 제공하고 있습니다.

나아가 파이썬은 스크립트 언어로 쉘 스크립트 대신 시스템 수준의 작업을 수행할 때 사용할 수 있습니다.
과거에는 awk, \alt{펄}{Perl} 등의 언어를 사용하는 작업을 파이썬으로 작성하는 경우가 증가하고 있습니다.

이처럼 파이썬은 한 번 배워두면 수 많은 곳에서 유용하게 사용할 수 있습니다.

\section{구성 및 독자층}
본 교재는 파이썬 3의 문법과 알고리즘을 기초부터 익힐 수 있도록 구성되어 있습니다.
특히 파이썬의 기본적인 문법 숙지와, 자주 사용되는 패턴에 익숙해지는 것에 초점을 맞추고 있습니다.

전체적으로는 프로그래밍 언어를 처음 접하더라도 익힐 수 있도록 구성되어 있으나, 일부 내용은 어느 정도의 수학적 배경 지식을 가정합니다.

\end{document}
