\documentclass[../main.tex]{subfiles}

\begin{document}

\section{리스트}
지난 단원에서는 반복문으로 단순 반복 작업을 수행하는 법에 대해 알아보았습니다.
하지만 지난 시간에 맛 본 \texttt{for}문은 단순히 \pyin{range}에서 받아온 값을 그대로 사용하는 것에 그쳤습니다.
만약 \texttt{i}에 4를 대입하는 경우, 4를 그대로 사용해 일반항을 계산한 것입니다.

\alt*{리스트}{list}[lists]를 사용하면 임의의 수열에 대해 일반적인 연산을 할 수 있습니다.
리스트는 일종의 (유한) 수열로 생각할 수 있는데, \texttt{for}문과 리스트를 적절히 사용하면 수열의 원소를 접근해 일반적인 계산을 수행할 수 있습니다.

지금까지는 10개의 변수를 받아 합을 반환하는 함수를 작성하기 위해 아래와 같은 코드를 작성했어야 합니다.
\begin{minted}{python}
def sum_ten(n1, n2, n3, n4, n5, n6, n7, n8, n9, n10):
    summ = 0
    summ += n1
    summ += n2
    ...
    summ += n10
    return summ
\end{minted}
심지어 \texttt{for}문을 활용할 수도 없습니다.
다음과 같은 문법은 허용되지 않기 때문입니다:
\begin{minted}{python}
def sum_ten(n1, n2, n3, n4, n5, n6, n7, n8, n9, n10):
    summ = 0
    for i in range(1, 11):
        summ += ni
    return summ
\end{minted}
위에서 \texttt{ni}는 \texttt{ni}라는 이름을 가진 변수에 불과합니다.
하지만 이런 이름의 변수는 정의된 적이 없으므로 오류를 뿜어내게 됩니다.

리스트를 사용하면 임의의 개수의 변수를 (담는 리스트를) 받아 계산을 할 수 있습니다:
\begin{minted}{python}
def sum_arb(numbers):
    summ = 0
    for i in range(len(numbers)):
        summ += numbers[i]
    return summ
\end{minted}
혹은 더 간단히,
\begin{minted}{python}
def sum_arb(numbers):
    summ = 0
    for n in numbers:
        summ += n
    return summ
\end{minted}
와 같이 바로 \verb/numbers/에서 원소를 \verb/n/이라는 이름으로 하나씩 꺼내올 수도 있습니다.

\subsection{리스트의 생성}
리스트는 보통 세 가지 방법으로 생성할 수 있습니다:
\begin{enumerate}
\item 원소들이 이미 정해진 경우,
\item 원소들의 개수가 이미 정해진 경우,
\item 임의의 원소들이 들어갈 수 있는 경우
\end{enumerate}
입니다.
각 경우를 살펴봅시다.

\subsubsection{원소들이 이미 정해진 경우}
원소들이 이미 정해진 경우 리스트에 직접 원소를 입력하여 생성할 수 있습니다.
\begin{minted}{python}
numbers = [3, 1, 4, 1, 5, 9, 2, 6]
\end{minted}
리스트의 값은 나중에 수정할 수 있기 때문에 처음에 채워 넣은 값들은 초기값입니다.
하지만 이렇게 값을 입력하는 것은 리스트가 담고 있는 원소의 개수가 적을 경우입니다.

\subsubsection{원소들의 개수가 이미 정해진 경우}
리스트에 담을 값이 규칙적이면서 크기가 정해져 있다면, 값을 \pyin{None}으로 초기화한 후 정해진 크기를 가진 리스트를 생성할 수 있습니다.
아래는 크기 10의 리스트를 생성한 것입니다.
\begin{minted}{python}
numbers = [None] * 10
\end{minted}
\pyin{[None] * 10}은 \pyin{[None, None, None, None, None, None, None, None, None, None]}과 동일합니다.
위에서와 같이 \pyin{None} 말고 정수인 \texttt{-1}이나 문자열인 \pyin{""} 등으로도 채워넣을 수 있습니다.
단, \texttt{0}과 같은 값은 실제로 값을 채워 넣은 유의미한 \texttt{0}과 혼동될 수 있으니 유의해야 합니다.

혹은, 처음에 적응하는데 조금 시간이 걸릴 수도 있지만 \alt*{리스트 표현}{list comprehension}을 사용할 수도 있습니다:
\begin{minted}{python}
numbers = [None for i in range(10)]
\end{minted}
위 코드에서 \verb/i/는 사용되지 않으므로 \pyin{[None for _ in range(10)]}와 같이 다시 쓸 수도 있습니다.
이 같은 리스트 표현은 다차원 리스트를 초기화할 때 필요한데요, 이는 해당 절에서 더 자세히 확인할 수 있습니다.\footnote{리스트의 \texttt{*}는 \alt*{얕은 복사}{shallow copy}를 하는 방면 리스트 표현을 쓰면 복사가 아니라 각 값이 새로 만들어지기 때문입니다.}

이렇게 만든 리스트 안의 \pyin{None}들을 대신할 값을 채워야 합니다.
바로 \texttt{for}문으로 해결할 수 있습니다.
다음은 첫 열 개의 3의 배수를 넣는 과정입니다.
\begin{minted}{python}
numbers = [None] * 10

for i in range(len(numbers)):
    numbers[i] = (i + 1) * 3
\end{minted}
\pyin{len}\index{\texttt{len}} 함수는 리스트의 길이를 반환합니다.
예컨대 \pyin{len(numbers)}는 \texttt{10}을 반환합니다.
보다 일반적이고 재사용 가능한 명료한 코드를 작성하기 위해서는 리스트의 길이를 숫자 \texttt{10}과 같이 직접 치는 것보다는 \pyin{len}을 사용하는 것이 좋습니다.

리스트의 원소를 접근할 때는 \pyin{lst[index]}와 같이 쓸 수 있습니다.
\pyin{index}는 \texttt{0}에서 \pyin{len(lst) - 1}까지의 값을 가집니다.
즉, 리스트의 첫 번째 원소를 접근하려면 \pyin{lst[1]}이 아닌 \pyin{lst[0]}을, \texttt{n}번째 원소를 접근하려면 \pyin{lst[n]}이 아닌 \pyin{lst[n - 1]}을, 마지막 원소를 접근하려면 \pyin{lst[len(lst)]}이 아닌 \pyin{lst[len(lst) - 1]}을 사용해야 합니다.
정리하자면, \pyin{lst[n]}은 \pyin{lst}의 $\texttt{n} + 1$번째 원소를 가리킵니다.

\subsubsection{임의의 원소들이 들어갈 수 있는 경우}
마지막으로는 리스트의 크기도 정해지지 않은 경우입니다.
물론, 이 방법은 원소들이나 개수가 정해진 경우에도 사용할 수 있는 가장 일반적인 방법입니다.
리스트의 \pyin{append}\index{\texttt{append}} \alt*{메소드}{method}\footnote{객체에 딸린 함수를 뜻하는 말입니다. 객체 지향 프로그래밍을 배울 때 알아볼 것입니다.}를 사용하는 방법입니다.
\begin{minted}{python}
numbers = []

for i in range(10):
    numbers.append((i + 1) * 3)
\end{minted}
처음에 아무것도 담지 않은 리스트 \pyin{[]}에다가, \pyin{append}를 통해 새로운 원소를 뒤에 하나씩 추가하는 방법입니다.
예를 들어, \pyin{a = [1, 2, 3]}일 때 \pyin{a.append(4)}를 수행하면 \pyin{a}는 \pyin{[1, 2, 3, 4]}가 됩니다.

\subsection{리스트의 원소와 인덱스}
리스트 \texttt{lst}의 원소 \pyin{lst[index]}에서 \pyin{index}는 \alt*{인덱스}{index}\footnote{Index의 복수형은 indices입니다.}라고 합니다.
인덱스는 \texttt{0}부터 \pyin{len(lst) - 1}까지의 값 이외에도, \pyin{-len(lst)}에서부터 \texttt{-1}까지 사용할 수 있습니다.
\pyin{lst[-1]}은 \pyin{lst[len(lst) - 1]}과 같으며, \pyin{lst[-len(lst)]}은 \pyin{lst[0]}과 같습니다.
이때 \pyin{len(lst)} 이상의 값이나 \texttt{-len(lst)} 미만의 값을 인덱스로 사용하면 \texttt{IndexError: list index out of range} 에러를 볼 수 있습니다.
복잡한 코드에서 주의를 기울이지 않으면 자주 보게 될 에러 메시지인데, 해당 메시지가 뜨면 \texttt{for}문의 리스트 인덱스를 다시 한 번씩 살펴보아야 합니다.

리스트는 굉장히 범용적인 \alt*{컨테이너}{container}입니다.
리스트는 정수 혹은 실수 뿐만이 아니라 문자열, 불리언, 나아가 다른 리스트들도 담을 수 있습니다.
또한 한 가지 타입의 원소들만이 아니라 여러 타입이 섞인 원소들도 한 리스트에 담을 수 있습니다.
즉, 아래와 같은 예시 모두 가능합니다.
\begin{minted}{python}
>>> type(["Mon", "Tue", "Wed", "Thu", "Fri", "Sat", "Sun"])
<type 'list'>
>>> type([True, None, [3, 2, 1], 3.14, "one"])
<type 'list'>
\end{minted}
이런 리스트를 \alt*{불균일}{heterogeneous}하다고 부릅니다.
하지만 이렇게 인덱스별로 다양한 타입의 값을 담고 있는 리스트는 다루기 힘들 뿐만이 아니라 (잘 짜여진 코드에서는) 많이 보기도 어렵습니다.

인덱스를 통해 리스트에 담긴 원소를 다룰 때에는 해당 원소에 대한 연산들을 그대로 사용할 수 있습니다:\footnote{인덱스로 접근한 이후에는 그 원소에 해당하는 변수를 사용하는 것과 다름이 없기 때문에 당연한 일입니다.}
\begin{minted}{python}
>>> numbers = [None] * 4
>>> numbers[0] = 3
>>> numbers[0]
3
>>> numbers[1]
None
>>> numbers[1] = numbers[0] * 2
>>> numbers[1]
6
>>> numbers[2] = numbers[1] - 7
>>> numbers[2]
-1
>>> numbers[3] = numbers[2] ** 2
>>> numbers[3]
1
>>> numbers[3] /= 2
>>> numbers[3]
0.5
>>> numbers[4] = 2 * numbers[3]
>>> numbers[4]
1.0
\end{minted}

리스트의 인덱스는 정수 타입이고 적절한 범위에 들어가는 값으로 계산된다면 어떤 식이든 사용할 수 있습니다:
\begin{minted}{python}
for i in range(4):
    print(numbers[(i + 2) % 4])
\end{minted}
위 예시는 \pyin{numbers[2]}, \pyin{numbers[3]}, \pyin{numbers[0]}과 \pyin{numbers[1]}을 차례대로 출력합니다.

\subsection{리스트의 조작}
\subsubsection{슬라이싱}
리스트에서 유용한 연산 중에는 \alt*{슬라이싱}{slicing}이 있습니다.
슬라이싱이란 리스트의 부분 리스트를 만드는 연산으로, 다음과 같이 사용할 수 있습니다:
\begin{minted}{python}
>>> l = [0, 1, 2, 3, 4, 5]
>>> l[1:4]
[1, 2, 3]
>>> l[0:3]
[0, 1, 2]
>>> l[:3]
[0, 1, 2]
>>> l[2:6]
[2, 3, 4, 5]
>>> l[2:]
[2, 3, 4, 5]
>>> l[:]
[0, 1, 2, 3, 4, 5]
>>> l[1:-1]
[1, 2, 3, 4]
\end{minted}

\texttt{l}이라는 리스트가 주어졌을 때 \pyin{l[i:j]}는 \pyin{l[i]}부터 \pyin{l[j - 1]}까지의 원소를 뽑아 새로운 리스트를 만듭니다.
만약 \texttt{i}가 생략된다면 첫 원소---인덱스 \verb/0/---부터 시작하며, \texttt{j}가 생략된다면 마지막 원소---인덱스 \verb/-1/---까지 포함합니다.
둘 다 생략하는 경우 같은 원소들을 담고 있는 리스트의 복사본을 만들게 됩니다.\footnote{하지만 한 단계만 복사하는 얕은 복사\myindex{얕은 복사}{shallow copy}입니다.}

원소들을 일정 간격으로 건너뛰면서 부분 리스트를 만들 때에도 슬라이싱을 사용할 수 있습니다.
이 경우 \pyin{l[i:j:k]}와 같이 슬라이싱을 하는데, \texttt{k}만큼 건너뛰면서 원소들을 추출합니다.
\pyin{range(i, j, k)}에 대응된다고 생각하면 됩니다.
사실 \pyin{range}는 리스트처럼 값을 순회할 수 있게 하는 값을 반환하는 메소드이기 때문입니다.
\begin{minted}{python}
>>> l = [0, 1, 2, 3, 4, 5, 6, 7, 8, 9]
>>> l[1:9:2]
[1, 3, 5, 7]
>>> l[:9:2]
[0, 2, 4, 6, 8]
>>> l[1::2]
[1, 3, 5, 7, 9]
>>> l[1:-1:2]
[1, 3, 5, 7]
>>> list(range(10))
[0, 1, 2, 3, 4, 5, 6, 7, 8, 9]
\end{minted}
위에서 볼 수 있듯 \pyin{l[i:j:k]}의 \texttt{i}, \texttt{j}, \texttt{k} 중 \texttt{i}와 \texttt{j}, 혹은 둘 다 생략할 수 있습니다.
\texttt{k}가 음수일 경우, 원소를 뒤에서부터 골라서 추출하게 됩니다.
특히 \texttt{k = -1}인 경우 \texttt{l[::-1]}과 같이 사용하면 \texttt{l}의 원소를 거꾸로 나열한 새로운 리스트를 만들게 됩니다.

\subsubsection{리스트의 비교와 포함 관계 확인}
두 리스트를 비교할 때는 다른 타입과 마찬가지로 \pyin{==}를 사용해 비교할 수 있습니다.
또한 특정 원소가 리스트에 해당하는지를 알아보려면 \pyin{in}을 사용하면 됩니다:
\begin{minted}{python}
>>> l = [0, 1, 2, 3, 4, 5, 6, 7, 8, 9]
>>> m = [0, 1, 2, 3, 4, 5, 6, 7, 8, 9]
>>> l == m
True
>>> 4 in l
True
>>> 10 in l
False
>>> [1, 4] in l
False
>>> [1, 4] in [[1, 4], 2, 3]
True
\end{minted}

\subsubsection{리스트의 연결}
두 리스트를 \alt*{연결}{concatenate}[concatenation]하고 싶은 경우 \texttt{+}를 사용하여 연결하고, $n$번 연결한 리스트를 만드려면 \texttt{*}를 사용하면 됩니다\footnote{얕은 복사\myindex{얕은 복사}{shallow copy}를 하므로 주의해야 합니다.}:
\begin{minted}{python}
>>> l = [0, 1, 2] + [3, 4, 5, 6, 7, 8, 9]
>>> m = [0, 1, 2, 3, 4, 5, 6, 7, 8, 9]
>>> l == m
True
>>> l = [1, 2] * 3
>>> l
[1, 2, 1, 2, 1, 2]
\end{minted}

\subsection{다차원 리스트}
다차원 리스트는 리스트들의 리스트입니다.
특히 \alt*{행렬}{matrix}을 만들 때 유용하게 사용할 수 있습니다.

행렬 $M$이
\[M = \begin{pmatrix}
    1 & 2 & 3 & 4 & 5 \\
    3 & 4 & 5 & 6 & 7
\end{pmatrix}\]
와 같을 때, 이는 다음과 같은 리스트 \verb/l/로 표현할 수 있습니다.
\begin{minted}{python}
l = [[1, 2, 3, 4, 5], [3, 4, 5, 6, 7]]
\end{minted}

만약 \texttt{l}의 행 크기를 알고 싶다면 \pyin{len(l[0])}을, 열 크기를 알고 싶다면 \pyin{len(l)}을 사용하면 됩니다.

각 원소를 접근할 때는, 1행 1열의 원소라면 \pyin{l[0][0]}, 1행 2열의 원소라면 \pyin{l[0][1]}, 2행 3열의 원소를 접근하고 싶다면 \pyin{l[1][2]}를 사용하면 됩니다.
일반적으로 \texttt{i}행 \texttt{j}열의 원소는 \pyin{l[i - 1][j - 1]}로 접근할 수 있습니다.

다차원 리스트의 경우에도 미리 원소를 지정하지 않고 \pyin{None}으로 초기화하여 리스트를 생성할 수 있습니다.
행 크기 \texttt{height}, 열 크기 \texttt{width}인 \pyin{None}으로 초기화된 리스트는 다음과 같이 생성할 수 있습니다.
아래 예시에는 \texttt{i}행 \texttt{j}열의 값을 \texttt{i + 2*j + 1}로 지정하는 과정까지 담고 있습니다.
\begin{minted}{python}
height = 3
width = 4
table = [[None] * width for i in range(height)]
# or table = [[None for j in range(width)] for i in range(height)]
# or just table = [[None for _ in range(width)] for _ in range(height)]

for i in range(height):
    for j in range(width):
        table[i][j] = i + 2 * j + 1
\end{minted}
위 과정을 통해
\[
  \begin{pmatrix}
    1 & 3 & 5 & 7\\
    2 & 4 & 6 & 8\\
    3 & 5 & 7 & 9
  \end{pmatrix}
\]
에 대응되는 리스트 \pyin{[[1, 3, 5, 7], [2, 4, 6, 8], [3, 5, 7, 9]]}를 얻을 수 있습니다.

나아가 임의의 차원을 가지는 리스트도 만들 수 있습니다.
3차까지는 머리 속에서 시각화할 수 있는데요, 길이, 너비, 높이를 가지는 격자에 값이 놓여 있다고 생각하시면 됩니다.\footnote{4차도 시간에 따라 3차 격자에서 값들이 바뀌는 것으로 시각화할 수 있습니다.}
이 경우 삼중 루프를 써서 값을 채워 넣을 수 있습니다:
\begin{minted}{python}
height = 3
width = 4
depth = 2
table = [[[None] * width for j in range(height)] for i in range(depth)]
# or table = [[[None for _ in range(width)] for _ in range(height)] for _ in range(depth)]

for i in range(depth):
    for j in range(height):
        for k in range(width):
            table[i][j][k] = i + 2 * j + k + 1
\end{minted}

\subsection{알아두면 유용한 리스트 내장 함수}
파이썬에서는 리스트를 조작할 수 있는 다양한 내장 함수들을 미리 준비해 두었는데요, 다음의 연산들이 있습니다:
\begin{itemize}
\item \pyin{sum(l)}: \texttt{l}의 모든 원소의 합을 반환합니다.
\item \pyin{min(l)}: \texttt{l}의 최소원을 반환합니다.
\item \pyin{max(l)}: \texttt{l}의 최대원을 반환합니다.
\item \pyin{l.append(x)}: \texttt{l}에 \texttt{x}를 맨 뒤에 추가합니다. \pyin{None}을 반환합니다.
\item \pyin{l.count(x)}: \texttt{l}에 포함된 \texttt{x}의 개수를 반환합니다.
\item \pyin{l.insert(i, x)}: \texttt{l}의 \texttt{i}번째 인덱스에 \texttt{x}를 추가합니다. \pyin{None}을 반환합니다.
\item \pyin{l.pop(i)}: \texttt{l}의 \texttt{i}번째 인덱스에 해당하는 원소를 없애고 해당 값을 반환합니다.
\item \pyin{l.remove(x)}: \texttt{l}의 첫 번째 \texttt{x}를 없앱니다.
\item \pyin{l.reverse()}: \texttt{l}을 뒤집습니다.
\item \pyin{l.sort()}: \texttt{l}의 원소들을 정렬하며 \pyin{None}을 반환합니다.
\item \pyin{reversed(l)}: \texttt{l}의 뒤집힌 버전을 반환합니다.\footnote{이렇게 반환된 값은 사실 리스트가 아니라 \alt*{이터레이터}{iterator}인데, \pyin{list(reversed(l))}과 같이 \texttt{list} 함수로 변환하여 리스트로 변환할 수 있습니다.} \texttt{l}은 그대로 남아 있습니다.
\item \pyin{sorted(l)}: \texttt{l}의 정렬된 버전을 반환합니다. \texttt{l}은 그대로 남아 있습니다.
\end{itemize}
반환값이 \pyin{None}인 메소드와 그렇지 않은 것들을 잘 구분해야 합니다.
예컨대 \pyin{l.sort()}를 하면 \texttt{l} 자체가 정렬이 되지만, \pyin{m = sorted(l)}을 하면 \verb/m/에 \verb/l/의 정렬된 버전이 대입되고 \verb/l/은 변화가 없습니다.

아래 예시를 통해 확인해봅시다:
\begin{minted}{python}
>>> l = [0, 1, 2]
>>> m = sum(l)
>>> m
3
>>> m = min(l)
>>> m
0
>>> m = max(l)
2
>>> m = l.append('hi')
>>> m
None
>>> l
[0, 1, 2, 'hi']
>>> m = l.count(0)
>>> m
1
>>> m = l.insert(1, 3}
>>> m
None
>>> l
[0, 3, 1, 2, 'hi']
>>> m = l.pop(3)
>>> m
2
>>> l
[0, 3, 1, 'hi']
>>> m = l.remove(0)
>>> m
None
>>> l
[3, 1, 'hi']
>>> m = l.reverse()
>>> m
None
>>> l
['hi', 1, 3]
>>> m = list(reversed(l))
>>> m
[3, 1, 'hi']
>>> m = l.sort()
>>> m
None
>>> l
[1, 3, 'hi']
\end{minted}

\section{문자열}
지금까지 문자열은 단순히 어떤 문구를 출력하기 위해 사용하는 용도였다면, 이제는 문자열을 가지고 연산하는 법을 알아볼 것입니다.

문자열은 리스트와 비슷하게 다룰 수 있습니다.
리스트 \texttt{l}의 \texttt{i}번째 원소를 접근하기 위하여 \pyin{l[i - 1]}을 사용하였다면, 어떠한 문자열 \texttt{s}의 \texttt{i}번째 문자를 접근하기 위하여 동일한 방식으로 \pyin{s[i - 1]}을 사용할 수 있습니다.

슬라이싱도 동일한 문법을 따릅니다.
나아가 두 문자열을 연결하기 위해서는 \texttt{+}를, 반복하기 위하여 \texttt{*}에 수를 곱하는 방식으로 연산하는 것까지 동일합니다.

또한 부등호를 통해 사전식 배열을 하였을 때 어떤 문자열이 먼저 나오는지를 판단할 수 있습니다.
\pyin{in}과 \pyin{not in}을 통해 어떤 문자열이 다른 문자열의 부분 문자열이 되는지도 판단할 수 있습니다.
\begin{minted}{python}
>>> s = "Python"
>>> s[2]
't'
>>> s[1:4]
'yth'
>>> s * 3
'PythonPythonPython'
>>> s = s + "PYTHON"
>>> s
'PythonPYTHON'
>>> "Python" > "PYTHON"
True
>>> 'thon' in s
True
\end{minted}

\subsection{문자열의 불변성}
리스트와의 차이라면 리스트는 \alt*{가변}{mutable} 객체이고, 문자열은 \alt*{불변}{immutable} 객체라는 것입니다.
이에 따라 한 문자열을 ``조작''하고 싶다면 새로운 문자열을 만들어야지, 기존의 문자열을 수정하는 것은 불가능합니다.

그렇지만 \pyin{+=}이나 \pyin{*=}을 사용하는 것은 가능합니다.
이는 \pyin{s += "a"}는 단순히 \texttt{s = s + "a"}를 축약한 것으로, \verb/s/의 값을 수정하는 것이 아니라 \verb/s/에 새로운 값을 대입하는 것이기 때문입니다.
\begin{minted}{python}
>>> s = "Python"
>>> s[0] = "p"
Traceback (most recent call last):
  File "<stdin>", line 1, in <module>
TypeError: 'str' object does not support item assignment
>>> s += "PYTHON"
>>> s
"PythonPYTHON"
\end{minted}

\subsection{알아두면 유용한 문자열 내장 함수들}
문자열을 더 편리하게 다루기 위한 다양한 내장 함수들도 준비되어 있습니다.

문자열 \texttt{s}와 \texttt{t}가 주어졌을 때,
\begin{itemize}
\item \pyin{s.upper()}: \texttt{s}의 모든 문자를 대문자로 바꾸어 반환합니다.
\item \pyin{s.lower()}: \texttt{s}의 모든 문자를 소문자로 바꾸어 반환합니다.
\item \pyin{s.capitalize()}: \texttt{s}의 첫 문자는 대문자로, 나머지는 소문자로 바꾸어 반환합니다.
\item \pyin{s.isupper()}: \texttt{s}의 모든 문자가 대문자라면 \pyin{True}를, 아니라면 \pyin{False}를 반환합니다.
\item \pyin{s.islower()}: \texttt{s}의 모든 문자가 소문자라면 \pyin{True}를, 아니라면 \pyin{False}를 반환합니다.
\item \pyin{s.isdigit()}: \texttt{s}의 모든 문자가 숫자라면 \pyin{True}를, 아니라면 \pyin{False}를 반환합니다.
\item \pyin{s.split()}: \texttt{s}에서 띄어쓰기 문자로 구분된 구절들을 원소로 가지는 리스트를 반환합니다.
\item \pyin{s.find(t)}: \texttt{s}에 \texttt{t}가 포함되는 경우, \texttt{t}가 나타나는 첫 위치의 인덱스를 반환하며, 포함되지 않는 경우 -1을 반환합니다.
\end{itemize}

리스트와는 다르게 문자열을 수정하는 메소드는 없는 것을 확인할 수 있습니다.
전에 언급하였던 것처럼 문자열은 불변 객체이기 때문입니다.
\begin{minted}{python}
>>> s = "pYtHon".upper()
>>> s
'PYTHON'
>>> s = "pYtHon".lower()
>>> s
'python'
>>> s = "pYtHon".capitalize()
>>> s
'Python'
>>> "PYTHON12".isupper()
True
>>> "pyThon12".islower()
False
>>> "1242".isdigit()
True
>>> l = "Hello World!".split()
>>> l
['Hello', 'World!']
>>> s = "Python"
>>> s.find("thon")
2
\end{minted}

\section{카운터 패턴과 휴보}
리스트를 배웠기 때문에 조금 더 자유로운 반복문의 활용이 가능합니다.
이번 절에서는 카운터 패턴에 대해 집중적으로 살펴볼 것이며, \texttt{hubo.cs1robots} 패키지를 통해 ``휴보''를 조작할 것입니다.\footnote{이는 카이스트에서 프로그래밍을 가르치기 위해 만들어진 패키지입니다.}

카운터 패턴은 리스트에서 어떤 조건을 만족하는 원소의 개수를 구하는데 사용합니다.
아래는 \texttt{numbers}에서 7의 배수를 세는 함수 \pyin{count_seven}입니다.
\begin{minted}{python}
def count_seven(numbers):
    cnt = 0
    for i in range(len(numbers)):
        if numbers[i] % 7 == 0:
            cnt += 1
    return cnt

num = [1, 7, 2, 4, 3, 5, 34, 12, 42, 26]
print(count_seven(num))  # 2
\end{minted}
\texttt{for}문을 통해 각 원소에 대해 반복적인 실행을 표현하고, 내부에 \texttt{if}문을 두어 특정 조건을 만족하면 개수를 세는 패턴이 카운터 패턴입니다.

또한 \texttt{for}문으로 휴보의 움직임을 제어해볼 것입니다~(그림~\ref{fig:lecture4:example}).
\begin{figure}[H]
\centering
\includegraphics[width=0.5\linewidth]{"./lectures/lecture4_example"}
\caption{휴보를 원하는대로 조작하는 법을 알아볼 것입니다.}\label{fig:lecture4:example}
\end{figure}
휴보는 전진, 좌회전, 우회전에 해당하는 \pyin{move()}, \pyin{turn_left()}, \pyin{turn_right()} 등의 동작을 취할 수 있습니다.

\verb/hubo/ 패키지는 pip\index{pip}으로 설치할 수 있습니다.
pip을 사용하면 PyPI\myindex{파이썬 패키지 인덱스}{Python Package Index, PyPI}에 있는 패키지들을 손쉽게 설치할 수 있습니다:
\begin{minted}{shell}
pip install hubo
\end{minted}

\subsection{카운터 패턴}
카운터 패턴은 다음의 같은 형식을 따릅니다.
\begin{minted}{python}
cnt = 0
for i in range(len(lst)):
    if cond(lst[i]):
        cnt += 1
\end{minted}
리스트 \texttt{lst}가 주어졌을 때, 해당 리스트의 모든 원소에 대해(줄 2) 어떤 조건 \texttt{cond}을 \pyin{lst[i]}가 만족한다면(줄 3) \texttt{cnt}의 값을 1씩 더하는(줄 4) 것입니다.

위에서 보았던 7의 배수의 개수를 세는 예시도 마찬가지입니다.
\texttt{numbers[i] \% 7 == 0}의 조건이었는데, 이를 사용해 7의 배수이면 \pyin{True}를, 아니면 \pyin{False}를 반환하는 함수 \verb/is_mult_7/로 만든다면 정의한다면 \pyin{is_mult_7(lst[i])}의 형태로 쓸 수 있습니다.
불리언 함수를 배울 때 언급하였듯, 조건이 복잡한 경우에는 함수를 정의하는 방식으로 코드를 작성해야 합니다.
소수의 개수를 세는 카운터 패턴의 경우, 소수를 판별하는 불리언 함수 \pyin{is_prime}으로 \pyin{if is_prime(lst[i]):}와 같이 사용하면 됩니다.
주어진 리스트에서 소수의 개수를 세는 함수가 아래 예시에 나와 있습니다.
\begin{minted}{python}
def is_prime(p):
    """A naive implementation"""
    for i in range(2, p // 2):
        if p % i == 0:
            return False
    return True

def count_primes(numbers):
    cnt = 0
    for i in range(len(numbers)):
        if is_prime(numbers[i]):
            cnt += 1
    return cnt

num = [217, 287, 181, 143, 163, 319, 233, 399, 203]
print(count_primes(num))  # 3
\end{minted}

\subsection{휴보}
\begin{figure}[htbp]
\centering
\includegraphics[width=0.5\linewidth]{"./lectures/lecture4_emptyworld"}
\caption{휴보가 움직일 빈 세상입니다.}\label{fig:lecture4:emptyworld}
\end{figure}

지금까지 배운 함수, 조건문, 루프, 리스트 등의 개념을 모두 활용하여 휴보를 움직이게 할 것입니다.
시작하기 위해 다음과 같은 코드를 작성해봅시다:
\begin{minted}{python}
from hubo.cs1robots import *
create_world()
\end{minted}
이를 실행시키면 그림~\ref{fig:lecture4:emptyworld}\와 같은 창이 뜹니다.
이것이 앞으로 휴보가 움직일 세상입니다.
세상은 기본적으로 가로 세로로 10칸의 격자입니다.

이제 \pyin{hubo = Robot()}을 실행하면, 그림~\ref{fig:lecture4:hubo}\와 같이 휴보가 위치 $(1, 1)$에 생성됩니다.
\begin{figure}[htbp]
\centering
\includegraphics[width=0.5\linewidth]{"./lectures/lecture4_hubo"}
\caption{휴보가 생성된 모습입니다.}\label{fig:lecture4:hubo}
\end{figure}

휴보가 한 칸 전진하기 위해서는 \pyin{hubo.move()}를 실행하고, 90도 좌회전하기 위해서는 \pyin{hubo.turn_left()}를, 90도 우회전하기 위해서는 \pyin{hubo.turn_right()}를 실행하면 됩니다.
그림~\ref{fig:lecture4:huboleftmove}에 휴보를 좌회전시킨 후 한 칸 전진한 후의 모습이 나와 있습니다.
\begin{figure}[htbp]
\centering
\includegraphics[width=0.5\linewidth]{"./lectures/lecture4_huboleftmove"}
\caption{휴보가 좌회전한 후 한 칸 전진한 모습입니다.}\label{fig:lecture4:huboleftmove}
\end{figure}

휴보가 이동한 경로를 가시화하고 싶다면 \pyin{hubo.set_trace("blue")}와 같이 \pyin{set_trace} 메소드에 인자로 색상을 문자열로 입력하면 됩니다.
또한 휴보가 느리게 움직이도록 하려면 \pyin{hubo.set_pause(0.5)}와 같이 매 명령마다 정지할 시간을 초 단위로 넘겨주면 됩니다.

휴보가 한 변의 길이가 2인 정사각형의 경로를 따라 이동하는 코드가 아래에 나타나 있습니다.
\begin{minted}{python}
from hubo.cs1robots import *
create_world()

hubo = Robot()
hubo.set_trace("blue")
hubo.set_pause(0.5)

def move_square():
    for i in range(4):
        hubo.move()
        hubo.move()
        hubo.turn_left()

move_square()
\end{minted}
이를 실행한 결과는 그림~\ref{fig:lecture4:hubosquare}입니다.

\begin{figure}[htbp]
\centering
\includegraphics[width=0.5\linewidth]{"./lectures/lecture4_hubosquare"}
\caption{휴보가 한 변의 길이가 2인 정사각형의 경로를 따라 이동한 모습입니다.}\label{fig:lecture4:hubosquare}
\end{figure}

\begin{figure}[htbp]
\centering
\includegraphics[width=0.5\linewidth]{"./lectures/lecture4_hubostairs"}
\caption{계단형 지형입니다.}\label{fig:lecture4:hubostairs}
\end{figure}

세상에는 벽도 추가할 수 있습니다.
그림~\ref{fig:lecture4:hubostairs}에 계단형 지형을 만든 세상을 볼 수 있습니다.
여기서 나타난 움직임을 구현하는 코드는 아래와 같습니다.
\begin{minted}{python}
hubo.move()
for _ in range(4):
    hubo.turn_left()
    hubo.move()
    hubo.turn_right()
    hubo.move()
    hubo.move()
\end{minted}
\verb/fot/문에서 사용한 변수 이름이 \verb/_/인 것은, 해당 값을 실제로 사용하지 않기 때문에 저렇게 놓은 것입니다.
기억한다면 \verb/_/는 가능한 변수 이름에 해당하고, 파이썬 프로그래머들은 보통 사용되지 않는 변수 이름으로 \verb/_/를 자주 사용합니다.

\section{예제}
\begin{enumerate}
\item 리스트의 원소들의 제곱의 합을 반환하는 함수 \verb/sum_squares/를 작성하세요.
\begin{minted}{python}
def sum_squares(a):
    # Add here!

print(sum_squares([3, 5, 4]))  # 50
print(sum_squares([2, 5, 4, 0, 1, -1, 5, 1]))  # 73
\end{minted}

\item 길이 \verb/n/인 정수 리스트 \verb/a/와 정수 \verb/x/를 받아
\[
\sum_{i = 0}^{\texttt{n} - 1} \texttt{a[}i\texttt{]} \cdot \texttt{x}^i
\]
을 계산하는 함수 \pyin{compute_polynomial(a, x)}를 작성하세요.
\begin{minted}{python}
def compute_polynomial(a, x):
    # Add here!

print(compute_polynomial([3, 5, 4], 5))  # 128
print(compute_polynomial([2, 0, 4, 0, 1, -1, 5, 1], 3))  # 5708
\end{minted}

\item 리스트 \verb/a/의 조화 평균을 계산하는 함수 \verb/harmonic_mean/을 작성하세요.
  $a_1, a_2, \dots, a_n$의 조화 평균은
\[
\frac{n}{\sum_{i = 1}^{n} \frac{1}{a_i}}
\]
와 같이 계산됩니다.

또, \verb/a/의 기하 평균을 계산하는 함수 \verb/geometric_mean/을 작성하세요.
$a_1, a_2, \dots, a_n$의 기하 평균은
\[
\sqrt[n]{\prod_{i = 1}^{n} a_i}
\]
입니다.

\begin{minted}{python}
def harmonic_mean(a):
    # Add here!

def geometric_mean(a):
    # Add here!

numbers = [2, 4, 3, 10, 7, 2, 5, 6]
print(harmonic_mean(numbers))  # 3.64820846906
print(geometric_mean(numbers))  # 4.22116731332
\end{minted}

\item 반복문을 사용해서 주어진 리스트 \verb/a/의 뒤집힌 리스트를 반환하는 함수 \verb/reversed(a)/을 작성하세요.
\begin{minted}{python}
def reversed(a):
    n = len(a)
    b = [None] * n
    # Add here!

print(reversed([3, 1, 5, 2, 4]))  # [4, 2, 5, 1, 3]
print(reversed([7, 6, 3, 1, 5, 8, 2, 4]))  # [4, 2, 8, 1, 3, 6, 7]
\end{minted}

\item 첫 \verb/n/개의 피보나치 수들의 리스트를 반환하는 \pyin{fibonacci(n)}을 작성하세요.
  \verb/n/은 양의 정수라고 가정하고, 피보나치 수열의 첫 두 값은 1입니다.
\begin{minted}{python}
def fibonacci(n):
    # Add here!

print(fibonacci(10))  # [1, 1, 2, 3, 5, 8, 13, 21, 34, 55]
\end{minted}

\item 좌표 평면 상의 점들의 리스트 \verb/points/가 주어졌을 때, 이 점들이 구성하는 다각형의 면적을 계산하는 함수 \pyin{area(points)}을 작성하세요.
\texttt{points}는 \pyin{[3, 4]}처럼 $x$ 좌표와 $y$ 좌표의 성분이 리스트로 표현된 원소들의 리스트입니다.
예를 들어 그림~\ref{fig:lecture4q6}을 나타내는 \texttt{points}는 \pyin{[[3, 1], [6, 3], [4, 4], [7, 6], [2, 7], [0, 5], [2, 3], [1, 2]]}입니다.
\begin{figure}[H]
\centering
\includegraphics[width=0.5\linewidth]{"./lectures/lecture4_q6"}
\caption{문제 6의 예시 다각형입니다.}\label{fig:lecture4q6}
\end{figure}

일반적으로 점 $(x_i, y_i)$ ($i = 0, 1, \dots, n - 1$)들이 구성하는 다각형의 면적은 다음의 ``신발끈'' 공식으로 계산할 수 있습니다:
\[
\frac 12 \left| \sum_{i = 0}^{n - 1} x_i \left(y_{(i + 1) \mod n} - y_{(i - 1) \mod n}\right)\right|
\]

또, 이 점들이 구성하는 다각형의 둘레를 계산하는 \pyin{perimeter(points)}도 작성하세요.

\verb/area/와 \verb/perimeter/를 작성하기 위해 다른 함수들을 작성할 수 있습니다.

\begin{minted}{python}
# Add here!

def area(points):
    # Add here!

def perimeter(points):
    # Add here!

points = [[3, 1], [6, 3], [4, 4], [7, 6], [2, 7], [0, 5], [2, 3], [1, 2]]
print(area(points))  # 22.0
print(perimeter(points))  # 23.8533258314
\end{minted}

\item 리스트 \verb/a/의 최소원을 구하는 함수 \pyin{find_min(a)}을 작성하세요.
\begin{minted}{python}
def find_min(a):
    # Add here!

print(find_min([7.2, 5, 21, -1, 4, 0.4]))  # -1
\end{minted}

\item \verb/points/의 가장 가까운 두 점의 길이를 반환하는 함수 \pyin{closest_pair(points)}을 작성하세요.
\begin{minted}{python}
# Add here!

print(closest_pair([[4, -4], [7, 5], [2, 1], [-2, -1], [-3, 5]]))  # 4.472135955
\end{minted}

\item 리스트 \verb/numbers/가 주어졌을 때, \verb/lower/ 이상 \verb/upper/ 이하의 값들의 개수를 세는 함수 \pyin{count_range(numbers, lower, upper)}를 작성하세요.
\begin{minted}{python}
def count_range(numbers, lower, upper):
    # Add here!

print(count_range([8, 9, 10, 2, 4, 5, 9, 7, 2, 3, 7], 3, 7))  # 5
\end{minted}

\item $x^2 + y^2 \le \texttt{radius}^2$을 만족하는 점 $(x, y)$들의 개수를 세는 함수 \pyin{count_within_circle(points, radius)}를 작성하세요.
\begin{minted}{python}
def count_within_circle(points, radius):
    # Add here!

points = [[2, 1], [7, 5], [-5, 2], [-3, 5], [-7, 4], [-2, -1], [-2, -4], [-4, -2], [-6, -4], [4, -4], [6, -2]]
print(count_within_circle(points, 3))  # 2
print(count_within_circle(points, 5))  # 4
print(count_within_circle(points, 8))  # 9
\end{minted}

\item \verb/top/, \verb/bottom/, \verb/left/, \verb/right/으로 표현되는 직사각형의 내부 혹은 경계 위에 $(\texttt{x}, \texttt{y})$가 포함되는지 확인하는 함수 \pyin{within_rect(top, bottom, left, right, x, y)}를 작성하세요.
  예를 들어서 $(\texttt{top}, \texttt{bottom}, \texttt{left}, \texttt{right}) = (2, -4, -5, 6)$으로 표현되는 직사각형은 그림~\ref{fig:lecture4q11}\와 같습니다.
\begin{figure}[H]
\centering
\includegraphics[width=0.5\linewidth]{"./lectures/lecture4_q11"}
\caption{문제 11의 직사각형입니다.}\label{fig:lecture4q11}
\end{figure}
\begin{minted}{python}
def within_rect(top, bottom, left, right, x, y):
    # Add here!

print(within_rect(2, -4, -5, 6, -5, 2))  # True
print(within_rect(2, -4, -5, 6, 6, -1))  # True
print(within_rect(2, -4, -5, 6, 0, 1))  # True
print(within_rect(2, -4, -5, 6, -6, 0))  # False
print(within_rect(2, -4, -5, 6, 0, 3))  # False
\end{minted}

이제 위의 \verb/within_rect/ 함수를 사용해서 \verb/points/의 점들 중 직사각형에 포함되는 (경계 포함) 것들의 개수를 세는 함수 \pyin{count_within_rect(top, bottom, left, right, points)}를 작성하세요.
\begin{minted}{python}
def count_within_rect(top, bottom, left, right, points):
    # Add here!

points = [[2, 1], [7, 5], [-5, 2], [-3, 5], [-7, 4], [-2, -1], [-2, -4], [-4, -2], [-6, -4], [4, -4], [6, -2]]
print(count_within_rect(2,-4,-5,6, points))  # 7
\end{minted}

\item \verb/years/에서 윤년의 개수를 세는 함수 \pyin{count_leap_year(numbers)}를 작성하세요.
\begin{minted}{python}
def count_leap_years(numbers):
    # Add here!

print(count_leap_years([2008, 2011, 2012, 2000]))  # 3
print(count_leap_years([2100, 2300, 2400, 2200]))  # 1
\end{minted}

\item \verb/numbers/에서 합성수의 개수를 세는 함수 \pyin{count_composite(numbers)}를 작성하세요.
\begin{minted}{python}
# Add here!

def count_composite(numbers):
    # Add here!

numbers = [217, 287, 181, 143, 163, 319, 233, 399, 203]
print(count_composite(numbers))  # 1
\end{minted}
\end{enumerate}
\end{document}
