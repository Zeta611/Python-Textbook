\documentclass[../main.tex]{subfiles}

\begin{document}

\subsection{Lists리스트}
지난 시간에 반복문을 통해서 단순 반복 작업을 Python으로 수행하는 법에 대해서 알아보았습니다.
하지만 지난 시간에서 \texttt{for} 문의 활용은 단순히 값을 대입하여, 해당 값을 그대로 사용하는 것에 그쳤습니다.
만약 \texttt{i}를 4로 대입하는 경우에는 4를 그대로 사용하여 일반항을 계산하는 작업을 하였던 것입니다.
이번 시간에 list리스트 자료형을 배우게 된다면, 임의의 주어진 수열에 대해서 연산을 수행할 수 있습니다.
리스트는 일종의 수열로 생각할 수 있는데, \texttt{for} 문과 리스트를 사용하면 수열의 첨자를 통해 일반적인 계산을 수행할 수 있습니다.

지금까지는 10개의 변수를 받아 합을 반환하는 함수를 작성하기 위해서는 아래와 같은 코드를 작성했어야 합니다.
\begin{minted}[mathescape,
               linenos,
               breaklines,
               numbersep=5pt,
               frame=lines,
               framesep=2mm]{python}
def sum_ten(n1, n2, n3, n4, n5, n6, n7, n8, n9, n10):
    summ = 0
    summ += n1
    summ += n2
    ...
    summ += n10
    return summ
\end{minted}
심지어 \texttt{for} 문을 활용할 수도 없습니다.
다음과 같은 문법은 허용되지 않기 때문입니다.
\begin{minted}[mathescape,
               linenos,
               breaklines,
               numbersep=5pt,
               frame=lines,
               framesep=2mm]{python}
def sum_ten(n1, n2, n3, n4, n5, n6, n7, n8, n9, n10):
    summ = 0
    for i in range(1, 11):
        summ += ni
    return summ
\end{minted}
위에서 \texttt{ni}는 단지 새로운 변수 \texttt{ni}라는 이름을 가진 변수에 불과합니다.
그러나 리스트를 사용한다면, 단지 10개의 변수가 아니라 임의의 개수의 변수를 받아서--엄밀하게는 임의의 개수의 변수를 담고 있는 하나의 리스트형 변수를 받아서--계산을 할 수 있습니다.
\begin{minted}[mathescape,
               linenos,
               breaklines,
               numbersep=5pt,
               frame=lines,
               framesep=2mm]{python}
def sum_arb(numbers):
    summ = 0
    for i in range(len(numbers)):
        summ += numbers[i]
    return summ
\end{minted}

요약하자면, 묶음의 데이터를 한 번에 처리할 수 있는 방법이 리스트와 \texttt{for} 문을 활용하는 것입니다.
이제 리스트의 정의와, 이를 어떻게 다룰 수 있을지 자세히 알아봅니다.

\subsubsection{리스트의 생성}
리스트의 생성은 세 가지 경우로 나눌 수 있습니다.
첫 번째 경우는 리스트에 어떤 원소가 들어갈 것인지 이미 정해진 경우입니다.
이러한 경우, 직접 원소를 입력하여 리스트를 생성할 수 있습니다.
\begin{minted}[mathescape,
               linenos,
               breaklines,
               numbersep=5pt,
               frame=lines,
               framesep=2mm]{python}
numbers = [3, 1, 4, 1, 5, 9, 2, 6]
\end{minted}
리스트의 값은 나중에 수정할 수 있기 때문에 처음에 채워 넣은 값을 initial values초기값이라고 합니다.
하지만 이렇게 값을 입력하는 것은 리스트가 담고 있는 원소의 개수가 적을 때만 가능합니다.
그렇지만 값이 불규칙할 경우에는 직접 입력할 수 밖에 없습니다.

리스트에 담을 값이 규칙적이면서, 크기가 정해져 있다면 값을 초기화하지 않은 상태에서 일정 크기를 가진 리스트를 생성할 수 있습니다.
아래는 크기 10의 리스트를 생성한 것입니다.
\begin{minted}[mathescape,
               linenos,
               breaklines,
               numbersep=5pt,
               frame=lines,
               framesep=2mm]{python}
numbers = [None] * 10
\end{minted}
\texttt{[None] * 10}은 \texttt{[None, None, None, None, None, None, None, None, None, None]}과 동일합니다.
위에서와 같이 \texttt{None} 말고 정수형인 \texttt{-1}이나 문자열인 \texttt{"empty"} 등으로도 채워넣을 수 있습니다.
단, \texttt{0}과 같은 값은 실제로 값을 채워 넣은 유의미한 \texttt{0}과 혼동될 수 있으니 사용하지 않는 것이 바람직합니다.
무의미한 값으로 리스트를 채웠으니 값을 채워야 합니다.
이는 \texttt{for} 문으로 해결할 수 있습니다.
아래는 첫 10개의 3의 배수를 넣는 과정입니다.
\begin{minted}[mathescape,
               linenos,
               breaklines,
               numbersep=5pt,
               frame=lines,
               framesep=2mm]{python}
numbers = [None] * 10

for i in range(len(numbers)):
    numbers[i] = (i + 1) * 3
\end{minted}
\texttt{len($\cdot$)} 함수가 처음 등장했는데, 이는 리스트의 길이를 반환합니다.
위의 예시에서 \texttt{len(numbers)}는 \texttt{10}을 반환합니다.
코드를 보다 일반적이고 명료하게 나타내기 위하여 리스트의 길이를 숫자 \texttt{10}과 같이 직접 치는 것보다는 \texttt{len($\cdot$)}을 사용하는 것이 좋습니다.
또한 리스트의 원소를 접근하는 방식은 \texttt{리스트\_명[원소\_첨자]}입니다.
\texttt{원소\_첨자}는 \texttt{0}에서 \texttt{len(리스트\_명) - 1}까지의 값을 가집니다.
즉, 리스트의 첫 원소를 호출하려면 \texttt{리스트\_명[1]}이 아닌 \texttt{리스트\_명[0]}을, \texttt{n}번째 원소를 호출하려면 \texttt{리스트\_명[n]}이 아닌 \texttt{리스트\_명[n - 1]}을, 마지막 원소를 호출하려면 \texttt{리스트\_명[len(리스트\_명)]}이 아닌 \texttt{리스트\_명[len(리스트\_명) - 1]}을 사용해야 합니다.
정리하자면, \texttt{리스트\_명[n]}은 \texttt{리스트\_명}의 $\texttt{n} + 1$번째 원소를 가리킵니다.

마지막으로는 리스트의 크기도 정해지지 않은 경우입니다.
물론, 위 두 경우처럼 개수나 원소가 정해진 경우에도 사용할 수 있는 일반적인 방법입니다.
리스트의 \texttt{append($\cdot$)} method메소드\footnote{객체에 관련된 함수를 뜻하는 말입니다. 객체 지향 프로그래밍을 배울 때 알아볼 것입니다. 지금으로는 단순히 어떤 변수에 딸린 함수라고 생각하면 됩니다.}를 사용하는 것입니다.
\begin{minted}[mathescape,
               linenos,
               breaklines,
               numbersep=5pt,
               frame=lines,
               framesep=2mm]{python}
numbers = []

for i in range(len(numbers)):
    numbers.append((i + 1) * 3)
\end{minted}
처음에 아무것도 담지 않은 리스트 \texttt{[]}에다가, \texttt{.append($\cdot$)}를 통해 새로운 원소를 뒤에 하나씩 추가하는 방법입니다.
예컨대, \texttt{a = [1, 2, 3]}일 때 \texttt{a.append(4)}를 수행하면 \texttt{a}는 \texttt{[1, 2, 3, 4]}가 됩니다.
간단히 ``리스트 버전의 \texttt{+=} 연산''이라고 생각하면 됩니다.

\subsubsection{리스트의 elements원소와 indices인덱스}
리스트 \texttt{리스트\_명}의 element원소 \texttt{리스트\_명[원소\_첨자]}에서 \texttt{원소\_첨자}는 index인덱스\footnote{복수형은 indices입니다.}라고 합니다.
지금까지는 수열처럼 첨자라고 했는데, 앞으로는 인덱스라고 부릅니다.
인덱스는 \texttt{0}부터 \texttt{len(리스트\_명) - 1}까지의 값 이외에도, \texttt{-len(리스트\_명)}에서부터 \texttt{-1}까지 사용할 수 있습니다.
\texttt{리스트\_명[-1]}은 \texttt{리스트\_명[len(리스트\_명) - 1]}과 같으며, \texttt{리스트\_명[-len(리스트\_명)]}은 \texttt{리스트\_명[0]}과 같습니다.
Modulo모듈로 \texttt{len(리스트\_명)}로 생각할 수 있는데, \texttt{len(리스트\_명)} 이상의 값이나 \texttt{-len(리스트\_명)} 미만의 값은 인덱스로 사용하면 \texttt{IndexError: list index out of range} 에러를 볼 수 있습니다.
복잡한 코드에서 주의를 기울이지 않으면 자주 보게 될 에러 메시지인데, 해당 메시지가 뜨면 \texttt{for} 문의 리스트 인덱스를 다시 한 번씩 살펴보아야 합니다.

리스트는 굉장히 범용적인 container용기라고 생각할 수 있습니다.
리스트는 정수형 혹은 실수형 수 뿐만이 아니라 문자열, 불리언, 나아가 리스트들을 담을 수 있습니다.
또한, 한 종류의 자료형이 아니라 여러 종류의 자료형을 한 리스트에 담을 수 있습니다.
즉, 아래와 같은 예시 모두 가능합니다.
\begin{minted}[mathescape,
               linenos,
               breaklines,
               numbersep=5pt,
               frame=lines,
               framesep=2mm]{python}
>>> type(["Mon", "Tue", "Wed", "Thu", "Fri", "Sat", "Sun"])
<type 'list'>
>>> type([True, None, [3, 2, 1], 3.14, "one"])
<type 'list'>
\end{minted}
나아가 인덱스를 통해 리스트에 담긴 원소를 다룰 때는 해당 원소의 자료형처럼 그대로 다룰 수 있습니다.
\begin{minted}[mathescape,
               linenos,
               breaklines,
               numbersep=5pt,
               frame=lines,
               framesep=2mm]{python}
>>> numbers = [None] * 4
>>> numbers[0] = 3
>>> numbers[0]
3
>>> numbers[1]
None
>>> numbers[1] = numbers[0] * 2
>>> numbers[1]
6
>>> numbers[2] = numbers[1] - 7
>>> numbers[2]
-1
>>> numbers[3] = numbers[2]**2
>>> numbers[3]
1
>>> numbers[3] /= 2.
>>> numbers[3]
0.5
>>> numbers[4] = 2 * numbers[3]
>>> numbers[4]
1.0
\end{minted}

리스트의 인덱스는 정수형이라면 어떠한 표현을 사용하여도 문제가 되지 않습니다.
\begin{minted}[mathescape,
               linenos,
               breaklines,
               numbersep=5pt,
               frame=lines,
               framesep=2mm]{python}
for i in range(4):
    print numbers[(i + 2)%4]
\end{minted}
위 예시는 \texttt{numbers[2]}, \texttt{numbers[3]}, \texttt{numbers[0]}과 \texttt{numbers[1]}을 차례대로 출력합니다.

\subsubsection{리스트의 연산}
리스트에서 유용한 연산 중에는 slicing슬라이싱이 있습니다.
슬라이싱이란 리스트의 부분 리스트를 만드는 연산으로, 아래와 같이 사용할 수 있습니다.
\begin{minted}[mathescape,
               linenos,
               breaklines,
               numbersep=5pt,
               frame=lines,
               framesep=2mm]{python}
>>> l = [0, 1, 2, 3, 4, 5]
>>> l[1:4]
[1, 2, 3]
>>> l[0:3]
[0, 1, 2]
>>> l[:3]
[0, 1, 2]
>>> l[2:6]
[2, 3, 4, 5]
>>> l[2:]
[2, 3, 4, 5]
>>> l[:]
[0, 1, 2, 3, 4, 5]
>>> l[1:-1]
[1, 2, 3, 4]
\end{minted}

즉, \texttt{l}이라는 리스트가 주어졌을 때 \texttt{l[i:j]}는 \texttt{l[i]}부터 \texttt{l[j - 1]}까지의 원소를 뽑아서 새로운 리스트를 만듭니다.
만약 \texttt{i}가 생략된다면 첫 문자부터 시작하며, \texttt{j}가 생략된다면 마지막 문자까지 포함을 합니다.
둘 다 생략하는 경우 같은 원소들을 담고 있는 리스트의 복사본을 만들게 됩니다.

원소들을 특정 간격으로 띄워서 부분 리스트를 만들 때도 슬라이싱을 사용할 수 있습니다.
이 경우 \texttt{l[i:j:k]}와 같이 슬라이싱을 하는데, \texttt{k}만큼의 간격을 두고 원소들을 추출하게 됩니다.
\texttt{range(i, j, k)}와 동일하다고 생각하면 됩니다.
곧 알게 되겠지만 \texttt{range($\cdot$)}는 리스트를 호출하는 메소드이기 때문입니다.
\begin{minted}[mathescape,
               linenos,
               breaklines,
               numbersep=5pt,
               frame=lines,
               framesep=2mm]{python}
>>> l = [0, 1, 2, 3, 4, 5, 6, 7, 8, 9]
>>> l[1:9:2]
[1, 3, 5, 7]
>>> l[:9:2]
[0, 2, 4, 6, 8]
>>> l[1::2]
[1, 3, 5, 7, 9]
>>> l[1:-1:2]
[1, 3, 5, 7]
\end{minted}
위에서 볼 수 있듯이, \texttt{i}, \texttt{j}, \texttt{k}에서 \texttt{i}와 \texttt{j}, 혹은 둘 다 생략을 할 수 있습니다.
또한 \texttt{k}가 음수일 경우, 원소를 뒤에서부터 골라서 추출하게 됩니다.
특히 \texttt{k = -1}로 지정하여 \texttt{l[::-1]}과 같이 사용한다면 \texttt{l}의 원소를 거꾸로 나열한 새로운 리스트를 생성하게 됩니다.

또한 두 리스트를 비교할 때는 다른 자료형과 마찬가지로 \texttt{==}를 사용하여 비교할 수 있습니다.
만약 특정 원소가 리스트에 해당하는지를 알아보려면 \texttt{in}을 사용하면 됩니다.
이 두 연산은 불리언을 반환하게 됩니다.
\begin{minted}[mathescape,
               linenos,
               breaklines,
               numbersep=5pt,
               frame=lines,
               framesep=2mm]{python}
>>> l = [0, 1, 2, 3, 4, 5, 6, 7, 8, 9]
>>> m = [0, 1, 2, 3, 4, 5, 6, 7, 8, 9]
>>> l == m
True
>>> 4 in l
True
>>> 10 in l
False
>>> [1, 4] in l
False
>>> [1, 4] in [[1, 4], 2, 3]
True
\end{minted}

두 리스트를 잇고 싶은 경우 단순히 \texttt{+}를 사용하여 연결하고, $n$ 회 반복한 만큼의 리스트를 만들고 싶다면 \texttt{*}를 사용하면 됩니다.
아래와 같이 사용할 수 있습니다.
\begin{minted}[mathescape,
               linenos,
               breaklines,
               numbersep=5pt,
               frame=lines,
               framesep=2mm]{python}
>>> l = [0, 1, 2] + [3, 4, 5, 6, 7, 8, 9]
>>> m = [0, 1, 2, 3, 4, 5, 6, 7, 8, 9]
>>> l == m
True
>>> l = [1, 2] * 3
>>> l
[1, 2, 1, 2, 1, 2]
\end{minted}

\subsubsection{Multi-Dimensional 다차원 리스트}
다차원 리스트는 리스트들의 리스트입니다.
이는 matrix행렬을 Python으로 구현하고자 할 때 유용하게 사용할 수 있습니다.
(일단 행렬은 아래와 같이 행과 열에 맞추어 수들을 나열한 수학적 객체라고 생각하시면 됩니다.)
\[
l =
  \begin{pmatrix}
    1 & 2 & 3 & 4 & 5 \\
    3 & 4 & 5 & 6 & 7
  \end{pmatrix}
\]
위와 같은 행렬은 아래와 같은 리스트로 표현할 수 있습니다.
\begin{minted}[mathescape,
               linenos,
               breaklines,
               numbersep=5pt,
               frame=lines,
               framesep=2mm]{python}
l = [[1, 2, 3, 4, 5], [3, 4, 5, 6, 7]]
\end{minted}
만약 \texttt{l}의 행의 길이를 알고 싶다면 \texttt{len(l[0])}을, 열의 길이를 알고 싶다면 \texttt{len(l)}을 사용하면 됩니다.
또한 각 원소를 접근할 때에, 1행 1열의 원소를 접근하고 싶다면 \texttt{l[0][0]}, 1행 2열의 원소를 접근하고 싶다면 \texttt{l[0][1]}, 2행 3열의 원소를 접근하고 싶다면 \texttt{l[1][2]}를 사용하면 됩니다.
일반적으로, \texttt{i}행 \texttt{j}열의 원소는 \texttt{l[i - 1][j - 1]}로 접근할 수 있습니다.

다차원 리스트의 경우에도 미리 원소를 지정하지 않고 \texttt{None}으로 초기화하여 리스트를 생성할 수도 있습니다.
만약 \texttt{height}만큼의 행의 개수, \texttt{width}만큼의 열의 개수를 가지는, \texttt{None}으로 초기화된 리스트를 생성한다면, 다음과 같이 생성할 수 있습니다.
아래 예시에는 \texttt{i}행 \texttt{j}열의 값을 \texttt{i + 2*j + 1}로 지정하는 과정까지 담고 있습니다.
\begin{minted}[mathescape,
               linenos,
               breaklines,
               numbersep=5pt,
               frame=lines,
               framesep=2mm]{python}
height = 3
width = 4
table = [[None]*width for i in range(height)]

for i in range(height):
    for j in range(width):
        table[i][j] = i + 2*j + 1
\end{minted}
위 과정을 통해서 
\[
l =
  \begin{pmatrix}
    1 & 3 & 5 & 7\\
    2 & 4 & 6 & 8\\
    3 & 5 & 7 & 9
  \end{pmatrix}
\]
에 대응되는 리스트 \texttt{[[1, 3, 5, 7], [2, 4, 6, 8], [3, 5, 7, 9]]}를 구성하게 됩니다.

나아가, 임의 차원을 가지는 리스트를 구성할 수 있습니다.
3차까지는 시각적으로 길이, 너비, 높이를 가지는 각 격자점에 값이 놓여 있다고 생각하시면 됩니다.
이 경우, 3중 루프를 구성하여 값을 채워 넣을 수 있습니다.
\begin{minted}[mathescape,
               linenos,
               breaklines,
               numbersep=5pt,
               frame=lines,
               framesep=2mm]{python}
height = 3
width = 4
depth = 2
table = [[[None]*width for j in range(height)] for i in range(depth)]

for i in range(depth):
    for j in range(height):
        for k in range(width):
            table[i][j][k] = i + 2*j + k + 1
\end{minted}

\subsubsection{알아두면 유용한 내장 함수}
리스트 \texttt{l}가 주어졌을 때,
\begin{itemize}
\item \texttt{sum(l)}: \texttt{l}의 모든 원소의 합을 반환합니다.
\item \texttt{min(l)}: \texttt{l}의 최소원을 반환합니다.
\item \texttt{max(l)}: \texttt{l}의 최대원을 반환합니다.
\item \texttt{l.append(x)}: \texttt{l}에 \texttt{x}를 마지막 위치에 추가하며 반환값이 없습니다.
\item \texttt{l.count(x)}: \texttt{l}에 포함된 \texttt{x}의 개수를 반환합니다.
\item \texttt{l.insert(i, x)}: \texttt{l}의 \texttt{i}번째 인덱스에 \texttt{x}를 추가하며 반환값이 없습니다.
\item \texttt{l.pop(i)}: \texttt{l}의 \texttt{i}번째 인덱스에 해당하는 원소를 없애고 해당 값을 반환합니다.
\item \texttt{l.remove(x)}: \texttt{l}의 첫 번째 \texttt{x}를 없앱니다.
\item \texttt{l.reverse()}: \texttt{l}을 뒤집으며 반환값이 없습니다.
\item \texttt{l.sort()}: \texttt{l}의 원소들을 정렬하며 반환값이 없습니다. 이 때, 수들 먼저 정렬된 후 문자열이 알파벳 순서로 정렬이 됩니다.
\end{itemize}
\textbf{이 때 반환값이 없는 메소드에 대해서는 다른 원소를 정의할 때 사용할 수가 없다는 점을 유의해야 합니다.}
예컨대, \texttt{m = l.append(x)}를 하면 \texttt{l}에는 \texttt{x}가 추가되지만, \texttt{m}에는 \texttt{None}이 배정됩니다.
아래 예시를 모아두었습니다.
\begin{minted}[mathescape,
               linenos,
               breaklines,
               numbersep=5pt,
               frame=lines,
               framesep=2mm]{python}
>>> l = [0, 1, 2]
>>> m = sum(l)
>>> m
3
>>> m = min(l)
>>> m
0
>>> m = max(l)
2
>>> m = l.append('hi')
>>> m
None
>>> l
[0, 1, 2, 'hi']
>>> m = l.count(0)
>>> m
1
>>> m = l.insert(1, 3}
>>> m
None
>>> l
[0, 3, 1, 2, 'hi']
>>> m = l.pop(3)
>>> m
2
>>> l
[0, 3, 1, 'hi']
>>> m = l.remove(0)
>>> m
None
>>> l
[3, 1, 'hi']
>>> m = l.reverse()
>>> m
None
>>> l
['hi', 1, 3]
>>> m = l.sort()
>>> m
None
>>> l
[1, 3, 'hi']
\end{minted}

\subsection{Strings문자열}
문자열은 리스트와 비슷하게 다룰 수 있습니다.
지금까지 문자열은 단순히 어떤 문구를 출력하기 위해 사용하는 정도로 그쳤었다면, 이제는 문자열을 가지고 연산을 하는 법을 알아볼 것입니다.
리스트 \texttt{l}의 \texttt{i} 번째 원소를 접근하기 위하여 \texttt{l[i - 1]}을 사용하였다면, 어떠한 문자열 \texttt{s}의 \texttt{i} 번째 문자를 접근하기 위하여 동일한 방식으로 \texttt{s[i - 1]}을 사용할 수 있다.
슬라이싱도 동일한 문법을 따릅니다.
나아가 두 문자열을 연결하기 위해서는 \texttt{+}를, 반복하기 위하여 \texttt{*}에 수를 곱하는 방식으로 연산하는 것까지 동일합니다.
또한 부등호를 통해 사전식 배열을 하였을 때 어떤 문자열이 먼저 나오는지를 판단할 수 있습니다.
\texttt{in}과 \texttt{not in}을 통해 어떤 문자열이 다른 문자열의 부분 문자열이 되는지도 판단할 수 있습니다.
\begin{minted}[mathescape,
               linenos,
               breaklines,
               numbersep=5pt,
               frame=lines,
               framesep=2mm]{python}
>>> s = "WEIZMANN"
>>> s[2]
'I'
>>> s[1:4]
'EIZ'
>>> s * 3
'WEIZMANNWEIZMANNWEIZMANN'
>>> s = s + "PYTHON"
>>> s
'WEIZMANNPYTHON'
>>> "WEIZMANN" > "PYTHON"
True
>>> 'MANN' in s
True
\end{minted}

\subsubsection{문자열의 연산}
리스트와의 차이라면 리스트는 mutable가변 객체이고, 문자열은 immutable불변 객체라는 것입니다.
이에 따라 한 문자열을 통해 새로운 값을 만들고 싶다면 새로운 문자열을 정의하는 방식을 사용해야 하며, 기존의 문자열을 수정하는 것은 불가능합니다.
반면, \texttt{+=}이나 \texttt{*=}을 사용하는 것은 가능합니다.
이는 \texttt{s += \textquotesingle a\textquotesingle}는 단순히 \texttt{s = s + \textquotesingle a\textquotesingle}를 축약한 것으로써 동일한 이름의 문자에 새로운 값을 배정하는 것을 의미하기 때문입니다.
\begin{minted}[mathescape,
               linenos,
               breaklines,
               numbersep=5pt,
               frame=lines,
               framesep=2mm]{python}
>>> s = "WEIZMANN"
>>> s[0] = "w"
Traceback (most recent call last):
  File "<stdin>", line 1, in <module>
TypeError: 'str' object does not support item assignment
>>> s += "PYTHON"
>>> s
"WEIZMANNPYTHON"
\end{minted}

\subsubsection{알아두면 유용한 내장 함수들}
문자열을 더 편리하게 다루기 위한 다양한 내장 함수들도 준비되어 있습니다.
문자열 \texttt{s}와 \texttt{t}가 주어졌을 때,
\begin{itemize}
\item \texttt{s.upper()}: \texttt{s}의 모든 문자를 대문자로 바꾸어 반환합니다.
\item \texttt{s.lower()}: \texttt{s}의 모든 문자를 소문자로 바꾸어 반환합니다.
\item \texttt{s.capitalize()}: \texttt{s}의 첫 문자는 대문자로, 나머지는 소문자로 바꾸어 반환합니다.
\item \texttt{s.isupper()}: \texttt{s}의 모든 문자가 대문자라면 \texttt{True}를, 아니라면 \texttt{False}를 반환합니다.
\item \texttt{s.islower()}: \texttt{s}의 모든 문자가 소문자라면 \texttt{True}를, 아니라면 \texttt{False}를 반환합니다.
\item \texttt{s.isdigit()}: \texttt{s}의 모든 문자가 숫자라면 \texttt{True}를, 아니라면 \texttt{False}를 반환합니다.
\item \texttt{s.split()}: \texttt{s}에서 띄어쓰기 문자로 구분된 구절들을 원소로 가지는 리스트를 반환합니다.
\item \texttt{s.find(t)}: \texttt{s}에 \texttt{t}가 포함되는 경우, \texttt{t}가 나타나는 첫 위치의 인덱스를 반환하며, 포함되지 않는 경우 -1을 반환합니다.
\end{itemize}
리스트와는 다르게 문자열을 수정하는 메소드는 없는 것을 확인할 수 있습니다.
전에 언급하였던 것처럼 문자열은 불변 객체이기 때문입니다.
\begin{minted}[mathescape,
               linenos,
               breaklines,
               numbersep=5pt,
               frame=lines,
               framesep=2mm]{python}
>>> s = "wEiZmANN".upper()
>>> s
'WEIZMANN'
>>> s = "weiZManN".lower()
>>> s
'weizmann'
>>> s = "weizmAN".capitalize()
>>> s
'Weizmann'
>>> "WEIZMANN12".isupper()
True
>>> "weizmanN12".islower()
False
>>> "1242".isdigit()
True
>>> l = "Hello World!".split()
>>> l
['Hello', 'World!']
>>> s = "WEIZMANN"
>>> s.find("IZ")
2
\end{minted}

\subsection{Turtle}
% \subsection{Counters카운터와 Toy Robot토이 로봇}
% 리스트를 배웠기 때문에, 조금 더 자유로운 반복문의 활용이 가능합니다.
% 이번 절에서는 지금까지 예제에서 살펴본 counter카운터 패턴에 대해 집중적으로 살펴볼 것이며, \texttt{cs1robots} 패키지를 통해 토이 로봇을 이용해볼 것입니다.
%
% 카운터 패턴은 주어진--리스트로써 표현된--집합에서 특정 조건을 만족하는 원소의 개수를 구하는데 사용할 수 있습니다.
% 아래는 \texttt{numbers}라는 수에서 7의 배수의 개수를 세는 함수 \texttt{count\_seven(numbers)}입니다.
% \begin{minted}[mathescape,
%                linenos,
%                breaklines,
%                numbersep=5pt,
%                frame=lines,
%                framesep=2mm]{python}
% def count_seven(numbers):
%     cnt = 0
%     for i in range(len(numbers)):
%         if numbers[i]%7 == 0:
%             cnt += 1
%     return cnt
%
% num = [1, 7, 2, 4, 3, 5, 34, 12, 42, 26]
% print count_seven(num)  # 2
% \end{minted}
% \texttt{for} 문을 통해서 각 원소에 대해 반복적인 시행을 표현하고, 내부에 \texttt{if} 문을 두어 특정 조건에만 명령이 수행되도록 하는 일반적인 패턴에 개수를 세는 역할을 부여한 것이 카운터 패턴입니다.
%
% 또한 \texttt{for} 문을 활용하여 toy robot의 움직임을 제어해볼 것입니다.
% 이를 위해 첫 시간에 \texttt{C:\textbackslash Python27\textbackslash Lib\textbackslash site-packages} 폴더에 복사한 \texttt{cs1robots} 패키지를 사용할 것입니다.
% \texttt{cs1robots} 패키지를 통해 로봇 객체\footnote{컴퓨터의 메모리에 할당된 공간을 포괄적으로 객체라고 하는데, object를 번역한 말입니다. 지금으로서는 간단히 대상이라고 생각해도 무방합니다.}를 불러올 때 앞으로 \texttt{hubo}라는 이름을 붙일 것입니다.
% 사실 일반적인 변수처럼 \texttt{a}, \texttt{my\_robot} 등 어떠한 이름을 붙여도 상관은 없으나, 수업 중에 \texttt{hubo}라고 관용적으로 붙이니 여기서도 이를 따르겠습니다.
% \begin{figure}[H]
% \centering
% \includegraphics[width=0.5\linewidth]{"./lectures/lecture4_example"}
% \label{fig:lecture4example}
% \end{figure}
% \texttt{hubo}는 앞으로 이동, 좌로 회전, 우로 회전에 해당하는 \texttt{move()}, \texttt{turn\_left()}, \texttt{turn\_right()} 등의 기본적인 동작을 취할 수 있습니다.
% 이번 절을 마치고 나면 \texttt{for} 문을 통해 \texttt{hubo}가 다양한 동작을 수행하도록 코드를 작성할 수 있을 것입니다.
%
% \subsubsection{Counters카운터 패턴}
% 카운터 패턴은 모두 아래와 같은 형식을 따릅니다.
% \begin{minted}[mathescape,
%                linenos,
%                breaklines,
%                numbersep=5pt,
%                frame=lines,
%                framesep=2mm]{python}
% cnt = 0
% for i in range(len(리스트_명)):
%     if 조건(리스트_명[i]):
%         cnt += 1
% \end{minted}
% 어떤 주어진 리스트 \texttt{리스트\_명}가 주어졌을 때, 해당 리스트의 모든 원소에 대해(줄 2) 어떤 조건 \texttt{조건}을 원소 \texttt{리스트\_명[i]}이 만족한다면(줄 3) \texttt{cnt}의 값을 하나씩 늘리는(줄 4) 것입니다.
% 위에서 보았던 7의 배수의 개수를 세는 예시도 마찬가지입니다.
% \texttt{numbers[i]\%7 == 0}의 조건이었는데, 7의 배수이면 \texttt{True}를, 아니면 \texttt{False}를 반환하는 함수를 정의한다면 \texttt{조건(리스트\_명[i])}의 형태로 만들 수 있습니다.
% 불리언 함수를 배울 때 언급하였듯, 조건이 복잡한 경우에는 함수를 정의하는 방식으로 코드를 작성해야 합니다.
% 예컨대 소수의 개수를 세는 카운터 패턴의 경우, 소수를 판별하는 불리언 함수 \texttt{is\_prime($\cdot$)}과 같은 함수를 정의하여 \texttt{if is\_prime(리스트\_명[i]):}와 같이 사용하면 됩니다.
% 주어진 리스트에서 소수의 개수를 세는 함수가 아래 예시에 나와 있습니다.
% \begin{minted}[mathescape,
%                linenos,
%                breaklines,
%                numbersep=5pt,
%                frame=lines,
%                framesep=2mm]{python}
% def is_prime(p):
%     for i in range(2, p/2):
%         if p%i == 0:
%             return False
%     return True
%
% def count_prime(numbers):
%     cnt = 0
%     for i in range(len(numbers)):
%         if is_prime(numbers[i]):
%             cnt += 1
%     return cnt
%
% num = [217, 287, 181, 143, 163, 319, 233, 399, 203]
% print count_prime(num)  # 3
% \end{minted}
%
% \subsubsection{Toy Robot토이 로봇 (선택)}
% 지금까지 수업을 통해 배운 함수, 조건문, 루프, 리스트 등의 개념을 종합적으로 활용하여 toy robot토이 로봇을 의도한 대로 움직이게 할 것입니다.
% Python 쉘에서 진행한 모든 Python 명령은 CLI\footnote{Command Line Interface, 즉 텍스트 기반의 명령행 인터페이스를 뜻한다.} 기반이었다면, 토이 로봇은 코드를 실행시키면 GUI\footnote{Graphic User Interface}인 창이 나올 것입니다.
% 일단 다음과 같은 코드를 작성해봅니다.
% \begin{minted}[mathescape,
%                linenos,
%                breaklines,
%                numbersep=5pt,
%                frame=lines,
%                framesep=2mm]{python}
% from cs1robots import *
% create_world()
% \end{minted}
% 이를 실행시키면, 아래와 같은 창이 뜹니다.
% \begin{figure}[H]
% \centering
% \includegraphics[width=0.5\linewidth]{"./lectures/lecture4_emptyworld"}
% \label{fig:lecture4emptyworld}
% \end{figure}
% 이것이 앞으로 로봇이 활동을 할 기본 ``세계''입니다.
% 기본적으로 가로세로 10칸의 공간이 생성됩니다.
%
% 위 코드에 \texttt{hubo = Robot()}를 추가하면, 로봇(앞으로 휴보라고 지칭하겠습니다)이 $(1, 1)$에 생성되는 것을 확인할 수 있습니다.
% \begin{figure}[H]
% \centering
% \includegraphics[width=0.5\linewidth]{"./lectures/lecture4_hubo"}
% \label{fig:lecture4hubo}
% \end{figure}
%
% 휴보를 앞으로 한 칸 움직이게 하기 위해서는 \texttt{hubo.move()}를 실행하고, 좌로 90도 회전하기 위해서는 \texttt{hubo.turn\_left()}를, 우로 90도 회전하기 위해서는 \texttt{hubo.turn\_right()}를 실행하면 됩니다.
% 아래는 휴보를 좌로 회전시킨 후 앞으로 한 번 전진한 후의 모습입니다.
% \begin{figure}[H]
% \centering
% \includegraphics[width=0.5\linewidth]{"./lectures/lecture4_huboleftmove"}
% \label{fig:lecture4huboleftmove}
% \end{figure}
%
% 만약 휴보가 이동한 경로를 가시화하고 싶다면 \texttt{hubo.set\_trace("blue")}와 같이 \texttt{set\_trace($\cdot$)} 메소드와 함께 인자로는 색상을 문자열로 입력하면 됩니다.
% 또한 휴보가 이동을 하는 것을 천천히 보려면 \texttt{hubo.set\_pause(0.5)}처럼 \texttt{set\_pause($\cdot$)} 메소드와 매 명령마다 정지할 시간을 초 단위의 인자를 입력하면 됩니다.
% 휴보가 한 변의 길이가 2인 정사각형의 경로를 따라 이동하는 코드와 그 후의 모습이 아래에 나타나 있습니다.
% \begin{minted}[mathescape,
%                linenos,
%                breaklines,
%                numbersep=5pt,
%                frame=lines,
%                framesep=2mm]{python}
% from cs1robots import *
% create_world()
%
% hubo = Robot()
% hubo.set_trace("blue")
% hubo.set_pause(0.5)
%
% def move_square():
%     for i in range(4):
%         hubo.move()
%         hubo.move()
%         hubo.turn_left()
%
% move_square()
% \end{minted}
% \begin{figure}[H]
% \centering
% \includegraphics[width=0.5\linewidth]{"./lectures/lecture4_hubosquare"}
% \label{fig:lecture4hubosquare}
% \end{figure}
%
% 조금 더 복잡한 종류의 명령도 얼마든지 가능합니다.
% 월드의 모양을 한 변의 길이 10짜리인 단조로운 정사각형이 아니라 칸 사이에 벽을 넣어 미로와 같은 지형을 만들 수 있는데, 계단형 지형을 살펴봅시다.
% \begin{figure}[H]
% \centering
% \includegraphics[width=0.5\linewidth]{"./lectures/lecture4_hubostairs"}
% \label{fig:lecture4hubostairs}
% \end{figure}
% 위 그림에서 나타난 움직임을 구현하는 코드는 아래와 같습니다.
% \begin{minted}[mathescape,
%                linenos,
%                breaklines,
%                numbersep=5pt,
%                frame=lines,
%                framesep=2mm]{python}
% hubo.move()
% for i in range(4):
%     hubo.turn_left()
%     hubo.move()
%     hubo.turn_right()
%     hubo.move()
%     hubo.move()
% \end{minted}
%
% \subsection{예제}
% \begin{enumerate}
% \item Write a function \texttt{sum\_squares(a)} that returns the sum of squares of the elements in the given list \texttt{a}.
% \begin{minted}[mathescape,
%                linenos,
%                breaklines,
%                numbersep=5pt,
%                frame=lines,
%                framesep=2mm]{python}
% def sum_squares(a):
%     n = len(a)
%     s = 0
%     # Add here!
%
% print sum_squares([3, 5, 4])  # 50
% print sum_squares([2, 5, 4, 0, 1, -1, 5, 1])  # 73
% \end{minted}
%
% \item Write function \texttt{compute\_polynomial(a, x)} that returns
% \[
% \sum_{i = 0}^{\texttt{n} - 1} \texttt{a[}i\texttt{]} \cdot \texttt{x}^i
% \]
% where \texttt{a} is an integer list of length \texttt{n} and \texttt{x} is an integer.
% \begin{minted}[mathescape,
%                linenos,
%                breaklines,
%                numbersep=5pt,
%                frame=lines,
%                framesep=2mm]{python}
% def compute_polynomial(a, x):
%     \ Add here!
%
% print compute_polynomial([3, 5, 4], 5)  # 128
% print compute_polynomial([2, 0, 4, 0, 1, -1, 5, 1], 3)  # 5708
% \end{minted}
%
% \item Write a function \texttt{harmonic\_mean(a)} that returns the harmonic mean of values in \texttt{a}, where \texttt{a} is a list of positive integers of length.
% A harmonic mean of given numbers $a_1, a_2, \dots, a_n$ is
% \[
% \frac{n}{\sum_{i = 1}^{n} \frac{1}{a_i}}.
% \]
% Also, write a function \texttt{geometric\_mean(a)} that returns the geometric mean of values in \texttt{a}.
% A geometric mean of given numbers $a_1, a_2, \dots, a_n$ is
% \[
% \sqrt[n]{\Pi_{i = 1}^{n} a_i}.
% \]
%
% \begin{minted}[mathescape,
%                linenos,
%                breaklines,
%                numbersep=5pt,
%                frame=lines,
%                framesep=2mm]{python}
% def harmonic_mean(a):
%     # Add here!
%
% def geometric_mean(a):
%     # Add here!
%
% numbers = [2, 4, 3, 10, 7, 2, 5, 6]
% print harmonic_mean(numbers)  # 3.64820846906
% print geometric_mean(numbers)  # 4.22116731332
% \end{minted}
%
% \item Write a function \texttt{reverse(a)} that returns a list with the same length of \texttt{a} where its order is reversed, e.g., \texttt{reverse([1, 5, 3, 7, 6])} returns \texttt{[6, 7, 3, 5, 1]}.
% \begin{minted}[mathescape,
%                linenos,
%                breaklines,
%                numbersep=5pt,
%                frame=lines,
%                framesep=2mm]{python}
% def reverse(a):
%     n = len(a)
%     b = [None] * n
%     # Add here!
%
% print reverse([3, 1, 5, 2, 4])  # [4, 2, 5, 1, 3]
% print reverse([7, 6, 3, 1, 5, 8, 2, 4])  # [4, 2, 8, 1, 3, 6, 7]
% \end{minted}
%
% \item Write a function \texttt{fibonacci(n)} that returns a list of first \texttt{n} Fibnonacci numbers where \texttt{n} is a positive integer.
% Assume that the first two terms of Fibonacci sequence are both 1.
% \begin{minted}[mathescape,
%                linenos,
%                breaklines,
%                numbersep=5pt,
%                frame=lines,
%                framesep=2mm]{python}
% def fibonacci(n):
%     # Add here!
%
% print fibonacci(10)  # [1, 1, 2, 3, 5, 8, 13, 21, 34, 55]
% \end{minted}
%
% \item Write a function \texttt{area(points)} that returns an area of a polygon given by the list \texttt{points}.
% \texttt{points} is a list of points where each point is represented by a list of x coordinate and y coordinate, e.g., \texttt{[3, 4]}.
% An example of \texttt{points} that represents the below polygon is \texttt{[[3, 1], [6, 3], [4, 4], [7, 6], [2, 7], [0, 5], [2, 3], [1, 2]]}.
% \begin{figure}[H]
% \centering
% \includegraphics[width=0.5\linewidth]{"./lectures/lecture4_q6"}
% \label{fig:lecture4q6}
% \end{figure}
% Note that the area of the polygon represented by points $(x_i, y_i)$ where $i = 0, 1, \dots, n - 1$ can be calculated via the so-called shoelace formula:
% \[
% \frac 12 \left| \sum_{i = 0}^{n - 1} x_i \left(y_{(i + 1)\ \mathrm{mod}\ n} - y_{(i - 1)\ \mathrm{mod}\ n}\right)\right|
% \]
% Also, write a function \texttt{perimeter(points)} that returns a perimeter of the given polygon represented by \texttt{points}.
% You may want to define your own functions other than \texttt{area} and \texttt{perimeter}.
% \begin{minted}[mathescape,
%                linenos,
%                breaklines,
%                numbersep=5pt,
%                frame=lines,
%                framesep=2mm]{python}
% # Add here!
%
% def area(points):
%     # Add here!
%
% def perimeter(points):
%     # Add here!
%
% points = [[3, 1], [6, 3], [4, 4], [7, 6], [2, 7], [0, 5], [2, 3], [1, 2]]
% print area(points)  # 22.0
% print perimeter(points)  # 23.8533258314
% \end{minted}
%
% \item Write a function \texttt{find\_min(a)} that returns the least element of the list \texttt{a}.
% \begin{minted}[mathescape,
%                linenos,
%                breaklines,
%                numbersep=5pt,
%                frame=lines,
%                framesep=2mm]{python}
% def find_min(a):
%     # Add here!
%
% print find_min([7.2, 5, 21, -1, 4, 0.4])  # -1
% \end{minted}
%
% \item Write a function \texttt{closest\_pair(points)} that returns a distance of the closest pair of points given in \texttt{points}.
% \begin{figure}[H]
% \centering
% \includegraphics[width=0.5\linewidth]{"./lectures/lecture4_q8"}
% \label{fig:lecture4q8}
% \end{figure}
% \begin{minted}[mathescape,
%                linenos,
%                breaklines,
%                numbersep=5pt,
%                frame=lines,
%                framesep=2mm]{python}
% # Add here!
%
% print closest_pair([[4, -4], [7, 5], [2, 1], [-2, -1], [-3, 5]])  # 4.472135955
% \end{minted}
%
% \item Write a function \texttt{count\_range(numbers, lower, upper)} that returns the number of integers in \texttt{numbers} which are in the range from \texttt{lower} to \texttt{upper}, where \texttt{numbers} is a list of integers.
% \begin{minted}[mathescape,
%                linenos,
%                breaklines,
%                numbersep=5pt,
%                frame=lines,
%                framesep=2mm]{python}
% def count_range(numbers, lower, upper):
%     # Add here!
%
% print count_range([8, 9, 10, 2, 4, 5, 9, 7, 2, 3, 7],  3, 7)  # 5
% \end{minted}
%
% \item Write a function \texttt{count\_within\_circle(points, radius)} that returns the number of points $(x, y)$ in \texttt{points} that is within $x^2 + y^2 \leq \texttt{radius}^2$.
% \begin{minted}[mathescape,
%                linenos,
%                breaklines,
%                numbersep=5pt,
%                frame=lines,
%                framesep=2mm]{python}
% def count_within_circle(points, radius):
%     # Add here!
%
% points = [[2, 1], [7, 5], [-5, 2], [-3, 5], [-7, 4], [-2, -1], [-2, -4], [-4, -2], [-6, -4], [4, -4], [6, -2]]
% print count_within_circle(points, 3)  # 2
% print count_within_circle(points, 5)  # 4
% print count_within_circle(points, 8)  # 9
% \end{minted}
%
% \item Write a boolean function \texttt{within\_rect(top, bottom, left, right, x, y)} where integers \texttt{top}, \texttt{bottom}, \texttt{left}, \texttt{right} represent an axis-aligned rectangle, e.g., the rectangle represented by \texttt{(top, bottom, left, right) = (2, -4, -5, 6)} is
% \begin{figure}[H]
% \centering
% \includegraphics[width=0.5\linewidth]{"./lectures/lecture4_q11"},
% \label{fig:lecture4q11}
% \end{figure}
% and \texttt{x} and \texttt{y} are integers.
% The function should return \texttt{True} if and only if the point $(\texttt{x}, \texttt{y})$ is in the rectangle including the boundary.
% \begin{minted}[mathescape,
%                linenos,
%                breaklines,
%                numbersep=5pt,
%                frame=lines,
%                framesep=2mm]{python}
% def within_rect(top, bottom, left, right, x, y):
%     # Add here!
%
% print within_rect(2, -4, -5, 6, -5, 2)  # True
% print within_rect(2, -4, -5, 6, 6, -1)  # True
% print within_rect(2, -4, -5, 6, 0, 1)  # True
% print within_rect(2, -4, -5, 6, -6, 0)  # False
% print within_rect(2, -4, -5, 6, 0, 3)  # False
% \end{minted}
% Write a function \texttt{count\_within\_rect(top, bottom, left, right, points)} using \texttt{within\_rect} implemented above that returns the number of points within the rectangle including boundary, where \texttt{points} is a list of points in the plane.
% \begin{minted}[mathescape,
%                linenos,
%                breaklines,
%                numbersep=5pt,
%                frame=lines,
%                framesep=2mm]{python}
% def count_within_rect(top, bottom, left, right, points):
%     # Add here!
%
% points = [[2, 1], [7, 5], [-5, 2], [-3, 5], [-7, 4], [-2, -1], [-2, -4], [-4, -2], [-6, -4], [4, -4], [6, -2]]
% print count_within_rect(2,-4,-5,6, points)  # 7
% \end{minted}
%
% \item Write a function \texttt{count\_leap\_year(numbers)} that returns the number of leap years in list \texttt{numbers}.
% \begin{minted}[mathescape,
%                linenos,
%                breaklines,
%                numbersep=5pt,
%                frame=lines,
%                framesep=2mm]{python}
% def count_leap_years(numbers):
%     # Add here!
%
% print count_leap_years([2008, 2011, 2012, 2000])  # 3
% print count_leap_years([2100, 2300, 2400, 2200])  # 1
% \end{minted}
%
% \item Write a function \texttt{count\_composite(numbers)} that returns the number of composites in list \texttt{numbers}.
% You may want to add your own function other than \texttt{count\_composite}.
% \begin{minted}[mathescape,
%                linenos,
%                breaklines,
%                numbersep=5pt,
%                frame=lines,
%                framesep=2mm]{python}
% # Add here!
%
% def count_composite(numbers):
%     # Add here!
%
% numbers = [217, 287, 181, 143, 163, 319, 233, 399, 203]
% print count_composite(numbers)  # 1
% \end{minted}
% \end{enumerate}
\end{document}
