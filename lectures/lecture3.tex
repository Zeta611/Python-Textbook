\documentclass[../main.tex]{subfiles}

\begin{document}
\newcommand{\pytrue}{\pyin{True}}
\newcommand{\pyfalse}{\pyin{False}}

\section{불리언 함수}
지난 단원에서는 조건문과 함께 불리언 타입에 대해 알아보았습니다.
또한 코드를 논리적 단위로 추상화한 함수에 대해서도 배웠습니다.
이번 절에서는 이러한 불리언 값인 \pytrue와 \pyfalse를 반환하는 불리언 함수들을 살펴볼 것입니다.

\subsection{예시}
불리언 함수는 \pytrue 혹은 \pyfalse를 반환하는 함수입니다.
이러한 불리언 함수는 길고 복잡한 조건문을 간결하게 표현할 때 유용합니다.

다음은 길이 \texttt{a}, \texttt{b}, \texttt{c}를 선분의 길이로 가지는 삼각형이 존재하는지 판별하는 조건문입니다:
\begin{minted}{python}
if a < b + c and b < c + a and c < a + b:
    ...
\end{minted}
이는 다음의 함수 \verb/can_form_triangle/로 대신할 수 있습니다:
\begin{minted}{python}
def can_form_triangle(a, b, c):
    if not a < b + c:  # if a >= b + c:
        return False
    if not b < c + a:
        return False
    if not c < a + b:
        return False
    return True

if can_form_triangle(a, b, c):
    ...
\end{minted}

조건문이 복잡해질수록, 불리언 함수를 사용해 조건문을 쪼개는 것이 도움이 됩니다.
``두 수 $n_1$과 $n_2$가 서로소라면''과 같은 표현을 한 조건문으로 표현하는 것보다는, 서로소이면 \pytrue를, 아니면 \pyfalse를 반환하는 함수 \verb/coprime/을 정의한 후 \pyin{if coprime(n1, n2): ...}와 같이 작성하는 것이 바람직합니다.

나아가, 불리언 함수는 (모든 함수가 그렇지만) 재사용성을 높혀줍니다.
예를 들어, 로봇이 거리를 측정한 후 장애물이 앞에 있는지 확인하는 함수 \verb/check_obstacle/을 생각해봅시다\footnote{\texttt{check\_obstacle}은 곧바로 \pyin{return distance < 10}으로 쓸 수도 있습니다. 이러한 패턴은 뒤에서 다시 살펴볼 것입니다.}:
\begin{minted}{python}
def check_obstacle(distance):
    if distance < 10:
        return True
    return False
\end{minted}
%%%%%%%%%%%%%%%%%%%%%%%%%%% REWRITE FROM HERE!
언뜻 보면 거리를 확인해야할 곳에 \pyin{distance < 10}을 써넣는 것이 굳이 함수를
작성하는 것보다 간단해 보일 수도 있지만, 로봇을 조종하는 코드에서 장애물 확인은
자주 발생할 것입니다.
그런데 만약 장애물이 있다고 인식하는 거리를 10에서 5로 줄이고 싶다고 합시다.
만약 \pyin{distance < 10}로 거리를 확인했다면, 코드의 수십, 아니 수백 곳에서
10을 5로 바꿔줘야 할 것입니다.
반면 함수를 정의한 후 \pyin{check_obstacle(distance)}와 같이 사용했다면 단 한
곳만 수정하면 될 것입니다.

\subsection{유의 사항}
불리언 \pytrue는 문자열 \pyin{"True"}가 아닙니다.
마찬가지로 불리언 \pyfalse는 문자열 \pyin{"False"}가 아닙니다.
\begin{minted}{python}
>>> type(True)
<type 'bool'>
>>> type("True")
<type 'str'>
>>> type(False)
<type 'bool'>
>>> type("False")
<type 'str'>
\end{minted}

\pytrue는 문자열이 아니라 그 자체로 불리언 자료형을 지닌 값입니다.\footnote{물론, 지난 시간에 언급했듯이 \pytrue는 \texttt{1}, \pyfalse는 \texttt{0}입니다.}
이를 강조하는 이유는, 간혹 \pytrue를 반환해야할 곳에 실수로
\pyin{"True"}라고 적는 경우가 있기 때문입니다.
이 둘은 서로 다른 값입니다.

조건문은 그 자체로 불리언 자료형을 가지기 때문에
\begin{minted}{python}
if ...:
    return True  # or False
\end{minted}
같은 형태는 단순히 \pyin{return ...}으로 쓸 수 있습니다.
아래는 $\texttt{a}x^2 + \texttt{b}x + c = 0$의 실근이 존재하는지를 판별하는 함수 \pyin{has_real_solution(a, b, c)}입니다:
\begin{minted}{python}
def has_real_solution(a, b, c):
    if b ** 2 - 4 * a * c >= 0:
        return True
    return False
\end{minted}
이는 다시 (간략히)
\begin{minted}{python}
def has_real_solution(a, b, c):
    return b ** 2 - 4 * a * c >= 0
\end{minted}
으로 나타낼 수 있습니다.
반대로 \pyin{if cond: return False}는 \pyin{return not cond}와 같이 정리가 됩니다.

나아가 불리언 함수를 조건문에 사용할 때는 \pyin{func(...) == True}보다는 바로 \pyin{func(...)}를 쓰는 것이 바람직합니다.
즉
\begin{minted}{python}
if real_solution(a, b, c) == True:
    ...
\end{minted}
보다는
\begin{minted}{python}
if real_solution(a, b, c):
    ...
\end{minted}
이 더 파이써닉\myindex{파이써닉}{Pythonic}한 표현입니다.
(\pyin{if func(...) == True:}을 쓰려다가 \pyin{==} 대신 \pyin{=}을 사용하는 실수를 미연에 방지할 수 있는 것은 덤입니다.)

언제 강조하여도 부족하지 않은 것은, \emph{복잡한 표현을 끊어서 표현하는 것이 미연의 실수를 방지하고 코드의 가독성을 높인다는 점입니다.}
다음의
\begin{minted}{python}
if ((x1 - x2)**2 + (y1 - y2)**2)**.5 <= ((x2 - x3)**2 + (y2 - y3)**2)**.5 + ((x3 - x1)**2 + (y3 - y1)**2)**.5:
    ...
\end{minted}
보다는
\begin{minted}{python}
def dist(x1, x2, y1, y2):
    return ((x1 - x2) ** 2 + (y1 - y2) ** 2) ** .5
a = dist(x1, x2, y1, y2)
b = dist(x2, x3, y2, y3)
c = dist(x3, x1, y3, y1)
if a < b + c:
    ...
\end{minted}
이 훨씬 읽기 편한 것을 볼 수 있습니다.

마지막으로, \pyin{float}끼리는 등호 연산 \pyin{==}을 하면 \emph{안 됩니다.}
\begin{minted}{python}
>>> 0.1 + 0.2
0.30000000000000004
>>> 0.1 + 0.2 == 0.3
False
\end{minted}
이러한 결과는 앞서 잠시 언급된 컴퓨터의 부동 소수점 처리 방식의 \alt*{반올림 오차}{rounding error} 때문입니다.
간단히는 이진법으로 십진법 소수(小數)를 정확하게 나타낼 수 없어 발생한 오차가 누적된 것이라 생각하면 됩니다.
따라서 \pyin{float}끼리는 \pyin{>}와 \pyin{<}를 사용해 오차 범위 내의 비교만 신뢰할 수 있습니다.
등호 비교는 되도록 \pyin{int} 상태에서 비교해야 합니다.
예를 들어, 두 정수 격자점 사이의 거리는 제곱근을 씌우기 전에 비교를 할 수 있습니다.
수학적으로는 $\lVert \vec v \rVert < \lVert \vec u \rVert \Leftrightarrow \lVert \vec v \rVert^2 < \lVert \vec u \rVert^2$이기 때문입니다.

\section{간단한 반복문}
컴퓨터는 인간보다 단순 반복 계산을 빠르게 수행할 수 있습니다.\footnote{다만, 최근에는 AlphaGo, ChatGPT 등의 등장으로 상황이 급변하고 있습니다.}
이번 절에서 배울 \alt*{반복문}{loops}을 사용하면 손으로 할 수 없는 수많은 계산을 파이썬으로 수행할 수 있게 됩니다.
지금까지 배운 조건문과 이번 절의 반복문을 배우면 이론적으로 어떠한 종류의 계산이든지 수행할 수 있게 됩니다.\footnote{물론 자기 자신(의 간단한 버전)을 이용해 스스로 정의되는 \alt*{재귀 함수}{recursive function}[recursive functions]를 사용하면 ``반복문'' 없이도 모든 계산을 수행할 수 있습니다.}

간단한 예시로는 로봇에게 반복적인 작업을 수행하도록 반복문을 작성할 수 있습니다.
로봇이 특정 거리만큼 앞으로 가게 하는 함수 \pyin{move_forward(distance)}, 특정
각도만큼 우회전하는 함수 \pyin{turn_right(angle)}가 주어져 있다고 합시다.
이때 정사각형으로 한 바퀴를 돌게 하는 코드는 반복문을 사용해 다음과
같이 쓸 수 있습니다.
\begin{minted}{python}
for i in range(4):
    move_forward(1)
    turn_right(90)
\end{minted}
줄 2와 줄 3의 내용을 총 4번 반복하도록 하는 \verb|for| 반복문으로, 어떤 일을
하는 코드인지 어렵지 않게 이해할 수 있습니다.

첫 번째 수업에서 1부터 $n$까지의 수를 더하는 문제를 풀 때에는 공식
$1 + 2 + \dots + n = \frac{n(n + 1)}{2}$을 사용하였습니다.
그렇다면 $1^m + 2^m + \dots + n^m$은 어떻게 계산할 수 있을까요?
임의의 $m$에 대해서 공식을 유도할 수는 없습니다.
이런 경우에 반복문을 활용할 수 있습니다.
지금까지는 `$n$회 무슨 작업을 반복하라'의 명령을 하려면 직접 $n$회 칠 수 밖에 없었지만,  \texttt{for} 반복문을 사용하면 임의의 $n$에 대해서 단 한 번의 일반적인 규칙을 제시하면 명령을 수행할 수 있습니다.
$\sum_{i = 1}^n i^m$을 반환하는 함수는 아래와 같습니다.
\begin{minted}{python}
def summ(m, n):
    summ = 0
    for i in range(1, n + 1):
        summ += i ** m
    return summ
\end{minted}
줄 3에서 \texttt{i}를 1 이상 \texttt{n + 1} 미만까지 하나씩 증가하며 밑의 줄 4를 반복한다는 뜻입니다.
이처럼 반복문의 기본은 \texttt{range}이라고 볼 수 있습니다.
이번 절에서는 \texttt{range} 함수를 활용하여, 조건문을 활용한 반복문이나 nested loops중첩 반복문에 대해 자세히 알아봅니다.

\subsection{\texttt{range} 함수를 활용한 반복문}
\texttt{range} 함수는 인자를 한 개에서 세 개까지 받을 수 있습니다.
일단 인자가 하나인 경우를 봅니다.
\begin{minted}{python}
for i in range(n):
    print(i)
\end{minted}
이는
\begin{minted}{python}
0
1
2
...
n - 1
\end{minted}
를 출력합니다.
즉, \pyin{range(n)}을 사용하여 0에서 \texttt{n}이 되기 직전까지 \texttt{i}를 1씩
증가시키며 \pyin{print(i)}을 실행하는 코드입니다.
참고로 반복문이 끝난 뒤 \texttt{i}는 \texttt{n - 1}로 남아 있습니다.

\texttt{range}에 두 개의 인자가 전달되는 경우는 아래에 나타나 있습니다.
\begin{minted}{python}
for i in range(m, n):
    print(i)
\end{minted}
이는
\begin{minted}{python}
m
m + 1
m + 2
...
n - 1
\end{minted}
의 값들을 출력합니다.
\pyin{range(m, n)}은 \texttt{m}에서 \texttt{n}이 되기 직전까지 \texttt{i}를
1씩 증가시키며 \pyin{print(i)}를 실행하는데, $\texttt{m} \geq \texttt{n}$이라면 반복문이 수행되지 않습니다.

마지막으로 세 개의 인자가 \texttt{range}에 전달되는 경우를 살펴봅시다.
\begin{minted}{python}
for i in range(m, n, k):
    print(i)
\end{minted}
이는
\begin{minted}{python}
m
m + k
m + 2 * k
...
m + l * k
\end{minted}
와 같습니다.
위에서 \texttt{m + l * k}는 \texttt{n}이 되기 직전의 값입니다.
\pyin{range(m, n, k)}은 \texttt{m}에서 \texttt{n}이 되기 직전까지
\texttt{i}를 \texttt{k}씩 증가시키며 \pyin{print(i)}를 실행하는 것입니다.
\texttt{k}는 음수가 될 수도 있습니다.
이 경우에는 \texttt{i}가 \texttt{m}에서부터 \texttt{n}이 되기 직전까지 감소됩니다.

정리하자면 \pyin{range(n)}은 \pyin{range(0, n)}, \pyin{range(0, n, 1)}과 같고, \pyin{range(m, n)}은 \pyin{range(m, n, 1)}과 같습니다.
아래는 인자를 넣는 세가지 경우를 모두 나타낸 예시입니다.
\begin{minted}{python}
for i in range(n):
    print(i, end=" ")  # 0 1 ... n - 1

for i in range(1, n + 1):
    print(i, end=" ")  # 1 2 ... n

for i in range(10, 0, -1):
    print(i, end=" ")  # 10 9 ... 1

for i in range(1, 10, 2):
    print(i, end=" ")  # 1 3 ... 9
\end{minted}
위 예시에서 \pyin{print()}안에 \pyin{end=" "}를 써준 것은 줄바꿈을 하지 않고
띄어쓰기를 하기 위함입니다.
(파이썬의 내장 \pyin{print} 함수에는 \pyin{end}에 값을 넣어 값을 출력한 후
뒤에 붙일 문자를 직접 지정할 수 있습니다.)

\subsection{예시}
이제는 \pyin{for} 문을 사용한 다양한 예시를 살펴보겠습니다.
아래는 팩토리얼을 계산하는 함수를 \pyin{for} 문을 통해 만든 것입니다.
\begin{minted}{python}
def factorial(x):
    prod = 1
    for i in range(1, x + 1):
        prod *= i
    return prod
\end{minted}
줄 2에서 \texttt{product}의 값을 1로 초기화를 한 후, 이후 반복하여
\verb|product|에 \verb|i|를 1부터 \verb|x|까지 반복하며 곱해주었습니다.

구구단표 작성과 같은 규칙적인 작업도 \pyin{for} 문을 통해 할 수 있습니다.
\begin{minted}{python}
for i in range(1, 10):
    for j in range(1, 10):
        print(i * j, end=" ")
    print()
\end{minted}
\texttt{i}를 고정시킨 후 \texttt{j}를 1부터 9까지 변화시키며 곱을 출력한 후, 줄을 바꾼 후 \texttt{i}를 1 증가시킨후 \texttt{j}를 1부터 9까지 변화시키며 곱을 출력하는 것을 반복하는 것이므로 결과적으로 구구단표를 작성하게 됩니다.
이렇게 \pyin{for} 문 안에서 또 다른 \pyin{for} 문이 수행되는 것을 중첩 반복문이라고 합니다.

코딩에서 개수 세기 등과 함께 중요한 패턴 중 하나는 최댓값과 최솟값을 찾는 것입니다.
$f(n) = (x - 2)^2$으로 정의된 함수에서 $f(0), \dots, f(4)$ 중 가장 작은 값을
찾고 싶다면 다음과 같이 코드를 작성할 수 있습니다.
\begin{minted}{python}
def f(x):
    return (x - 2) ** 2

def find_min_value():
    m = f(0)

    for i in range(1, 5):
        if f(i) < m:
            m = f(i)

    return m
\end{minted}
위 코드는 첫 번째 함숫값부터 마지막 함숫값까지 하나씩 살펴보는 것과 같습니다.
첫 번째 값을 보았을 때는 해당 값이 제일 작은 것(줄 5)이고 이후로는 이전까지의
최솟값보다 크면(줄 7) 해당 값으로 최솟값이 수정되는 것(줄 8)입니다.
최댓값을 구하는 것도 마찬가지로, 줄 8의 부등호 방향을 반대로 바꾸면 최댓값을 구하는 코드가 됩니다.
어떤 주어진 값에서 제일 작거나 큰 값을 찾을 때 자주 사용하는 패턴입니다.

\section{예제}
\begin{enumerate}
\item 홀수일 때 \pytrue를 반환하고 짝수일 때 \pyfalse를 반환하는 함수 \pyin{is_odd(n)}을 작성하세요.
\begin{minted}{python}
def is_odd(n):
    # Add here!

print(is_odd(12))  # False
print(is_odd(1))   # True
\end{minted}

\item 위의 \pyin{is_odd}를 사용해 \pyin{exactly_two_even(n1, n2, n3)}를 작성하세요. \pyin{exactly_two_even(n1, n2, n3)}은 \texttt{n1}, \texttt{n2} 와 \texttt{n3} 중 정확히 두 개가 짝수이면 \pytrue, 아니면 \pyfalse를 반환합니다.
\begin{minted}{python}
def exactly_two_even(n1, n2, n3):
    cnt = 0
    # Add here!
    return cnt == 2

print(exactly_two_even(1, 2, 3))   # False
print(exactly_two_even(4, 2, 3))   # True
print(exactly_two_even(7, 3, 4))   # False
print(exactly_two_even(2, 5, 8))   # True
print(exactly_two_even(2, 18, 4))  # False
\end{minted}

\item \texttt{a}, \texttt{b}, \texttt{c}를 세 변으로 가지는 삼각형이 있는지 판별하는 함수 \pyin{triangle(a, b, c)}를 작성하세요.
이때 \texttt{a}, \texttt{b}, \texttt{c}는 양의 정수입니다.
\begin{minted}{python}
# Add here!

print(triangle(3, 4, 5))  # True
print(triangle(1, 5, 2))  # False
print(triangle(1, 1, 1))  # True
print(triangle(3, 1, 1))  # False
\end{minted}

\item 세 점 \pyin{(x1, y1)}, \pyin{(x2, y2)}, \pyin{(x3, y3)}가 예각 삼각형을 이룰 수 있는지 판별하는 함수 \pyin{acute(x1, y1, x2, y2, x3, y3)}를 작성하세요.
\begin{minted}{python}
# Add here!

print(acute(1, 2, 4, 3, 2, 7))  # True
print(acute(1, 2, 4, 2, 5, 4))  # False
print(acute(1, 2, 4, 2, 4, 3))  # False
\end{minted}

\item 중심이 각각 \pyin{(x1, y1)}과 \pyin{(x2, y2)}이며 반지름은 각각 \texttt{r1}과 \texttt{r2}인 두 원이 있습니다.
이때 두 원이 두 점에서 만나는지 판별하는 함수 \pyin{intersect(x1, y1, r1, x2, y2, r2)}를 작성하세요.
\begin{minted}{python}
# Add here!

print(intersect(1, 1, 3, 5, 4, 2))  # False
print(intersect(1, 1, 3, 4, 3, 2))  # True
print(intersect(1, 1, 3, 2, 1, 2))  # False
\end{minted}

\item 어떤 수가 4의 배수이면 기본적으로는 윤년입니다. 그러나 해당 연도가 100으로
  나누어지면서 400으로는 안 나누어진다면, 윤년이 아닙니다. 윤년일 경우에
  \pytrue를 반환하고 아닐 경우에는 \pyfalse를 반환하는 함수
  \verb|leap_year|을 작성하세요.
\begin{minted}{python}
# Add here!

print(leap_year(2008), leap_year(2011))  # True False
print(leap_year(2000), leap_year(2100))  # True False
print(leap_year(2300), leap_year(2400))  # False True
print(leap_year(2012), leap_year(2200))  # True False
\end{minted}

\item 위에서 작성한 \pyin{leap_year} 함수를 사용해, \verb/year/년의 \verb/month/달에 며칠이 있는지 계산하는 함수 \pyin{num_days(year, month)}를 작성하세요.
\begin{minted}{python}
def num_days(year, month):
    assert 1 <= month <= 12
    if month == 1 or month == 3 or month == 5 or month == 7 or \
       month == 8 or month == 10 or month == 12:
        return 31
    # Add here!

print(num_days(2000, 1), num_days(2001, 4), num_days(2004, 8))  # 31 30 31
print(num_days(2004, 9), num_days(2005, 3), num_days(2005, 7))  # 30 31 31
print(num_days(2008, 2), num_days(2011, 2), num_days(2012, 2))  # 29 28 29
print(num_days(2000, 2), num_days(2100, 2), num_days(2200, 2))  # 29 28 28
print(num_days(2300, 2), num_days(2400, 2), num_days(3200, 2))  # 28 29 29
\end{minted}

\item 수혈은 다음과 같은 혈액형 조합에서 가능합니다:
\begin{itemize}
\item O형에서 O형
\item A형에서 A 또는 O형
\item B형에서 B 또는 O형
\item AB형에서 A, B, AB 또는 O형.
\end{itemize}
함수 \pyin{blood(supply_O, supply_A, supply_B, supply_AB, demand_O, demand_A, demand_B, demand_AB)}는 각 혈액형별 수요와 공급을 받아 혈액이 충분한지를 판별해야 합니다.
이를 작성하세요.
\begin{minted}{python}
def blood(supply_O, supply_A, supply_B, supply_AB,
          demand_O, demand_A, demand_B, demand_AB):
    if supply_O < demand_O:
        return False
    # Add here!

print(blood(50, 36, 11, 8, 45, 42, 10, 3))  # False
print(blood(50, 36, 11, 3, 45, 38, 10, 7))  # True
\end{minted}

\item 아래의 구구단 표를 작성하는 함수 \pyin{print_tables()}를 작성하세요.
\begin{verbatim}
1 2 3 4 5 6 7 8 9
2 4 6 8 10 12 14 16 18
...
9 18 27 36 45 54 63 72 81
\end{verbatim}

\item 주어진 수 \verb|n|에 따라 \verb|좌회전|, \verb|직진|을 줄마다 출력하는
  \pyin{turn_left_and_move(n)}를 작성하세요.
\begin{minted}{python}
def turn_left_and_move(n):
    # Add here!

turn_left_and_move(3)
# 좌회전
# 직전
# 좌회전
# 직전
# 좌회전
# 직전
\end{minted}

\item 다음과 같은 값들을 출력해야 합니다.
\begin{minted}{python}
1 2 3 4 5 6 7 8 9
3 6 9 12 15 18 21 24 27
5 10 15 20 25 30 35 40 45
7 14 21 28 35 42 49 56 63
9 18 27 36 45 54 63 72 81

1
3 6 9
5 10 15 20 25
7 14 21 28 35 42 49
9 18 27 36 45 54 63 72 81
\end{minted}
이런 일을 하는 함수 \pyin{print_tables()}를 작성하세요.
\begin{minted}{python}
def print_tables():
    # Add here!

print_tables()
\end{minted}

\item 함수 \pyin{sum_interval(a, b)}는 두 정수 \verb/a/와 \verb/b/를 받아 $\texttt{a} + (\texttt{a} + 1) + \dots + \texttt{b}$를 계산합니다.
  이를 작성하세요.
\begin{minted}{python}
def sum_interval(a, b):
    # Add here!

print(sum_interval(5, 10))  # 45
print(sum_interval(15, 100))  # 4945
\end{minted}

\item 원주율에 수렴하는 아래의 관계식이 알려져 있습니다\footnote{\alt*{라이프니츠}{Gottfried Wilhelm Leibniz}가 발견하였습니다.}:
\[
\pi = 4\left(\frac11 - \frac13 + \frac 15 - \frac17 + \frac19 - \dots\right)
\]
위 식의 첫 \verb/n/항들을 더해 $\pi$의 근사값을 구하는 함수 \pyin{calculate_pi(n)}를 작성하세요.
\begin{minted}{python}
def calculate_pi(n):
    # Add here!

print(calculate_pi(1000))
\end{minted}

\item 원점을 중심으로 하고 반지름이 \verb/r/인 원 내부의 정수 격자점을 세는 함수 \pyin{within_circ(r)}를 먼저 작성하세요.

  \alt*{몬테 카를로 기법}{Monte Carlo method}를 통해 다음과 같이 원주율을 계산할 수 있습니다:
\[
\pi = \lim_{\texttt{r} \rightarrow \infty} \frac{\texttt{within\_circle(r)}}{\texttt{r}^2}
\]
이를 통해 원주율을 근사하는 \pyin{calculate_pi(r)}를 작성하세요.
\begin{minted}{python}
def within_circ(r):
    cnt = 0
    # Add here!
    return cnt

def calculate_pi(r):
    # Add here!

print(calculate_pi(1000))
\end{minted}

\item 날짜가 주어졌을 때 요일을 \pyin{"Mon"}, \pyin{"Tue"}, \pyin{"Wed"}, \pyin{"Thu"}, \pyin{"Fri"}, \pyin{"Sat"}, 혹은 \pyin{"Sun"}로 반환하는 함수 \pyin{day_of_week(year, month, day)}를 작성하세요.
  1970년 1월 1일 목요일 이후의 날짜만 고려합니다.
  위에서 작성한 \pyin{leap_year}과 \pyin{num_days}를 활용하세요.
\begin{minted}{python}
def day_of_week(year, month, day):
    # Add here!

print(day_of_week(2023, 1, 19))  # Thu
\end{minted}

\item 함수 $f: \{0, 1, \dots, 99\}^2 \rightarrow \mathbb{Z}$는 다음과 같이 정의됩니다:
\[
f(i, j) = \left(i^5 + 2i^3j^2 + 5i^2 + j + 5000\right) \pmod{10000}
\]
$f$의 최댓값을 구하는 함수 \pyin{find_max()}를 작성하세요.
\begin{minted}{python}
def f(i, j):
    return (i**5 + 2* i**3 * j**2 + 5 * i**2 + j + 5000) % 10000

def find_max():
    # Add here!

print("Maximum is", find_max())  # 9997
\end{minted}

\item \verb/n/ 미만의 3 혹은 5의 배수의 합을 계산하는 함수 \pyin{sum_mlt35(n)}을 작성하세요.
\begin{minted}{python}
def sum_mlt35(n):
    # Add here!

print(sum_mlt35(10))    # 23
print(sum_mlt35(100))   # 2318
print(sum_mlt35(1000))  # 233168
\end{minted}

\item 피보나치 수열의 첫 \verb/n/ 항 중 짝수인 것들을 더한 값을 계산하는 함수 \pyin{sum_even_fib(n)}를 작성하세요.
  피보나치 수열의 첫 두 항은 1과 2입니다.

\begin{minted}{python}
def sum_even_fib(n):
    # Add here!

print(sum_even_fib(300)) == ...   # Too large to write here
print(sum_even_fib(400)) == ...   # Too large to write here
print(sum_even_fib(1000)) == ...  # Too large to write here
\end{minted}

\item 먼저 \verb/n/의 진약수의 합을 구하는 함수 \pyin{d(n)}을 작성하세요.

만약 $b = \texttt{d($a$)} = \texttt{d(d($b$))}$ ($a \neq b$)이면, $a$와 $b$는 우호적인 쌍이고, 각각을 \alt*{우호적인 수}{amicable number}[amicable numbers]라고 부릅니다.
\verb/n/이 우호적인 수인지 판별하는 함수 \pyin{is_amicable(n)}을 작성하세요.

이제 \verb/n/ 미만의 모든 우호적인 수들의 합을 구하는 함수 \pyin{sum_amicable(n)}을 작성하세요.
\begin{minted}{python}
def d(n):
    # Add here!

def is_amicable(n):
    # Add here!

def sum_amicable(n):
    # Add here!

print(sum_amicable(10000))  # 31626
\end{minted}
\end{enumerate}
\end{document}
