\documentclass[../main.tex]{subfiles}

\begin{document}
지금까지 파이썬에서
\begin{itemize}
\item 기본적인 타입과 이들에 대해 수행할 수 있는 연산자들,
\item 식들을 조립하여 복잡한 식을 구성하는 방법, 그리고
\item 변수를 정의하여 값을 담을 수 있는 방법을
\end{itemize}
알아보았습니다.
이러한 기능은 대부분의 언어에서 지원하는 프로그래밍 언어의 기본 요소입니다.

이번 단원에서는 일련의 연산을 하나의 단위로 구성하여 \alt*{함수}{function}를 작성하고,
경우에 따라 프로그램 실행의 순서를 다르게 변화할 수 있는 구성 요소인 \alt*{조건문}{conditional statements}에 대해서 알아봅니다.

\section{함수}
함수를 사용하면 논리적으로 하나의 역할을 하는 코드를 감싸서 하나의 블럭으로 만들 수 있습니다.
다음의 간단한 예시를 봅시다:
\begin{minted}{python}
def square(x):
    return x * x

print(square(9))  # 81
\end{minted}
위 코드는 어떤 값을 받아 제곱을 수행하는 함수를 파이썬에서 어떻게 작성하는지 보여줍니다.

우선 모든 함수\footnote{뒤에서 배우는 \alt*{람다 함수}{lambda function}가 있지만 여기서는 무시합니다.}의 정의는 \verb/def/로 시작합니다.
그 후 함수에 붙일 이름(\verb/square/)이 이어지고, 괄호 안에 함수가 받을 \alt*{인자들}{arguments}의 이름이 들어갑니다.
이러한 이름을 \alt*{매개변수}{parameters}라고 합니다.
매개변수와 인자의 관계는, 변수와 저장되는 값의 관계와 대응됩니다.
이 괄호가 끝나면 반드시 \alt{콜론}{colon}을 사용해 함수 이름과 매개변수의 나열이 끝났으며, 함수를 구성하는 코드가 시작된다는 표시를 해야 합니다.

여기서 주목해야할 것은, (콜론 뒤에 이어지는) 함수를 구성하는 코드가 반드시 들여쓰기되어야 한다는 것입니다.
파이썬에서는 기본적으로 네 칸 띄어쓰기로 \alt*{들여쓰기}{indentation}를 권장합니다.\footnote{\emph{특별한 사유가 있다면} 일정한 개수의 띄어쓰기나 탭 문자를 사용할 수도 있습니다.}
이에 대해서는 뒤에서 다시 살펴볼 것입니다.

마지막으로 함수를 구성하는 논리적인 코드 조각이 계산된 결과를 돌려주는 \alt*{반환}{return}입니다.
\pyin{return x * x}는 \pyin{x * x}를 계산한 결과가 함수 \verb/square/의 계산 결과라는 의미입니다.
반환이라고 이름 붙여진 이유는, \pyin{square(9)}와 같이 함수를 사용할 때, \verb/square/의 안에서 계산된 결과를 \pyin{square(9)}가 사용된 위치로 ``돌려주어야''하기 때문입니다.

이미 보았지만 이렇게 정의된 함수는 \pyin{square(9)}와 같이 괄호 안에 값을 넘겨주어 사용할 수 있습니다.
이를 \alt*{함수 호출}{function call}이라고 말합니다.
함수를 불러서 사용하는 것이기 때문입니다.
이렇게 호출된 함수는 값을 호출된 장소로 반환합니다.

이번에는 여러 식이 사용된 코드를 통해서 다시 한 번 살펴봅시다\footnote{변수 이름들과 식을 보면 알 수 있겠지만 \alt*{특수 상대성 이론}{special relativity theory}에서 상대 시간을 계산하는 코드입니다.}:
\begin{minted}{python}
import math

LIGHT_SPEED = 299792458

def square(x):
    return x * x

def gamma(v):
    b = v / LIGHT_SPEED
    return 1 / math.sqrt(1 - square(b))

def relative_time(x, t, v):
    k = t - v * x / square(LIGHT_SPEED)
    return gamma(v) * k

position = 10
time = 13
velocity = LIGHT_SPEED / 2

print(gamma(velocity))
print(relative_time(position, time, velocity))
\end{minted}
위 예시에는 앞선 \verb/square/과 다르게 둘 이상의 매개변수를 사용한 함수 \verb/relative_time/이 정의되어 있습니다.
코드가 길어졌지만, 구성 방식은 앞서 설명한 것과 동일합니다.

\subsection{구조적 프로그래밍}
기계어로 프로그래밍하는 방식에서 탈피하기 위한 첫 걸음마 중 하나가 함수라고 말할 수 있습니다.\footnote{반대로 \alt*{번역기}{compiler}가 하는 일에는 이렇게 정의된 함수를 기계어로 녹여내는 일이 포함됩니다.}
함수를 전혀 사용하지 않고 프로그램을 작성할 수는 있지만, 함수를 사용하는데에는 다양한
이점이 있기 때문입니다.

함수를 사용하면 프로그램을 논리적인 역할에 따라 여러 조각으로 나누는 \alt*{구조적 프로그래밍}{structural programming}\footnote{구조적 프로그래밍은 비구조적 프로그래밍과 대비되는
패러다임입니다. 비구조적 프로그래밍은 프로그램 전체를 하나의 큰 덩어리로 작성하는 방식입니다. 이러한 프로그램은 \textsc{goto}문\index{\textsc{goto}}과 같은 제어문에
의존할 수 밖에 없는데, 이는 프로그램의 디버깅을 어렵게 가독성도 해칩니다.
특히 구조화되지 않은 코드는 프로그램이 복잡해지면 실행 흐름이 복잡하게 얽히게
되는데, 이렇게 읽기 힘들고 잘 구조화되지 않은 코드를 \alt*{스파게티 코드}{spaghetti code}라고 부릅니다.
사실상 기계어를 제외한 현대의 모든 프로그래밍 언어는 구조적 프로그래밍을 지원합니다.}이 가능해집니다.
또한 함수를 사용하면 같은 코드를 반복하지 않음으로써 프로그램의 용량을 줄이고 범용성을 늘릴 수 있습니다.
자주 반복하여 사용하는 식을 하나의 변수에 대입하여 사용하면 효율적인 것과 비슷한 맥락입니다.

이렇게 코드 조각을 함수로 감싸는 것을 \alt*{캡슐화}{encapsulation}\index{encapsulation|seealso{abstraction}}\index{캡슐화|seealso{추상화}}라고 부릅니다.
뒤에서 \alt*{객체 지향 프로그래밍}{object-oriented programming}\index{OOP|see{object-oriented programming}}을 배우면, 연관된 일을 수행하는 여러 함수들과 변수들까지도 캡슐화를 할 수 있는 방법에 대해 알 수 있습니다.

한편, 함수로 논리를 감싸는 것을 \alt*{추상화}{abstraction}\index{abstraction|seealso{encapsulation}}\index{추상화|seealso{캡슐화}}라고도 부를 수 있습니다.
위에서 \verb/square/ 함수를 생각하면, 함수를 호출하는 입장에서는 \verb/square/이 \pyin{x * x}처럼 구현이 되었는지, 아니면 (비효율적으로) \verb/x/를 \verb/x/번 더해서 구현이 되었는지는 알 필요가 없습니다.
\verb/square/이 약속한 것처럼, 값을 제곱해서 돌려주기만 하면 그만이기 때문입니다.
이러한 관점으로 함수를 바라보면, 함수는 기능의 구현을 추상화하여 그 역할만 남겨놓은 블랙박스라고 생각할 수 있습니다.

\subsection{들여쓰기}
파이썬에서는 들여쓰기\myindex{들여쓰기}{indentation}가 논리적인 단위를 나누는 중요한 문법입니다.
C, C++ 등의 언어에서는 들여쓰기에 대한 문법 규칙 없이 중괄호(\verb|{}|)와
세미콜론(\verb|;|)으로 단위를 구분하며, 들여쓰기는 프로그램을 읽고 쓰는 인간을 위한 강제되지 않는 (중요한) 약속 정도입니다.
그러나 중괄호나 세미콜론을 사용하지 않는 파이썬에서는 들여쓰기가 문법적으로
반드시 필요합니다.\footnote{띄어쓰기를 하여 들여쓸 것인지, 탭 문자를 사용하여
들여쓸 것인지는 프로그래머들 사이에서 펼쳐지는 오랜 논쟁입니다. 다만 파이썬 개발자들은 띄어쓰기 4회를 주로 사용합니다. 이는 파이썬의 스타일 가이드인 PEP-8\index{PEP-8}에 명시되어 있습니다.}
참고로, 띄어쓰기 네 번 써서 들여쓴다고 해서 키보드의 스페이스 바를 네 번 누를 필요는 없습니다.
(제대로된) IDE\myindex{통합 개발 환경}{integrated development environment, IDE}나 텍스트 편집기\myindex{텍스트 편집기}{text editor}라면 환경설정에서 키보드의 탭 키를 \alt*{소프트 탭}{soft tab} (띄어쓰기 여러 개를 사용하여 들여쓰는 것)으로 설정하면 됩니다.

함수뿐만이 아니라 이번 단원 뒤에서 배울 조건문, 그리고 나중에 배울 반복문 등에서 들여쓰기는
논리적 블럭을 구분하고 포함 관계를 나타냅니다.
위의 예시처럼 함수에서는 함수 이름과 매개변수 나열 후 함수의 내용물을 적을 때에는 함수의 끝까지 한 단계 들여써야 합니다.
\emph{들여쓰기를 실수로 의도한 바와 다르게 한다면 코드의 수행 결과가 완전히 달라지거나 실행조차 안 될 수 있습니다.}
더 무서운 점은, 이렇게 잘못 작성된 코드가 문법적으로는 맞지만 논리적으로는 틀려서 에러 메시지 없이 원하지 않는 값만 나올 수도 있습니다.
이렇게 되면 디버깅을 하기가 굉장히 까다로워집니다.\footnote{들여쓰기를 명시적으로 나타내주는 IDE의 기능을 활용하면 생산성을 높일 수 있습니다.  대부분의 IDE나 프로그래밍용 텍스트 에디터들은 들여쓰기를 도와주는 기능이 포함되어 있습니다.}

\subsection{내장 함수와 \texttt{import}}
사실 지난 단원에서도 여러 함수를 보았는데요, \verb/type/이나 타입 변환을 위해 사용한
\verb/int/ 등이 바로 함수입니다.
이들 함수는 프로그래머가 직접 정의하지 않아도 사용할 수 있는 \alt*{내장 함수}{built-in functions}입니다.

어떤 내장 함수들은 기본적으로 바로 사용할 수 있는 것이 아니라, ``이러이러한 묶음 안의 함수들을 사용할 것이다''라는 것을 코드에 명시해야 쓸 수 있습니다.
이는 \verb/import/\index{\texttt{import}}문을 사용해서 불러오면 됩니다.
수식 계산을 위해 사용하는 패키지인 \texttt{math}로 예를 들자면, 제곱근 함수
\verb/sqrt/, 사인 함수 \verb/sin/와 같은 삼각
함수, 로그 함수 \verb/log/ 등을 사용하기 위해 \pyin{import math}를 해야 합니다.

지난 단원의 마지막 문제인 올림 함수는 \verb|math|
패키지에 들어있는 \alt*{올림 함수}{ceiling function} \verb/ceil/\index{\texttt{ceil}}로 이미 구현되어 있습니다.
마찬가지로 \texttt{int}와 같은 ``가짜''\footnote{$\texttt{int(-1.3)} = -1 \ne \lfloor-1.3\rfloor = -2$} \alt*{내림 함수}{floor function}가 아니라 진짜 내림 함수인 \verb/floor/\index{\texttt{floor}}도 \verb/math/에 들어 있습니다.
참고로 반올림 함수 \verb/round/\index{\texttt{round}}는 파이썬 2에서 \verb|math| 패키지에서 제공되었지만, 파이썬 3부터는 \verb/math/ 없이 사용할 수 있습니다.

다음의 코드로 확인해봅시다:
\begin{minted}{python}
>>> import math
>>> math.sqrt(4)
2.0
>>> math.sin(30 * math.pi / 180)
0.49999999999999994
>>> math.log(math.e)
1.0
>>> math.ceil(-1.5)
-1
>>> math.floor(-1.5)
-2
>>> round(-1.5)
-2
\end{minted}

만약 \texttt{math.}을 반복하기 귀찮다면, \pyin{from package import *}로 \verb/package/ 안의 모든 함수를 밖으로 끄집어내어 가져올 수 있습니다:
\begin{minted}{python}
>>> from math import *
>>> sqrt(4)
2.0
>>> sin(30 * pi / 180)
0.49999999999999994
>>> log(e)
1.0
\end{minted}
여기서 뭔가 무시무시한 것을 발견하였나요?
이제 자연 상수 $e$를 \verb/math.e/가 아니라 그대로 \verb/e/와 같이 사용해야 합니다.
(짧은 변수 이름을 사용하는 것이 좋은 습관은 아니지만) \verb/e/를 다른 곳에서 부주의로 인해 정의한다면 자연 상수에 해당하는 $e$의 값이 아니라 다른 값이 \verb/e/에 담기게 됩니다.
다음과 같이 말이죠.
\begin{minted}{python}
>>> from math import *
>>> e
2.718281828459045
>>> e = 1
>>> e
1
\end{minted}
또한 같은 함수 이름을 가진 서로 다른 두 함수---\texttt{fn}이라고 합시다---를 지닌 두 패키지 \texttt{pack1}과
\texttt{pack2}에서 \pyin{from pack1 import *}를 한 후 \pyin{from pack2
import *}를 한다면 뒤에서 불러온 \texttt{pack2}의 \texttt{fn}이 먼저 불러온
\texttt{pack1}의 \texttt{fn}을 덮어 씌울 것입니다.  예컨대 \texttt{numpy}라는
패키지를 컴퓨터에 설치한 후, \pyin{from numpy import *}와 \pyin{from math
import *}를 한다면 상당수의 함수가 겹치게 되어 두 패키지 중 하나의 것만 사용할 수 있게 됩니다.
따라서 불가피한 경우에만 \pyin{from package import *}를 사용하는 것이 권장됩니다.

나아가 자신이 \texttt{math} 패키지에서 \texttt{sin} 함수처럼 몇 개의 함수들만 필요하다면, 매번 \verb/math./을 붙이지 않고 직접 표시한 함수들을 사용할 수도 있습니다:
\begin{minted}{python}
>>> from math import sqrt, sin
>>> sqrt(4)
2.0
>>> sin(0)
0.0
\end{minted}

그리고 패키지를 불러올 때 이름을 원하는대로 바꿀 수도 있는데요, \pyin{import numpy as np}와 같이 패키지를 불러온다면, \pyin{numpy.sin} 대신 \pyin{np.sin}과 같이 사용할 수 있습니다.

\subsection{함수와 부작용}
함수는 반환 값이 있는 것과 그렇지 않은 것으로 나눌 수 있습니다.
그렇지 않은 것은 (함수가 어떤 식으로든지 유용한 일을 한다면) 무조건 \alt*{부작용}{side effects}을 일으킵니다.

아래는 원의 반지름을 받아서 면적을 반환하는 함수
\verb/circ_area/입니다.
\begin{minted}{python}
import math

def circ_area(radius):
    return math.pi * radius ** 2
\end{minted}
\texttt{return} 다음에 반환할 값이 나타나 있습니다.
아래는 원의 면적을 반환하지 않고 단순히 출력하는 함수 \verb/print_circ_area/입니다.
\begin{minted}{python}
import math

def print_circ_area(radius):
    print(math.pi * radius ** 2)
\end{minted}
\verb/circ_area/의 경우에는 원의 면적을 출력하기 위해
\texttt{print}를 사용해야 합니다.  그렇지만 다른 값을 계산하는 등의 작업에서
함수를 다른 식에 넣을 수 있습니다.  \verb/print_circ_area/는
자신이 직접 원의 면적을 출력하지만, 그 값을 다른 식에서 활용할 수는 없습니다.
반환하는 값이 없기 때문입니다.

함수를 정의하여 사용할 때에는, 호출하는 시점 이전에 정의가 되어 있으면 됩니다.
이 조건만 만족한다면 함수의 정의는 다른 코드들 사이에 끼워두어도 되고, 함수들 간의 정의
순서도 상관이 없습니다.
또한 함수를 정의하기 위해서 다른 함수를 (당연히) 사용할 수 있습니다.
\begin{minted}{python}
import math

def get_dist(x1, y1, x2, y2):
    dx = x1 - x2
    dy = y1 - y2
    return math.sqrt(dx ** 2 + dy ** 2)

def circ_area(cx, cy, px, py):
    r = get_dist(cx, cy, px, py)
    return math.pi * r ** 2

print(circ_area(0, 0, 3, 4))
\end{minted}
위 코드의 \verb/circ_area/ 함수를 정의하기 위해 \verb/get_dist/ 함수를 사용하였는데, \verb/get_dist/ 함수의 정의와 \verb/circ_area/ 함수 정의 순서는 바뀌어도 상관
없습니다.

\subsection{매개 변수와 인자}
함수를 정의할 때 사용되는 매개 변수는 함수 정의 밖에서 사용될 수 없습니다.
일종의 블랙 박스로 생각하면 되는데, 함수 안에서 정의되는 \alt*{지역 변수들}[지역 변수]{local variables}은 그 안에서만 사용할 수 있습니다.
예를 들어 직전 예시의 \verb/get_dist/ 함수의 \texttt{dx}나 \texttt{dy}는 함수 안에서
만들어졌다가 사라지는 변수입니다.
매개 변수도 지역 변수입니다.
나아가 함수 밖에서 \texttt{dx}라는 이름의 새로운 변수를 만들더라도 이는 함수 안에서 정의되었던 지역 변수 \texttt{dx}와는 전혀 다른 변수입니다.
우연히 이름이 겹친 것일 뿐입니다.

반면 함수 밖에서 먼저 정의되었으면서 함수 안에서 정의되지 않은 변수는 \alt*{전역 변수}{global variable}[global variables]라고 부릅니다.
이 경우에는 함수 밖에서 정의된 값이 함수 안에서 그대로 사용됩니다.

아래의 예시를 통해 지역 변수와 전역 변수를 나누는 기준을 완전히 이해했는지 확인해봅시다.
이해를 위해 (이름이 같더라도) 다른 변수들은 다른 색상으로 표시하였습니다.
\begin{minted}[escapeinside=||]{python}
def fn1(|\textcolor{red}{a}|, |\textcolor{red}{b}|):
    |\textcolor{brown}{c}| = |\textcolor{red}{a}| + |\textcolor{red}{b}|
    |\textcolor{brown}{d}| = |\textcolor{red}{a}| - |\textcolor{red}{b}|
    return (|\textcolor{brown}{c}| + |\textcolor{brown}{d}|) / 2.

|\textcolor{magenta}{a}|, |\textcolor{magenta}{b}| = 1, 2
print(fn1(|\textcolor{magenta}{a}|, |\textcolor{magenta}{b}|))
print(fn1(1, 2))

|\textcolor{violet}{c}|, |\textcolor{violet}{d}| = 3, 4

def fn2(|\textcolor{orange}{a}|, |\textcolor{orange}{b}|):
    |\textcolor{olive}{d}| = |\textcolor{orange}{a}| - |\textcolor{orange}{b}|
    return (|\textcolor{violet}{c}| + |\textcolor{olive}{d}|) / 2.

print(fn2(|\textcolor{violet}{c}|, |\textcolor{violet}{d}|))
print(fn2(3, 4))
\end{minted}
정리하자면, \verb/fn1/의 매개 변수 \texttt{\textcolor{red}{a}}와
\texttt{\textcolor{red}{b}}는 아래에서 인자로 사용된
\texttt{\textcolor{magenta}{a} = 1} 및 \texttt{\textcolor{magenta}{b} = 2}와
무관합니다.  줄 7과 줄 8의 출력 값이 같다는 사실을 통해, 줄 7의
\texttt{fn1(\textcolor{magenta}{a}, \textcolor{magenta}{b})}는 단지 밖에서
정의된 \texttt{\textcolor{magenta}{a}}의 값인 1과
\texttt{\textcolor{magenta}{b}}의 값인 2를 대입한 \texttt{fn1(1, 2)}와 완전히
동일한 역할을 한다는 것을 확인할 수 있습니다.  \texttt{fn2}의 경우는
약간 다릅니다.  함수 내부에서 \texttt{c}가 지역 변수로 새로 정의되지 않은
상태로, 줄 14에서 \texttt{\textcolor{violet}{c}}를 사용하였습니다.  여기에서
쓰인 \texttt{\textcolor{violet}{c}}는 \texttt{\textcolor{violet}{c} = 3}로
정의된 전역 변수입니다.  반면 \texttt{fn2} 내부에서 사용된
\texttt{\textcolor{olive}{d}}는 전역 변수 \texttt{\textcolor{violet}{d}}가
아니라 새로 \texttt{\textcolor{orange}{a} - \textcolor{orange}{b}}의 값을
가지는 지역 변수입니다.

이와 같이 어떤 변수가 유효한 영역을 \alt*{범위}{scope}라고 합니다.
마지막 예시를 통해 변수 범위의 의미를 확실히 파악할 수 있을 것입니다.
\begin{minted}{python}
def mult(a, b):
    x = a * b
    return x

def div(a, b):
    x = a / b
    return x

a, b = 1, 2
print(mult(b, a))  # is mult(2, 1)
print(div(a, b))   # is div(1, 2)
print(x)  # NameError: name 'x' is not defined
\end{minted}
함수 \texttt{mult}와 \texttt{div}, 그리고 줄 9 이후의 \texttt{a}, \texttt{b}, \texttt{x}는 서로 무관하며, 줄 12에서는 함수 \texttt{mult}와 \texttt{div}의 변수 영역 밖에서 \texttt{x}가 정의된 적이 없으므로 오류가 발생합니다.

\section{조건문}
절댓값 함수
\[
|x| = \begin{cases}
x & \text{if $x \geq 0$}\\
-x & \text{otherwise}
\end{cases}
\]
를 앞서 배운 함수로 구현해봅시다.\footnote{이미 \texttt{abs} 내장 함수가 있지만 직접 구현해봅시다.}
\begin{minted}{python}
def abs(x):
    return (x ** 2) ** 0.5
\end{minted}
와 같이 정의할 수도 있습니다.
다만 이는 불필요하게 제곱과 제곱근을 사용한 $|x| = \sqrt{x^2}$의 정의를 사용한 것으로, 위에 작성한 정의를 따르지 않았습니다.
나아가, 인자가 정수 타입이더라도 반환값은 실수 타입으로 바뀌게 됩니다.

이러한 함수를 자연스럽게 구현하고 싶을 때 \alt*{조건문}{conditional statement}[conditional statements]을 사용할 수 있습니다:
\begin{minted}{python}
def abs(x):
    if x >= 0:
        return x
    else:
        return -x
\end{minted}

절댓값 함수처럼 \alt*{조각적으로 정의된}{piecewise-defined} 함수는 보통 조건문을 통해 구현할 수 있습니다.
나아가 조건문은 경우를 나눠서 작동하는 수많은 상황들에 적용됩니다.
예를 들어, 사용자가 로그인하였을 때와 그렇지 않았을 때 보여줘야 하는 정보가 다른 웹페이지는 조건문을 사용하여 이를 구현합니다.
낮과 밤에 따라서 화면의 밝기를 조절하는 프로그램을 작성할 때에도 조건문을 쓸 것입니다.
장애물을 맞닥뜨렸을 때 로봇이 해야 하는 행동을 지정하기 위해서도 조건문이 필요합니다.
사실상 조건문 없이는 여러 경우에 대처하는 프로그램을 작성하기 힘듭니다.

로봇 예시를 사용하자면,
\begin{minted}{python}
def check_obstacle(distance):
    if distance < 10:
        print("앞에 장애물이 있습니다.")
    else:
        print("앞에 장애물이 없습니다.")
\end{minted}
와 같은 예시를 작성할 수 있습니다.
여기서 \pyin{distance < 10}는 \verb/distance/가 $10$ 미만의 값을 가졌는지 확인하는 \alt*{조건}{condition}입니다.
이렇게 \verb|if| 뒤에 오는 조건은 \pyin{True} 혹은 \pyin{False} 두 종류의 값을
가질 수 있는데요\footnote{혹은 참과 거짓을 판단할 수 있는 기준이 정해진 다른 타입의 값들을 쓸 수도 있는데요, 지금은 무시합니다.}, 이는 \alt*{불리언}{Boolean}이라는 타입의 값입니다.

\subsection{불리언 타입과 식}
지금까지 소개한 \texttt{int}, \texttt{float} 등의 타입 이외에도 자주 쓰게 될 불리언\footnote{영국의 수학자이자 논리학자인 George Boole의 이름을 따온 것입니다.} 타입은 참과 거짓을
나타냅니다.
특이한 점으로는, 불리언 타입은 유한한 개수의 값을 가진다는 점입니다.
그냥 유한한 것도 아니고, \pyin{True}와 \pyin{False} 단 두 값만 가집니다.

위 예시의 \pyin{x >= 0}은 \pyin{True}일수도 \pyin{False}일수도
있는 불리언 타입의 식입니다.  불리언 식을 구성하기 위해서는 대소 관계, 일치
여부, 포함 관계 등을 판단하는 관계 연산자와 \pyin{and}, \pyin{or}, \pyin{not} 등의
논리 연산자로 구성됩니다.
\pyin{x >= 0 or (y < 0 and z < 0)}이 불리언 식의 예시입니다.

관계 연산자는 다음의 여섯 개가 있습니다.
\begin{itemize}
  \item \pyin{x == y}: $\texttt{x} = \texttt{y}$이면 \pyin{True}
  \item \pyin{x != y}: $\texttt{x} \ne \texttt{y}$이면 \pyin{True}
  \item \pyin{x > y}:  $\texttt{x} > \texttt{y}$이면 \pyin{True}
  \item \pyin{x < y}:  $\texttt{x} < \texttt{y}$이면 \pyin{True}
  \item \pyin{x >= y}: $\texttt{x} \ge \texttt{y}$이면 \pyin{True}
  \item \pyin{x <= y}: $\texttt{x} \ne \texttt{y}$이면 \pyin{True}
\end{itemize}
지금까지 어떤 변수 \texttt{x}에 값을 대입할 때에는 \texttt{=}를 사용하였습니다.
\texttt{x = 10}과 같이 말입니다.
수학에서 두 값의 관계에 따라 다른 논리를 풀어나가려 할 때 ``$x = 10$인 경우,"와 같은 표현을 사용하는데, 파이썬에서는 값의 일치를 비교할 때 등호 하나가 아니라 두 개를 연달아 쓴 \texttt{==}를 사용합니다.  \pyin{if x == 10}와 같이 말입니다.

\begin{table}[htbp]
  \caption{논리 연산자 \pyin{and}, \pyin{or}, \pyin{not}에 대한 진리표입니다.}\label{tab:logical-operators}
  \centering
  \begin{tblr}{colspec={ccccc}}
    \toprule
    $p$ & $q$ & $p$ \pyin{and} $q$ & $p$ \pyin{or} $q$ & \pyin{not} $p$\\
    \midrule
    \pyin{True} & \pyin{True} & \pyin{True} & \pyin{True} & \pyin{False} \\
    \pyin{True} & \pyin{False} & \pyin{False} & \pyin{True} & \pyin{False} \\
    \pyin{False} & \pyin{True} & \pyin{False} & \pyin{True} & \pyin{True} \\
    \pyin{False} & \pyin{False} & \pyin{False} & \pyin{False} & \pyin{True}\\
    \bottomrule
  \end{tblr}
\end{table}
논리 연산자는 표~\ref{tab:logical-operators}에 볼 수 있듯 \pyin{and}, \pyin{or}, \pyin{not} 세 종류가 있는데요, 각 식에 대해서 어떤 불리언 값을 가지는지에 대한 진리표가 제시되어 있습니다.
위 표에는 각 표현에 대해서 어떤 불리언 값을 가지는지 진리표가 제시되어 있습니다.
\pyin{and}, \pyin{or}, \pyin{not}은 각각 수학의 $\wedge$, $\vee$, 그리고 $\neg$에 대응됩니다.

이들 \alt*{연산자의 우선순위}[연산자 우선순위]{operator precedence}는 지난 단원의 표~\ref{tab:operators}에 정리되어 있습니다.
모든 관계 연산자들은 논리 연산자들보다 순위가 높고, 산술 연산자들보다는 낮습니다.

지금까지 불리언 자료형에 대한 논리 연산자와 그 우선순위를 살펴보았는데요, 이제
조건문과 함께 사용한 예시를 살펴봅시다.

\subsection{\texttt{if}와 \texttt{else}}
\pyin{if}와 \pyin{else}를 사용한 조건문의 기본적인 형태는 다음과 같습니다.
\begin{minted}{python}
if expression:
    ...
else:
    ...
\end{minted}
위에서 \pyin{expression}에는 \pyin{x // 10 == 0 or x < 0} 등 임의의
불리언 식이 올 수 있습니다.  만약 제시된 불리언 식의 계산된 값이 \pyin{True}라면 줄
2의 내용이 실행되고, 그렇지 않다면 줄 4의 내용이 실행됩니다.  계산된 값이
\pyin{False}일 때 실행할 내용이 없다면 \pyin{else} 부분을 빠뜨릴 수
있습니다:
\begin{minted}{python}
if expression:
    ...
\end{minted}

다음의 예시를 봅시다.
\begin{minted}{python}
x = 76
if x % 3 == 0 and x % 2 == 0:
    print(f"{x}은(는) 6의 배수입니다.")
else:
    print(f"{x}은(는) 6의 배수가 아닙니다.")
\end{minted}
실행해보면 76은 6의 배수가 아니기에 \verb/76은(는) 6의 배수가 아닙니다./가 출력됩니다.

만약 6의 배수가 아니라면, 적어도 3의 배수인지 혹은 2의 배수인지까지 판별하고
싶다면 어떻게 해야 할까요?
지금까지 배운 것만으로는 다음과 같이 할 수 있을 것입니다.
\begin{minted}{python}
x = 76
if x % 3 == 0 and x % 2 == 0:
    print(f"{x}은(는) 6의 배수입니다.")
else:
    if x % 3 == 0:
        print(f"{x}은(는) 3의 배수입니다.")
        else:
            if x % 2 == 0:
                print(f"{x}은(는) 2의 배수입니다.")
            else:
                print(f"{x}은(는) 2의 배수도 3의 배수도 아닙니다.")
\end{minted}
6의 배수인지 체크한 후, 그렇지 않다면 다시 3의 배수인지 체크한 후, 그렇지
않다면 2의 배수인지 체크하는 코드입니다.
작동은 하겠지만, 이렇게 들여쓰기가 많다면 코드를 읽기 힘들 것입니다.
이런 상황에서는 \pyin{elif}를 쓸 수 있습니다.
조건문에 세 개 이상의 \alt*{가지}{branch}가 있을 때, \pyin{if}로 시작하여 두 번째부터 마지막 바로 직전
경우까지는 \pyin{elif}를, 마지막 예외의 경우에는 \pyin{else}로 마무리하는
방식입니다.
여기서 주의해야 할 것은, \pyin{if}와 \pyin{elif} 뒤에는 항상 조건문을 써야 하고, \pyin{else} 뒤에는 조건문을 절대 쓸 수 없다는 것입니다.
조건문을 처음 배운다면 \pyin{elif} 뒤에 조건문을 안 쓰거나 \pyin{else} 뒤에 조건문을 작성하는 실수를 흔히 합니다.

\pyin{else}는 어떠한 경우의 조건에도 해당하지 않을 때 실행할 코드를 쓰는 조각입니다.
모든 경우에 해당하지 않을 때 시행할 것이 없으면 마지막
\pyin{else}는 생략할 수 있습니다.

\pyin{elif}를 활용하면 위의 코드를 아래와 같이 간결하게 나타낼 수 있습니다.
\begin{minted}{python}
x = 76
if x % 3 == 0 and x % 2 == 0:
    print(f"{x}은(는) 6의 배수입니다.")
elif x % 3 == 0:
    print(f"{x}은(는) 3의 배수입니다.")
elif x % 2 == 0:
    print(f"{x}은(는) 2의 배수입니다.")
else:
    print(f"{x}은(는) 2의 배수도 3의 배수도 아닙니다.")
\end{minted}
만약 \pyin{if ... elif ... elif ... else}를 훑는 중간에 참인 조건이 나오면, 해당
조건에서 실행되는 명령을 수행한 후 이후의 경우는 모두 건너뛰게 됩니다.

참고로, \pyin{if-else}문에서 불리언 식 대신 \pyin{int} 등의 타입을 가진
변수를 넣어도 됩니다. 예컨대
\texttt{0}은 거짓, \texttt{0} 이외의 모든 수는 참의 값을 가집니다.
그렇다고 이러한 값들이 항상
\pyin{True} 혹은 \pyin{False}의 값을 가지는 것은 아닙니다.\footnote{실제로
\pyin{0 == False}, \pyin{1 == True}입니다.}
예컨대 불리언 식 \pyin{2 == True}의 값은 \pyin{False}입니다.
또한 문자열의 경우 \pyin{""} 외의 하나 이상의 문자를 담고 있는 문자열은 참입니다.
\pyin{""}는 거짓에 해당합니다.
그렇다고 \pyin{"" == False}인 것은 아닙니다.
이를 통해 특정 경우에 좀 더 \alt*{파이써닉}{Pythonic}한 코드를 작성할 수도 있습니다.
하지만 이것도 가독성을 해치지 않는 선에서 써야 합니다.\footnote{커누스는 ``Programs are meant to be read by humans and only incidentally for computers to execute''라고 말한 적이 있습니다.}

위에서 \pyin{True}와 \pyin{False}가 각각 \verb/1/과 \verb/0/의 값을 가진다는 사실을 사용해 다음과 같은 장난도 칠 수는 있습니다:
\begin{minted}{python}
>>> True * 3
3
>>> True * False
0
>>> (6 % 2 == 0) + False
1
\end{minted}
어디까지나 이것이 \emph{가능하다는} 사실을 알려주는 코드이고, 실제로는 이렇게 쓰면 큰 봉변을 당할 수도 있으니 주의하세요.

다만 위에서 \verb/0/이 아닌 \pyin{int}가 참인 것을 사용해 다음과 같이 간결한 코드를 작성할 수 있습니다:
\begin{minted}{python}
money = 100
if money:
    print("I have money")
else:
    print("I don't have money")
\end{minted}

\subsection{다중 반환점}
한 함수에 \pyin{if-else}문을 넣어 경우에 따라 다른 값을 반환하도록\myindex{반환}{return} 할 수 있습니다.
반환한다는 것은 함수에서 호출한 곳으로 값을 돌려주는 것이었습니다.
즉, \pyin{if-else}문을 사용해 함수 안 여러 곳에서 값을 돌려줄 수 있습니다.
위에서 작성된 \verb/abs/ 함수가 그 예시로, \pyin{x >= 0}일 경우 \pyin{x}를 반환하고 그렇지 않은 경우
\pyin{-x}를 반환하였습니다.  이러한 함수에는 여러 \pyin{return}을
사용하였기 때문에 \alt*{다중 반환점}{multiple exit points}이 있다고 표현합니다.
만약 \pyin{abs} 함수를 하나의 반환점만 사용한다면 아래와 같게 다시 쓸 수 있습니다.
\begin{minted}{python}
def abs(x):
    if x >= 0:
        result = x
    else:
        result = -x
    return result
\end{minted}
혹은 \pyin{else} 가지 없이
아래와 같이 다시 쓸 수도 있습니다.
\begin{minted}{python}
def abs(x):
    if x >= 0:
        return x
    return -x
\end{minted}
위 코드의 경우에는 가능한 경우를 모두 ``걸러서'' 마지막 경우에는 \texttt{-x}를
반환하라고 읽을 수 있습니다.
함수를 실행할 때, \pyin{return}을 만나는 순간 함수가 해당 값을 반환하고 종료되기 때문에 위와 같은 코드를 작성할 수 있습니다.
즉, \emph{\pyin{return}은 함수에서 값을 반환하는 동시에, 함수의 수행을
종료하라는 표시로 볼 수 있습니다.}

다중 반환점을 사용할 경우, 일반적으로는 다음과 같이 함수를 만들 수 있습니다.
\begin{minted}{python}
def foo(x):
    if ...:
        return ...
    ...
    if ...:
        return ...
    return ...
\end{minted}

\section{예제}
\begin{enumerate}
  \item 이름 \verb|name|을 문자열로 받아서 \verb|제 이름은 {name}입니다.|와 같이
    출력하는 함수 \verb|introduce(name)|를 작성하세요.
\begin{minted}{python}
def introduce(name):
    # Add here!

print(introduce("귀도 반 로섬"))  # 제 이름은 귀도 반 로섬입니다.
print(introduce("앨런 튜링"))     # 제 이름은 앨런 튜링입니다.
print(introduce("폰 노이만"))     # 제 이름은 폰 노이만입니다.
\end{minted}

  \item 두 수 \verb|a|와 \verb|b|를 받아 덧셈, 뺄셈, 곱셈, 나눗셈을 해주는 함수
    \verb|add(a, b)|, \verb|subtract(a, b)|, \verb|multiply(a, b)|,
    \verb|divide(a, b)|를 각각 작성하세요.

\begin{minted}{python}
def add(a, b):
    # Add here!

def subtract(a, b):
    # Add here!

def multiply(a, b):
    # Add here!

def divide(a, b):
    # Add here!


print(add(1, 2))       # 3
print(subtract(1, 2))  # -1
print(multiply(1, 2))  # 2
print(divide(1, 2))    # 0.5
\end{minted}

  \item 정규분포를 표현하기 위해 사용되는 가우시안 함수는
    \[
    g(x) = \frac{1}{\sigma \sqrt{2\pi}} e^{-\frac12 \left(\frac{x - \mu}{\sigma}\right)^2},
    \]
    이며, 이때 $\mu$는 기댓값, $\sigma$는 표준편차입니다.
    가우시안 함수 \verb|gaussian(mu, sigma, x)|를 작성하세요.
\begin{minted}{python}
import math

def gaussian(mu, sigma, x):
    # Add here!

print(gaussian(0, 1, 0))                # 0.3989422804014327
print(gaussian(-2, math.sqrt(0.5), 0))  # 0.010333492677046037
\end{minted}

\item 전하가 각각 $q_1$과 $q_2$인 두 물체 사이의 쿨롱힘은 거리 $r$에 따른 함수
\[
  F_C(r) = \frac{1}{4\pi \varepsilon_0} \frac{q_1 q_2}{r^2}\ \text{where $\varepsilon_0 = 8.854\,187\,817 \times 10^{-12}\ \mathrm{F \cdot m^{-1}}$}.
\]
로 쓸 수 있습니다.
또한, 질량이 각각 $m_1$, $m_2$인 두 물체 사이의 만유인력은 거리 $r$에 따른 함수
\[
  F_g(r) = G \frac{m_1 m_2}{r^2}\ \text{where $G = 6.674\,08 \times 10^{-11}\ \mathrm{m^3 \cdot kg^{-1} s^{-2}}$}.
\]
로 표현됩니다.
두 물체가 전하 \texttt{q1}과 \texttt{q2}, 그리고 질량 \texttt{m1}과
\texttt{m2}를 가지고, 거리 \texttt{r}만큼 떨어져있다고 가정합니다.
각각 쿨롱힘과 만유인력을 구하여 반환하는 함수 \verb|coulomb(q1, q2, r)|과
\verb|gravity(m1, m2, r)|을 작성하세요.
또한 이 둘 모두를 구하여 더하는 \verb/total_force(q1, q2, m1, m2, r)/도
작성하세요.
\begin{minted}{python}
import math

def coulomb(q1, q2, r):
    # Add here!

def gravity(m1, m2, r):
    # Add here!

def total_force(q1, q2, m1, m2, r):
    # Add here!

e_c = -1.6021766208e-19
e_m = 9.10938356e-31
p_c = -e_c
p_m = 1.672621898e-27
a_0 = 5.2917721067e-11

print(coulomb(e_c, p_c, a_0))
print(gravity(e_m, p_m, a_0))
print(total_force(e_c, p_c, e_m, p_m, a_0))
\end{minted}

\item  $a \neq 0$일 경우, 함수 $f(x) = ax^2 + bx + c$는 극값을 가집니다.
함수 $f(x) = \texttt{a}x^2 + \texttt{b}x + c$의 극값을 반환하는
\verb/extremum(a, b, c)/를 작성하세요.
\begin{minted}{python}
def f(a, b, c, x):
    return a * x ** 2 + b * x + c

def extremum(a, b, c):
    # Add here!

print(extremum(1, 5, 10))   # 3.75
print(extremum(1, -5, 10))  # 3.75
print(extremum(3, 7, 5))    # 0.916666666667
\end{minted}

\item 세 수 \verb|a|, \verb|b|, \verb|c| 중에서 가장 작은 수를 반환하는
  \verb/least(a, b, c)/를 작성하세요.
\begin{minted}{python}
# Add here!

print(least(1, 2, 3))  # 1
print(least(2, 1, 3))  # 1
print(least(1, 1, 2))  # 1
\end{minted}

\item 입력 받은 수 \verb|n|이 짝수이면 \verb|{n}은 짝수입니다.|, 홀수이면
  \verb|{n}은 홀수입니다.|를 출력하는 함수 \verb|check_parity(n)|을 작성하세요.
\begin{minted}{python}
# Add here!

check_parity(2)    # 2는 짝수입니다.
check_parity(101)  # 101은 홀수입니다.
\end{minted}

\item 입력 받은 문자 \verb|direction|이 \verb|w|일 경우 \verb|전진|,
  \verb|a|일 경우 \verb|좌회전|, \verb|s|일 경우 \verb|후진|, \verb|d|일 경우
  \verb|우회전|을 출력하는 함수 \verb|navigate(direction)|을 작성하세요.
  단, 이외의 문자가 입력되었을 경우에는 \verb|알 수 없는 명령입니다.|를
 \begin{minted}{python}
# Add here!

navigate('w')  # 전진
navigate('a')  # 좌회전
navigate('s')  # 후진
navigate('d')  # 우회전
navigate('z')  # 알 수 없는 명령입니다.
\end{minted}

\item 다음을 반환하는 함수 \texttt{quadrant(x, y)}를 작성하세요:
\begin{itemize}
  \item \texttt{"제1사분면"} if $\texttt{x} > 0 \wedge \texttt{y} > 0$
  \item \texttt{"제2사분면"} if $\texttt{x} < 0 \wedge \texttt{y} > 0$
  \item \texttt{"제3사분면"} if $\texttt{x} < 0 \wedge \texttt{y} < 0$
  \item \texttt{"제4사분면"} if $\texttt{x} > 0 \wedge \texttt{y} < 0$
\item \texttt{"경계선"} otherwise
\end{itemize}
\begin{minted}{python}
# Add here!

print(quadrant(10, 5))   # 제1사분면
print(quadrant(-5, 3))   # 제2사분면
print(quadrant(-5, -7))  # 제3사분면
print(quadrant(3, -5))   # 제4사분면
print(quadrant(0, -3))   # 경계선
\end{minted}

\item 10000보다 작은 자연수 \texttt{n}에 대해서, 각 자릿수를 거꾸로 출력하는
  함수 \texttt{reverse(n)}를 작성하세요.
\begin{minted}{python}
# Add here!

print(reverse(3702))  # 2073
print(reverse(370))   # 73
print(reverse(37))    # 73
print(reverse(3))     # 3
\end{minted}

\item 10 이상 990 이하의 10의 배수 \verb|n|이 주어졌을 때, 10, 50, 100, 500원
  동전을 최소로 사용해서 \verb|n|원을 만드는데 사용되는 동전의 수를 반환하는
  함수 \verb|count_coins(n)|을 작성하세요.
\begin{minted}{python}
# Add here!

print(count_coins(730))  # 6
print(count_coins(790))  # 8
print(count_coins(260))  # 4
print(count_coins(70))   # 3
\end{minted}

\item  함수 \texttt{ceiling(x)}이 $\lceil\texttt{x}\rceil$을
  반환하도록 작성하세요.
\begin{minted}{python}
# Add here!

print(ceil(4.3))   # 5
print(ceil(-0.3))  # 0
print(ceil(0.01))  # 1
\end{minted}

\item  Write a function \verb/min_x(a, b, c)/ that returns the integer $x$ that minimizes $\texttt{a}x^2 + \texttt{b}x + \texttt{c}$.
Assume that \texttt{a}, \texttt{b}, and \texttt{c} are \texttt{float} type, where $\texttt{a} > 0$ and $\texttt{b} < 0$.
If such $x$ is not unique, return the smallest one.
\begin{minted}{python}
def f(a, b, c, x):
    return a*x**2 + b*x + c

def min_x(a, b, c):
    # Add here!

print(min_x(1, -9, 2))    # 4
print(min_x(9, -5, 0))    # 0
print(min_x(9, -15, 0))   # 1
print(min_x(7, -13, 3))   # 1
\end{minted}

\item  Write a function \verb/area_triangle(a, b)/ that returns the area of a
  triangle formed by $y = \texttt{a}x + \texttt{b}$, $x$-axis, and $y$-axis.
  Return 0 if no triangle is formed. Assume \texttt{a} and \texttt{b} are
  either \pyin{int} or \pyin{float} type.

\begin{minted}{python}
def area_triangle(a, b):
    # Add here!

\end{minted}

\item \texttt{"S"}, \texttt{"R"}, and \texttt{"P"} represent scissors, rock,
  and paper, respectively.  Write a function \pyin{rock_paper_scissors(a,
  b)} that returns \texttt{"a"} if \texttt{a} wins, \texttt{"b"} if \texttt{b}
  wins, or \texttt{Tie} if the game is tied, where \texttt{a} and \texttt{b}
  are elements of $\{\texttt{"S"}, \texttt{"R"}, \texttt{"P"}\}$.
\begin{minted}{python}
def rock_paper_scissors(a, b):
    if a == b:
        return "Tie"
    # Add here!

print(rock_paper_scissors("R","R"))  # Tie
print(rock_paper_scissors("R","S"))  # a
print(rock_paper_scissors("R","P"))  # b
print(rock_paper_scissors("S","S"))  # Tie
print(rock_paper_scissors("S","P"))  # a
print(rock_paper_scissors("S","R"))  # b
print(rock_paper_scissors("P","P"))  # Tie
print(rock_paper_scissors("P","R"))  # a
print(rock_paper_scissors("P","S"))  # b
\end{minted}

\item \texttt{gold1}, \texttt{silver1}, and \texttt{bronze1} represent the
  numbers of gold, silver, and bronze medals of the first country,
  respectively.  \texttt{gold2}, \texttt{silver2}, and \texttt{bronze2}
  represent the numbers of gold, silver, and bronze medals of the second
  country, respectively.  Write a function \texttt{better(gold1, silver1,
  bronze1, gold2, silver2, bronze2)} that returns \texttt{"First"} if the first
  country achieved better than the second, \texttt{"Second"} if the second
  country achieved better, or \texttt{"Tie"} if they tied.  The score is
  evaluated according to the gold-silver-bronze order, not by the sum of total
  medals.
\begin{minted}{python}
def better(gold1, silver1, bronze1, gold2, silver2, bronze2):
    if gold1 > gold2:
        return "First"
    # Add here!

print(better(10,4,24, 1,35,25))  # First
print(better(1,35,25, 10,4,24))  # Second
print(better(10,18,0, 10,4,24))  # First
print(better(10,4,24, 10,18,0))  # Second
print(better(10,20,5, 10,20,4))  # First
print(better(10,20,4, 10,20,5))  # Second
print(better(10,20,5, 10,20,5))  # Tie
\end{minted}

\end{enumerate}
\end{document}
