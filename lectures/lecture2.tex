\documentclass[../main.tex]{subfiles}

\begin{document}

\subsection{Functions함수}
오늘날 프로그래밍의 기본 중 기본은 functions함수라고 말할 수 있습니다.  함수를
전혀 사용하지 않고 프로그래밍을 하는 것이 가능은 하지만, 함수를 쓰면 다양한
이점이 있습니다.  일단 프로그램을 역할에 따라 여러 부분으로 나누어 구조적
프로그래밍\footnote{구조적 프로그래밍은 비구조적 프로그래밍과 대비되는
프로그래밍 패러다임입니다. 비구조적 프로그래밍은 프로그램 전체를 하나의 연속된
코드로 작성하는 방식입니다. 이러한 프로그래밍 방식에서는 GOTO문과 같은 제어문에
의존할 수 밖에 없는데, 이는 디버깅이 어렵고 가독성이 떨어집니다. 특히
구조화되지 않은 코드는 프로그램이 복잡해지면 수행 방향이 복잡하게 얽히게
되는데, 이를 비유적으로 `스파게티 코드'라고 표현하기도 합니다. machine
language기계어와 assembly language어셈블리어를 제외한 현대의 대부분 프로그래밍
언어는 구조적 프로그래밍을 지원합니다.}이 가능해집니다.  또한 함수를 사용하면
같은 코드를 반복하지 않음으로써 프로그램의 용량을 줄이고 범용성을 늘릴 수
있습니다.  지난 시간에 자주 반복하여 사용하는 식을 하나의 변수에 대입하여
사용하면 효율적이라고 했던 것과 같은 맥락입니다.  나중에 객체 지향 프로그래밍
패러다임을 배우면 더 자세히 배우겠지만, 함수를 사용하여도 encapsulation캡슐화가
가능해집니다.  캡슐화란 함수를 한 번 구현해두면 내용을 모르더라도 그 기능을
사용할 수 있게 하는 것이라고 생각하면 됩니다.

한 번 Python에서 함수를 어떻게 정의하고 사용하는지, 아래의 예시를 통해서 살펴봅시다.
\begin{minted}[mathescape,
               linenos,
               breaklines,
               numbersep=5pt,
               frame=lines,
               framesep=2mm]{python}
def gamma(v):
    c = 299792458
    b = v / c
    return 1 / (1 - b**2)**0.5

def t_rel(x, t, v):
    c = 299792458
    return gamma(v) * (t - v * x / c**2)

pos = 10
time = 13
vel = 3E7

print(gamma(vel))
print(t_rel(pos, time, vel))
\end{minted}
일단 함수를 사용하기 위해서는 함수를 정의해야 합니다.
함수의 정의는 \texttt{def function(arg1, arg2, ...)}의 형태로 시작해서
\texttt{return value}으로 끝납니다. 이때 \verb|function|은 원하는 함수명이고,
\verb|arg1|, \verb|arg2|, $\cdots$은 함수의 인자에 해당하는 변수명입니다.
함수를 호출(사용)할 때에는 그냥 \texttt{function(a, b, c)}의 형태로 사용하시면
됩니다.
또한, 예시처럼 매개 변수의 수는 여러 개가 될 수 있습니다.
수학의 다변수 함수와 유사하게 받아들이면 됩니다.
본격적으로 함수를 어떻게 정의하고 사용해야 하는지 알아봅시다.

\subsubsection{Indentation들여쓰기}
Python에서는 indentation들여쓰기가 구문을 나누는 중요한 syntax입니다.  C, C++
등의 언어에서는 들여쓰기를 하지 않아도 중괄호\verb|{·}|와
세미콜론 \verb|;|으로 명령을 구분하고, 들여쓰기는 완전히 가독성만을 위한
것입니다.  그러나 중괄호나 세미콜론을 사용하지 않는 Python에서는 들여쓰기가
반드시 필요합니다. 띄여쓰기를 4회 하여 들여쓰기를 하던가, 탭을 통해서 들여쓰기를
할 수도 있습니다.\footnote{띄어쓰기를 하여 들여쓰기를 할것인지, 탭을 하여
들여쓰기를 할 것인지는 프로그래머들 사이에서 펼쳐지는 오랜 논쟁입니다.}
띄어쓰기 4회를 통해 들여쓰기를 한다고 스페이스 키를 네 번 누를 필요는 없습니다.
IDE나 텍스트 에디터 설정에서 탭 키를 soft tab (띄어쓰기 여러 개를 사용하여
들여쓰기를 하는 것)으로 설정하던지 hard tab (탭을 사용하여 들여쓰기를 하는
것)으로 설정하면 됩니다.  Python의 코딩 스타일 가이드인 PEP-8에서는 soft tab이
권장됩니다.

위에서 언급하였듯이, Python에서 들여쓰기는 필수적인 문법입니다.  함수뿐만이
아니라 이번 시간에 배울 조건문, 나중에 배울 반복문 등에서 들여쓰기는 구문을
구별하고 포함 관계를 나타내는 역할을 합니다.  위의 예시처럼 함수에서는
\texttt{def} 다음 줄부터 \texttt{return}의 해당 줄까지 한 수준 들여쓰기를 해야
합니다.  \textbf{들여쓰기를 실수로 의도한 바와 다르게 한다면 코드의 수행 결과가
  완전히 달라질 수 있습니다.  나아가 문법적으로는 맞을 수 있어서 에러 메시지는
뜨지 않고 결과만 차이가 생기는 경우도 빈번하니 유의해야 합니다.} 따라서
들여쓰기를 명시적으로 나타내주는 IDE의 기능을 활용하면 생산성을 높일 수
있습니다.  IDE의 기능에서 기본 중의 기본이니 대부분의 IDE에서 들여쓰기를
도와주는 기능이 있을 것입니다.  Wing IDE에서는 Preferences $>$ Editor $>$
Indentation으로 들어가면 Show Indent Guides 항목에 체크를 하면 됩니다.
PyCharm에서는 기본적으로 indentation guide를 보여줄 것입니다.

\subsubsection{Built-in Functions내장 함수}
사실 이번에 함수를 처음 접하는 것이 아닙니다.  지난 시간에도 여러 함수를
보았는데요, \texttt{type($\cdot$)}이나 형 변환을 위해 사용한
\texttt{int($\cdot$)} 등이 바로 함수입니다.  이러한 함수는 사용자가 직접
정의하지 않아도 사용할 수 있는 built-in functions내장 함수입니다.  또한 어떤
내장 함수들은 바로 사용할 수 있는 것이 아니라, `이러이러한 종류의 함수들을
사용할 것이다'라는 것을 코드에 작성하여야 쓸 수 있는 것들이 있습니다.  바로
\texttt{import package}을 (보통) 코드 맨 윗 줄에 적어주는 것입니다.  특히
수식 계산을 위해 자주 사용할 패키지는 \texttt{math} 패키지로, 제곱근 함수
\texttt{sqrt($\cdot$)}이라든지 사인 함수 \texttt{sin($\cdot$)}와 같은 삼각
함수, 로그 함수 \texttt{log($\cdot$)}을 사용하기 위해서는  \texttt{math}
패키지를 import해야 합니다.  지난 단원의 마지막 문제인 올림 함수는 \verb|math|
패키지의 \texttt{ceil($\cdot$)}\footnote{Ceiling function올림 함수의 앞에서 따온
명칭입니다.} 함수를 사용하면 됩니다.  마찬가지로 \texttt{int($\cdot$)}와 같은
`가짜' 내림 함수가 아니라 진짜 내림 함수인 \texttt{floor($\cdot$)}가
제공됩니다.
Python 2에서는 \verb|math| 패키지에서 제공되던 반올림
함수\texttt{round($\cdot$)}는 이제 \verb|math| 없이도 제공됩니다.  아래에서
사용 예시를 봅시다.
\begin{minted}[mathescape,
               linenos,
               breaklines,
               numbersep=5pt,
               frame=lines,
               framesep=2mm]{python}
>>> import math
>>> math.sqrt(4)
2.0
>>> math.sin(30 * math.pi / 180)
0.49999999999999994
>>> math.log(math.e)
1.0
>>> math.ceil(-1.5)
-1
>>> math.floor(-1.5)
-2
>>> round(-1.5)
-2
\end{minted}

만약 \texttt{math.\dots($\cdot$)}를 반복하는 것이 귀찮다면, \texttt{from 패키지
명 import *}로 import를 하면 아래와 같이 계속 \texttt{math.}를 앞에 붙일 필요가
없어집니다.
\begin{minted}[mathescape,
               linenos,
               breaklines,
               numbersep=5pt,
               frame=lines,
               framesep=2mm]{python}
>>> from math import *
>>> sqrt(4)
2.0
>>> sin(30. * pi / 180.)
0.49999999999999994
>>> log(e)
1.0
\end{minted}
단, \texttt{sin} 등의 함수를 사용자가 새로 정의하거나, \texttt{pi}나 \texttt{e}와 같은 값을 새로 지정한다면 값이 덮어 씌워져서 실수할 가능성이 커집니다.
\begin{minted}[mathescape,
               linenos,
               breaklines,
               numbersep=5pt,
               frame=lines,
               framesep=2mm]{python}
>>> from math import *
>>> e
2.718281828459045
>>> e = 1
>>> e
1
\end{minted}
또한 같은 함수 명 \texttt{fn}을 지닌 두 개의 패키지 \texttt{pack1}과
\texttt{pack2}에서 \texttt{from pack1 import *}를 한 후 \texttt{from pack2
import *}를 한다면 뒤에서 불러온 \texttt{pack2}의 \texttt{fn}이 먼저 불러온
\texttt{pack1}의 \texttt{fn}을 덮어 씌울 것입니다.  예컨데 \texttt{numpy}라는
패키지를 컴퓨터에 설치한 후, \texttt{from numpy import *}와 \texttt{from math
import *}를 한다면 상당수의 함수가 겹치게 되어 문제가 생길 가능성이 높습니다.
따라서 문제가 생기지 않을 것이라고 확신하거나 불가피할 경우에만 \texttt{from
\dots import *}를 사용하는 것이 권장됩니다.  나아가 자신이 \texttt{math}
패키지에서 \texttt{sin($\cdot$)} 함수만 필요하다는 경우에는 \texttt{from math
import sin}을 하여 바로 \texttt{math.} 없이 \texttt{sin} 함수를 사용할 수
있습니다.\footnote{보통 \texttt{import numpy as np}로 numpy 패키지를
불러옵니다. 이러한 경우, \texttt{np.sin}과 같은 형식으로 패키지를 사용할 수
있습니다.}

\subsubsection{User-Defined Functions사용자 정의 함수}
본인이 원하는 함수가 없다면 직접 정의를 해서 함수를 만들 수 있습니다.
함수는 반환 값이 있는 것과 그렇지 않은 것 두 종류가 있습니다.
아래는 원의 반지름을 받아서 면적을 반환하는 함수
\texttt{circ\_area($\cdot$)}입니다.
\begin{minted}[mathescape,
               linenos,
               breaklines,
               numbersep=5pt,
               frame=lines,
               framesep=2mm]{python}
import math

def circ_area(radius):
    return math.pi * radius**2
\end{minted}
\texttt{return} 다음에 반환할 값이 나타나 있습니다.
반면 아래는 원의 면적을 반환하지 않고 단순히 출력을 하는 함수
\texttt{print\_circ\_area($\cdot$)}입니다.
\begin{minted}[mathescape,
               linenos,
               breaklines,
               numbersep=5pt,
               frame=lines,
               framesep=2mm]{python}
import math

def print_circ_area(radius):
    print(math.pi * radius**2)
\end{minted}
\texttt{circ\_area($\cdot$)}의 경우에는 원의 면적을 출력하기 위해
\texttt{print}를 사용해야 합니다.  그렇지만 다른 값을 계산하는 등의 작업에서
함수를 표현에 넣을 수 있습니다.  \texttt{print\_circ\_area($\cdot$)}는
직접적으로 원의 면적을 출력하지만, 그 값을 다른 표현에서 활용할 수는 없습니다.
반환하는 값이 없기 때문입니다.

함수를 정의하여 사용할 때에는, 사용하기 위해 호출하기 이전에 정의를 하기만 하면
됩니다.  위 조건만 만족한다면 다른 코드들 사이에 끼워두어도 되고, 함수들 정의
순서도 상관이 없습니다.  또한 함수를 정의하기 위해서 다른 함수를 사용할 수
있습니다.
\begin{minted}[mathescape,
               linenos,
               breaklines,
               numbersep=5pt,
               frame=lines,
               framesep=2mm]{python}
import math

def get_dist(x1, y1, x2, y2):
    dx = x1 - x2
    dy = y1 - y2
    return (dx**2 + dy**2)**0.5

def circ_area(x1, y1, x2, y2):
    r = get_dist(x1, y1, x2, y2)
    return math.pi * r**2

print(circ_area(0, 0, 3, 4))
\end{minted}
위 코드의 \texttt{circ\_area($\cdot$)} 함수를 정의하기 위해
\texttt{get\_dist($\cdot$)} 함수를 사용하였는데, \texttt{get\_dist($\cdot$)}
함수의 정의와 \texttt{circ\_area($\cdot$)} 함수 정의의 순서는 바뀌어도 상관은
없습니다.  하지만 코드를 읽는 사람이 위부터 읽어 내려갈 때 모르는 함수가 있으면
코드를 이해하는데 어려움이 생길 수 있으니 가독성을 위해서 순서를 적당히
지켜주는 것이 좋을 것 같습니다.

코드의 뼈대 부분을 \texttt{main()} 등의 함수로 묶어두고, 함수 정의 밖에는
\texttt{main()} 한 번만 호출되게 할 수도 있습니다.  이는 C/C++, Java 등의
언어에서는 기본적으로 채택되는 방식입니다.  예컨대, C에서는 \texttt{Hello,
world!}를 출력하는 코드가 다음과 같습니다.
\begin{minted}[mathescape,
               linenos,
               breaklines,
               numbersep=5pt,
               frame=lines,
               framesep=2mm]{C}
#include <stdio.h>

int main()
{
    printf("Hello, world!\n");
    return 0;
}
\end{minted}
Python 코드에서는 함수 없이 가장 윗 수준에 적힌 코드가 실행되는 반면, C에서는
\texttt{main()} 함수에 적힌 내용이 실행됩니다.  하지만 Python에서도 다음과 같이
코드를 구성할 수 있습니다.
\begin{minted}[mathescape,
               linenos,
               breaklines,
               numbersep=5pt,
               frame=lines,
               framesep=2mm]{python}
def main():
    print("Hello, world!")

main()
\end{minted}
방금 정의한 \texttt{main()} 함수는 매개 변수를 가지지 않는데, 이러한 종류의
함수는 매개 변수 없이 항상 동일한 기능을 하게 됩니다.

\subsubsection{Parameters매개 변수와 Arguments전달 인자}
함수를 정의할 때 사용되는 매개 변수들은 함수 정의 밖에서 호출될 수 없습니다.
일종의 블랙 박스로 생각하면 되는데, 함수 안에서 정의되는 local variable지역
변수는 그 안에서만 사용을 할 수 있습니다.  예를 들어 예시에서 본
\texttt{get\_dist($\cdot$)} 함수에서 \texttt{dx}나 \texttt{dy}는 함수 안에서만
생성되었다가 사라지는 변수라고 생각하면 됩니다.  나아가 함수 밖에서
\texttt{dx}라는 이름의 새로운 변수를 만들더라도 이는 함수 안에서 정의되었던
지역 변수 \texttt{dx}와는 다른 변수입니다.  반면, 함수 밖에서 먼저 정의가
되었으면서 함수 안에서 다시 정의되지 않은 변수는 global variable전역 변수라고
합니다.  이러한 경우에는 함수 밖에서 정의가 된 값이 그대로 함수 안에서
사용됩니다.  또한 함수의 매개 변수도 함수 밖에서 사용될 수 없는 지역
변수입니다.  매개 변수의 이름과 함수의 밖에서 정의된 어떤 변수의 이름이
같더라도 이 둘은 관련이 없습니다.  물론, 해당 변수에 지정된 값이 같은 이름의
매개 변수로 정의된 함수의 arguments전달 인자로 지정될 수는 있습니다.  매개
변수는 항상 전달 인자로 값이 지정되기 때문에 전역 변수와 이름이 겹친다고 해도
영향을 받지 않습니다.  여기서 전달 인자는 \texttt{fn(3)}의 3처럼 단순히 함수에
전달되는 값입니다.  아래의 예시를 통해 좀 더 명확히 이해를 해보시길 바랍니다.
이해를 위해 같은 이름을 가졌더라도 가리키는 값이 다르면 다른 색상으로
표시하였습니다.
\begin{minted}[mathescape,
               escapeinside=||,
               linenos,
               breaklines,
               numbersep=5pt,
               frame=lines,
               framesep=2mm]{python}
def fn1(|\textcolor{red}{a}|, |\textcolor{red}{b}|):
    |\textcolor{brown}{c}| = |\textcolor{red}{a}| + |\textcolor{red}{b}|
    |\textcolor{brown}{d}| = |\textcolor{red}{a}| - |\textcolor{red}{b}|
    return (|\textcolor{brown}{c}| + |\textcolor{brown}{d}|) / 2.

|\textcolor{magenta}{a}|, |\textcolor{magenta}{b}| = 1, 2
print(fn1(|\textcolor{magenta}{a}|, |\textcolor{magenta}{b}|))
print(fn1(1, 2))

|\textcolor{violet}{c}|, |\textcolor{violet}{d}| = 3, 4

def fn2(|\textcolor{orange}{a}|, |\textcolor{orange}{b}|):
    |\textcolor{olive}{d}| = |\textcolor{orange}{a}| - |\textcolor{orange}{b}|
    return (|\textcolor{violet}{c}| + |\textcolor{olive}{d}|) / 2.

print(fn2(|\textcolor{violet}{c}|, |\textcolor{violet}{d}|))
print(fn2(3, 4))
\end{minted}
정리하자면, \texttt{fn1($\cdot$)}의 매개 변수 \texttt{\textcolor{red}{a}}와
\texttt{\textcolor{red}{b}}는 아래에서 전달 인자로 사용된
\texttt{\textcolor{magenta}{a} = 1} 및 \texttt{\textcolor{magenta}{b} = 2}와
무관합니다.  줄 7과 줄 8의 출력 값이 같다는 사실을 통해, 줄 7의
\texttt{fn1(\textcolor{magenta}{a}, \textcolor{magenta}{b})}는 단지 밖에서
정의된 \texttt{\textcolor{magenta}{a}}의 값인 1과
\texttt{\textcolor{magenta}{b}}의 값인 2를 대입한 \texttt{fn1(1, 2)}와 완전히
동일한 역할을 한다는 것을 확인할 수 있습니다.  \texttt{fn2($\cdot$)}의 경우는
약간 다릅니다.  함수 내부에서 \texttt{c}가 지역 변수로 새로 정의되지 않은
상태로, 줄 14에서 \texttt{\textcolor{violet}{c}}를 사용하였습니다.  여기에서
쓰인 \texttt{\textcolor{violet}{c}}는 \texttt{\textcolor{violet}{c} = 3}로
정의된 전역 변수입니다.  반면 \texttt{fn2($\cdot$)} 내부에서 사용된
\texttt{\textcolor{olive}{d}}는 전역 변수 \texttt{\textcolor{violet}{d}}가
아니라 새로 \texttt{\textcolor{orange}{a} - \textcolor{orange}{b}}의 값을
가지는 지역 변수입니다.  이와 같이 어떤 변수가 유효성을 가지는 영역을 scope변수
영역이라고 합니다.  마지막 예시를 통해 변수 영역의 의미를 확실히 파악할 수 있을
것입니다.
\begin{minted}[mathescape,
               linenos,
               breaklines,
               numbersep=5pt,
               frame=lines,
               framesep=2mm]{python}
def mult(a, b):
    x = a * b
    return x

def div(a, b):
    x = a / float(b)
    return x

a, b = 1, 2
print(mult(b, a))  # mult(2, 1)
print(div(a, b))  # div(1, 2)
print(x)  # NameError: name 'x' is not defined
\end{minted}
함수 \texttt{mult($\cdot$)}와 \texttt{div($\cdot$)}, 그리고 줄 9 이후의 \texttt{a}, \texttt{b}, \texttt{x}는 서로 무관하며, 줄 12에서는 함수 \texttt{mult($\cdot$)}와 \texttt{div($\cdot$)}의 변수 영역 밖에서 \texttt{x}가 정의된 적이 없으므로 오류가 발생합니다.

\subsection{Conditionals조건문}
그렇다면, 입력값(전달 인자)에 따라서 반환값이 달라지는 함수를 만들 때에는 어떻게 해야 할까요?
단순히, 절댓값 함수를 구현하려고 해도 다음과 같이 경우가 갈리게 됩니다.
$$
|x| = \begin{cases}
x & \text{if $x \geq 0$}\\
-x & \text{otherwise}
\end{cases}
$$
이러한 함수를 구현하고 싶을 때 활용해야 하는 것이 바로 conditionals조건문입니다.
내장 함수인 \texttt{abs($\cdot$)}를 사용할 수도 있지만, 절댓값 함수는 다음과
같이 구현할 수 있습니다.
\begin{minted}[mathescape,
               linenos,
               breaklines,
               numbersep=5pt,
               frame=lines,
               framesep=2mm]{python}
def abs(x):
    if x >= 0:
        return x
    else:
        return -x
\end{minted}

위와 같이, 수학에서 정의역에 따라 다르게 정의되는 조각적-정의된
(piecewise-defined) 함수를 코드로 작성하기 위해서는 조건문이 필요합니다.
하지만 조건문은 이와 같이 수식을 계산하는 일을 넘어, 경우를 나눠야 하는 모든
종류의 상황에 적용됩니다.
예를 들어, 어떤 프로그램에서 사용자가 로그인하였을 때와 그렇지 않았을 때
보여줘야 하는 항목이 다를 경우에 조건문을 사용해야 합니다.
낮과 밤에 따라서 화면의 밝기를 조절하는 툴을 작성할 때에도 조건문을 쓸 수
있습니다.
어떤 로봇이 장애물을 맞닥뜨렸을 때 해야 하는 행동과 앞이 트여 있을 때 해야 하는
행동을 조절하기 위해서 조건문이 필요합니다.
이처럼, 컴퓨터가 상황에 따라 다른 연산을 하도록 하는 일에 조건문을 사용할 수
있습니다.

\subsubsection{Boolean Type불리언 자료형과 Expressions표현}
지금까지 소개한 \texttt{int}, \texttt{float} 등의 자료형 외에도 자주 쓰게 될
자료형은 Boolean type불리언 자료형\footnote{영국의 수학자이자 논리학자인 George
Boole의 이름을 따온 것입니다.}입니다.  불리언 자료형은 어떤 표현의 참/거짓을
나타내는 형으로, \texttt{True}와 \texttt{False} 두 값을 가집니다.  위의
예시에서 본 표현인 \texttt{x >= 0}은 \texttt{True}일수도 \texttt{False}일수도
있는 불리언 자료형입니다.  불리언 표현을 구성하기 위해서는 대소 관계와 일치
여부 등을 판단하는 관계 연산자와 \texttt{and}, \texttt{or}, \texttt{not} 등의
논리 연산자로 구성될 수 있습니다.  \texttt{x >= 0 or (y < 0 and z < 0)}이
불리언 자료형을 가지는 표현의 예시입니다.

관계 연산자는 다음의 여섯 가지 종류가 있습니다.
\begin{itemize}
  \item \texttt{x == y}: \texttt{x}와 \texttt{y}가 일치하면($=$) \texttt{True}
  \item \texttt{x != y}: \texttt{x}와 \texttt{y}가 일치하지 않으면($\neq$) \texttt{True}
  \item \texttt{x > y}: \texttt{x}가 \texttt{y} 초과이면($>$) \texttt{True}
  \item \texttt{x < y}: \texttt{x}가 \texttt{y} 미만이면($<$) \texttt{True}
  \item \texttt{x >= y}: \texttt{x}가 \texttt{y} 이상이면($\geq$) \texttt{True}
  \item \texttt{x <= y}: \texttt{x}가 \texttt{y} 이하면($\leq$) \texttt{True}
\end{itemize}
지금까지 어떤 변수 \texttt{x}에 값을 대입할 때에는 \texttt{=}를 사용하였습니다.
\texttt{x = 10}과 같이 말입니다.  수학에서도 ``let $x= 10$"처럼 $=$를 사용하는
것과 같은 맥락입니다.  나아가 수학에서는 보통 두 값을 비교할 때, ``if $x = 10$,
"과 같은 표현을 사용하는데, Python에서는 두 값을 비교할 때 단순히 등호 하나가
아니라 두 개를 연달아 쓴 \texttt{==}를 사용합니다.  \texttt{if x == 10}과 같이
말입니다.

논리 연산자는 다음 표의 세 가지 종류가 있습니다.
\begin{table}[H]
  \centering
  \begin{tabular}{c|c|c|c|c}
    $p$ & $q$ & \texttt{$p$ and $q$} & \texttt{$p$ or $q$} & \texttt{not $p$}\\ \hline
    \texttt{True} & \texttt{True} & \texttt{True} & \texttt{True} & \texttt{False}          \\
    \texttt{True} & \texttt{False} & \texttt{False} & \texttt{True} & \texttt{False}          \\
    \texttt{False} & \texttt{True} & \texttt{False} & \texttt{True} & \texttt{True}          \\
    \texttt{False} & \texttt{False} & \texttt{False} & \texttt{False} & \texttt{True}
  \end{tabular}
\end{table}
위 표에는 각 표현에 대해서 어떤 불리언 값을 가지는지 진리표가 제시되어 있습니다.
\texttt{and}, \texttt{or}, \texttt{not}은 각각 수학의 $\wedge$, $\vee$, 그리고 $\neg$에 대응됩니다.

이제 여러가지 관계 연산자와 논리 연산자에 대해서 알아보았으니, 기존에 알았던 연산의 순서에 새로 배운 연산자들을 넣어봅시다.
연산의 순서는 우선 순위에 따라 다음과 같이 나열할 수 있습니다.
\begin{enumerate}
  \item \texttt{($\cdot$)}
  \item \texttt{\texttt{**}}
  \item \texttt{*}, \texttt{/}, \texttt{\%}
  \item \texttt{+}, \texttt{-}
  \item \texttt{==}, \texttt{!=}, \texttt{>}, \texttt{<}, \texttt{>=}, \texttt{<=}
  \item \texttt{not}
  \item \texttt{and}
  \item \texttt{or}
  \item \texttt{=}
\end{enumerate}
모든 관계 연산자들은 논리 연산자들보다는 순위가 높고, 산술 연산자들보다는 낮습니다.
이전과 같이 우선 순위가 겹친다면 좌에서 우로 순서가 결정됩니다.

\subsubsection{\texttt{if-else} Conditionals조건문}
\texttt{if-else} statements문의 기본적인 형태는 아래와 같습니다.
\begin{minted}[mathescape,
               linenos,
               breaklines,
               numbersep=5pt,
               frame=lines,
               framesep=2mm]{python}
if expression:
    ...
else:
    ...
\end{minted}
위에서 \texttt{expression}에는 \texttt{x\%2 == 0 or x\%3 == 0} 등의 임의의
불리언 표현이 올 수 있습니다.  만약 제시된 불리언 표현이 \texttt{True}라면 줄
2의 내용이 실행되고, 그렇지 않다면 줄 4의 내용이 실행됩니다.  제시된 불리언
표현이 \texttt{False}일 때 실행할 내용이 없다면 \texttt{else} 부분은 빠뜨릴 수
있습니다.
다음과 같은 예시를 통해 확인해 봅시다.
\begin{minted}[mathescape,
               linenos,
               breaklines,
               numbersep=5pt,
               frame=lines,
               framesep=2mm]{python}
x = 76
if x % 3 == 0 and x % 2 == 0:
    print(x, "is divisible by 6")
else:
    print(x, "is indivisible by 6")
\end{minted}
실제로 실행해보면 76은 6의 배수가 아니기에 \texttt{76 is indivisible by 6}이
출력되는 것을 볼 수 있습니다.

만약 6의 배수가 아니라면, 적어도 3의 배수인지, 혹은 2의 배수인지까지 판별하고
싶다면 어떻게 해야 할까요?
일단 배운 것만으로는 다음과 같이 할 수 있을 것입니다.
\begin{minted}[mathescape,
               linenos,
               breaklines,
               numbersep=5pt,
               frame=lines,
               framesep=2mm]{python}
x = 76
if x % 3 == 0 and x % 2 == 0:
    print(x, "is divisible by 6")
else:
    if x % 3 == 0:
        print(x, "is divisible by 3")
        else:
            if x % 2 == 0:
                print(x, "is divisible by 2")
            else:
                print(x, "is neither divisible by 3 nor 2")
\end{minted}
6의 배수인지 체크한 후, 그렇지 않다면 다시 3의 배수인지 체크한 후, 그렇지
않다면 2의 배수인지 체크하는 코드입니다.  분명 작동은 잘 하겠지만, 가독성도
좋지 않고 깔끔하지도 않습니다.  이런 상황에서 쓸 수 있는 것이
\texttt{elif}\footnote{\texttt{else if}에서 \texttt{else}의 \texttt{el}만 따온
것입니다. C 언어에서는 그대로 \texttt{else if}를 사용합니다.}입니다.  세 가지
이상의 경우가 있을 때, \texttt{if}로 시작하여 두 번째부터 마지막 바로 직전
경우까지는 \texttt{elif}를, 마지막 예외의 경우에는 \texttt{else}로 마무리하는
방식입니다.  마찬가지로 모든 경우에 해당하지 않을 때 시행할 것이 없으면 마지막
\texttt{else}는 생략할 수 있습니다.  \texttt{elif}를 활용하면 위의 코드를
아래와 같이 간결하게 나타낼 수 있습니다.
\begin{minted}[mathescape,
               linenos,
               breaklines,
               numbersep=5pt,
               frame=lines,
               framesep=2mm]{python}
x = 76
if x % 3 == 0 and x % 2 == 0:
    print(x, "is divisible by 6")
elif x % 3 == 0:
    print(x, "is divisible by 3")
elif x % 2 == 0:
    print(x, "is divisible by 2")
else:
    print(x, "is neither divisible by 3 nor 2")
\end{minted}
만약 \texttt{if-elif-...-elif-else}를 훑는 중간에 참인 것이 나오면, 해당
조건에서 실행되는 명령을 수행한 후 이후의 경우는 모두 건너뛰게 됩니다.

또한 흥미로운 사실은, \texttt{if-else} 문에서 불리언 표현 대신 \texttt{int}
등의 자료형을 가진 변수를 넣어도 된다는 것입니다.  \texttt{0}은 거짓,
\texttt{0} 이외의 모든 수는 참의 값을 가집니다.  그렇다고 이러한 값들이 정확히
\texttt{True} 혹은 \texttt{False}의 값을 가지는 것은 아닙니다.  실제로
\texttt{0 == False}, \texttt{1 == True}의 시행 결과는 \texttt{True}가 나오지만,
\texttt{2 == True}에 대한 결괏값은 \texttt{False}이기 때문입니다.  또한
문자열의 경우 \texttt{""} 외의 하나 이상의 문자를 담고 있는 문자열은 참입니다.
\texttt{""}는 거짓에 해당합니다.
아직은 배우지 않았지만, 문자열과 마찬가지로 list리스트 또한 하나 이상의 값을
담고 있는 경우 참, 그렇지 않은 \texttt{[]}은 거짓입니다.  이는 사실
\texttt{bool}형이 \texttt{int}형의 부분이기 때문인데, \texttt{True}는 정확히
\texttt{1}과, \texttt{False}는 정확히 \texttt{0}과 동일합니다.\footnote{사실
Python 2에서는 \texttt{True}와 \texttt{False}가 키워드로 지정이 되어 있지 않아
일반 변수명으로 사용하여 아무 값으로나 재지정할 수 있습니다. 따라서 사용자가
\texttt{True = 3}과 같은 식으로 값을 배정할 수 있습니다. 반면 Python 3에서는
\texttt{True}와 \texttt{False} 모두 키워드라 값을 지정하려고 하면 syntax 에러가
발생하게 됩니다.} 따라서, 다음과 같은 결과가 나옵니다.
\begin{minted}[mathescape,
               linenos,
               breaklines,
               numbersep=5pt,
               frame=lines,
               framesep=2mm]{python}
>>> True * 3
3
>>> True * False
0
>>> (6 % 2 == 0) + False
1
\end{minted}
그리고 이 사실을 활용하여 다음과 같은 (유의미한) 코드를 쓸 수 있습니다.
\begin{minted}[mathescape,
               linenos,
               breaklines,
               numbersep=5pt,
               frame=lines,
               framesep=2mm]{python}
money = 100
if money:
    print("I have money")
else:
    print("I don't have money")
\end{minted}
\textbf{단, 여기서 중요한 사실은 불리언 \texttt{True}와 참인 표현이 같지 않다는
것입니다. 예컨대 \texttt{True != 100}입니다.}

\subsubsection{Multiple Exit Points 다중 반환점}
한 함수에 \texttt{if-else} 문을 넣어 경우별로 값을 반환하도록 만들 수 있습니다.
\textbf{2.2 Conditionals조건문} 처음에 제시한 \texttt{abs($\cdot$)} 함수가 그
예시로, \texttt{x >= 0}일 경우 \texttt{x}를 반환하고 그렇지 않은 경우
\texttt{-x}를 반환하였습니다.  이러한 함수에는 여러 개의 \texttt{return}을
사용하였기 때문에 multiple exit points다중 반환점\footnote{한국어로 적절한
번역이 없어서 다중 반환점이라는 표현을 사용하였습니다.}이 있다고 표현하기도
합니다.  함수에 여러 개의 반환점이 있으면 경우에 따라 실수를 하여 예기치 못한
버그가 생길 수 있기 때문에, 다중 반환점을 쓰는 것이 바람직한가에 대한 논쟁이
있습니다.  하지만, 적절한 상황에서 사용한다면 분명 편리할 것입니다.  만약
\texttt{abs($\cdot$)} 함수를 하나의 반환점만을 사용한다면 아래와 같게 나타낼 수
있습니다.
\begin{minted}[mathescape,
               linenos,
               breaklines,
               numbersep=5pt,
               frame=lines,
               framesep=2mm]{python}
def abs(x):
    if x >= 0:
        y = x
    else:
        y = -x
    return y
\end{minted}
2.2절 처음에서 제시한 코드와 위 코드의 우열을 가릴 수는 없습니다.
자신이 편한 방식을 택하면 됩니다.
다만 다중 반환점을 2.2절 처음에서 제시한 코드와 다르게, \texttt{else} 없이
아래와 같이 다시 쓸 수 있습니다.
\begin{minted}[mathescape,
               linenos,
               breaklines,
               numbersep=5pt,
               frame=lines,
               framesep=2mm]{python}
def abs(x):
    if x >= 0:
        return x
    return -x
\end{minted}
둘 다 다중 반환점을 사용하였지만, 약간의 의미 차이가 있습니다.  위 코드의
경우에는 가능한 경우를 모두 걸러서 마지막 경우에는 \texttt{-x}를 반환하라는
의미가 있습니다.  이렇게 다중 반환점을 사용할 경우, 일반적으로는 다음과 같이
함수를 만들 수 있습니다.
\begin{minted}[mathescape,
               linenos,
               breaklines,
               numbersep=5pt,
               frame=lines,
               framesep=2mm]{python}
def abs(x):
    if ...:
        return ...
    if ...:
        return ...
    ...
    if ...:
        return ...
    return ...
\end{minted}

\subsection{예제}
\begin{enumerate}
  \item 정규분포를 표현하기 위해 사용되는 가우시안 함수는
    \[
    g(x) = \frac{1}{\sigma \sqrt{2\pi}} e^{-\frac12 \left(\frac{x - \mu}{\sigma}\right)^2},
    \]
    이며, 이때 $\mu$는 기댓값, $\sigma$는 표준편차입니다.
    가우시안 함수 \verb|gaussian(mu, sigma, x)|를 작성하세요.
\begin{minted}[mathescape,
               linenos,
               breaklines,
               numbersep=5pt,
               frame=lines,
               framesep=2mm]{python}
import math

def gaussian(mu, sigma, x):
    # Add here!

print(gaussian(0, 1, 0))                # 0.3989422804014327
print(gaussian(-2, math.sqrt(0.5), 0))  # 0.010333492677046037
\end{minted}

\item 전하가 각각 $q_1$과 $q_2$인 두 물체 사이의 쿨롱힘은 거리 $r$에 따른 함수
\[
  F_C(r) = \frac{1}{4\pi \varepsilon_0} \frac{q_1 q_2}{r^2}\ \text{where $\varepsilon_0 = 8.854\,187\,817 \times 10^{-12}\ \mathrm{F \cdot m^{-1}}$}.
\]
로 쓸 수 있습니다.
또한, 질량이 각각 $m_1$, $m_2$인 두 물체 사이의 만유인력은 거리 $r$에 따른 함수
\[
  F_g(r) = G \frac{m_1 m_2}{r^2}\ \text{where $G = 6.674\,08 \times 10^{-11}\ \mathrm{m^3 \cdot kg^{-1} s^{-2}}$}.
\]
로 표현됩니다.
두 물체가 전하 \texttt{q1}과 \texttt{q2}, 그리고 질량 \texttt{m1}과
\texttt{m2}를 가지고, 거리 \texttt{r}만큼 떨어져있다고 가정합니다.
각각 쿨롱힘과 만유인력을 구하여 반환하는 함수 \verb|coulomb(q1, q2, r)|과
\verb|gravity(m1, m2, r)|을 작성하세요.
또한 이 둘 모두를 구하여 더하는 \texttt{total\_force(q1, q2, m1, m2, r)}도
작성하세요.
\begin{minted}[mathescape,
               linenos,
               breaklines,
               numbersep=5pt,
               frame=lines,
               framesep=2mm]{python}
import math

def coulomb(q1, q2, r):
    # Add here!

def gravity(m1, m2, r):
    # Add here!

def total_force(q1, q2, m1, m2, r):
    # Add here!

e_c = -1.6021766208e-19
e_m = 9.10938356e-31
p_c = -e_c
p_m = 1.672621898e-27
a_0 = 5.2917721067e-11

print(coulomb(e_c, p_c, a_0))
print(gravity(e_m, p_m, a_0))
print(total_force(e_c, p_c, e_m, p_m, a_0))
\end{minted}

\item When $a \neq 0$, a function $f(x) = ax^2 + bx + c$ has an extremum.
Write a function \texttt{extremum(a, b, c)} that returns the extremum of a function $f(x) = \texttt{a}x^2 + \texttt{b}x + c$.
\begin{minted}[mathescape,
               linenos,
               breaklines,
               numbersep=5pt,
               frame=lines,
               framesep=2mm]{python}
def f(a, b, c, x):
    return a*x**2 + b*x + c

def extremum(a, b, c):
    # Add here!

print(extremum(1, 5, 10))   # 3.75
print(extremum(1, -5, 10))  # 3.75
print(extremum(3, 7, 5))    # 0.916666666667
\end{minted}

\item An integer \texttt{n} where $1 \leq \texttt{n} \leq 9999$ is given. Write
  a function \texttt{reverse(n)} that returns an integer with reversed digits
  of \texttt{n}.
\begin{minted}[mathescape,
               linenos,
               breaklines,
               numbersep=5pt,
               frame=lines,
               framesep=2mm]{python}
# Add here!

print reverse(3702)  # 2073
print reverse(370)   # 73
print reverse(37)    # 73
print reverse(3)     # 3
\end{minted}

\item 10 이상 990 이하의 10의 배수 \verb|n|이 주어졌을 때, 10, 50, 100, 500원
  동전을 최소로 사용해서 \verb|n|원을 만드는데 사용되는 동전의 수를 반환하는
  함수 \verb|count_coins(n)|을 작성하세요.
\begin{minted}[mathescape,
               linenos,
               breaklines,
               numbersep=5pt,
               frame=lines,
               framesep=2mm]{python}
# Add here!

print(count_coins(730))  # 6
print(count_coins(790))  # 8
print(count_coins(260))  # 4
print(count_coins(70))   # 3
\end{minted}

\item Write a function \texttt{least(a, b, c)} that returns the least number
  among three numbers \texttt{a}, \texttt{b}, and \texttt{c}.  The type of the
  numbers can be \texttt{int} or \texttt{float}.
\begin{minted}[mathescape,
               linenos,
               breaklines,
               numbersep=5pt,
               frame=lines,
               framesep=2mm]{python}
# Add here!

print(least(1, 2, 3))  # 1
print(least(2, 1, 3))  # 1
print(least(1, 1, 2))  # 1
\end{minted}

\item Write a function \texttt{count(n1, n2, n3, n4)} that counts the number of
  the multiple of 2 or 3 but not 6. Assume inputs are \texttt{int} type.
\begin{minted}[mathescape,
               linenos,
               breaklines,
               numbersep=5pt,
               frame=lines,
               framesep=2mm]{python}
# Add here!

print(count(1, 6, 3, 4))    # 2
print(count(12, 3, 5, 15))  # 2
\end{minted}

\item Write a function \texttt{quadrant(x, y)} that returns:
\begin{itemize}
  \item \texttt{"First Quadrant"} if $\texttt{x} > 0\ \&\ \texttt{y} > 0$
  \item \texttt{"Second Quadrant"} if $\texttt{x} < 0\ \&\ \texttt{y} > 0$
  \item \texttt{"Third Quadrant"} if $\texttt{x} < 0\ \&\ \texttt{y} < 0$
  \item \texttt{"Fourth Quadrant"} if $\texttt{x} > 0\ \&\ \texttt{y} < 0$
\item \texttt{"On the Boundary"} otherwise
\end{itemize}
Assume \texttt{x} and \texttt{y} are \texttt{float} type.
\begin{minted}[mathescape,
               linenos,
               breaklines,
               numbersep=5pt,
               frame=lines,
               framesep=2mm]{python}
# Add here!

print(quadrant(10, 5))   # First Quadrant
print(quadrant(-5, 3))   # Second Quadrant
print(quadrant(-5, -7))  # Third Quadrant
print(quadrant(3, -5))   # Fourth Quadrant
print(quadrant(0, -3))   # On the Boundary
\end{minted}

\item Write a function \texttt{ceiling(x)} that returns $\lceil\texttt{x}\rceil$.
Use \texttt{if-else} statements.
\begin{minted}[mathescape,
               linenos,
               breaklines,
               numbersep=5pt,
               frame=lines,
               framesep=2mm]{python}
# Add here!

print(ceil(4.3))   # 5
print(ceil(-0.3))  # 0
print(ceil(0.01))  # 1
\end{minted}

\item Write a function \texttt{min\_x(a, b, c)} that returns the integer $x$ that minimizes $\texttt{a}x^2 + \texttt{b}x + \texttt{c}$.
Assume that \texttt{a}, \texttt{b}, and \texttt{c} are \texttt{float} type, where $\texttt{a} > 0$ and $\texttt{b} < 0$.
If such $x$ is not unique, return the smallest one.
\begin{minted}[mathescape,
               linenos,
               breaklines,
               numbersep=5pt,
               frame=lines,
               framesep=2mm]{python}
def f(a, b, c, x):
    return a*x**2 + b*x + c

def min_x(a, b, c):
    # Add here!

print(min_x(1, -9, 2))    # 4
print(min_x(9, -5, 0))    # 0
print(min_x(9, -15, 0))   # 1
print(min_x(7, -13, 3))   # 1
\end{minted}

\item \texttt{"S"}, \texttt{"R"}, and \texttt{"P"} represent scissors, rock,
  and paper, respectively.  Write a function \texttt{rock\_paper\_scissors(a,
  b)} that returns \texttt{"a"} if \texttt{a} wins, \texttt{"b"} if \texttt{b}
  wins, or \texttt{Tie} if the game is tied, where \texttt{a} and \texttt{b}
  are elements of $\{\texttt{"S"}, \texttt{"R"}, \texttt{"P"}\}$.
\begin{minted}[mathescape,
               linenos,
               breaklines,
               numbersep=5pt,
               frame=lines,
               framesep=2mm]{python}
def rock_paper_scissors(a, b):
    if a == b:
        return "Tie"
    # Add here!
    
print(rock_paper_scissors("R","R"))  # Tie
print(rock_paper_scissors("R","S"))  # a
print(rock_paper_scissors("R","P"))  # b
print(rock_paper_scissors("S","S"))  # Tie
print(rock_paper_scissors("S","P"))  # a
print(rock_paper_scissors("S","R"))  # b
print(rock_paper_scissors("P","P"))  # Tie
print(rock_paper_scissors("P","R"))  # a
print(rock_paper_scissors("P","S"))  # b
\end{minted}

\item \texttt{gold1}, \texttt{silver1}, and \texttt{bronze1} represent the
  numbers of gold, silver, and bronze medals of the first country,
  respectively.  \texttt{gold2}, \texttt{silver2}, and \texttt{bronze2}
  represent the numbers of gold, silver, and bronze medals of the second
  country, respectively.  Write a function \texttt{better(gold1, silver1,
  bronze1, gold2, silver2, bronze2)} that returns \texttt{"First"} if the first
  country achieved better than the second, \texttt{"Second"} if the second
  country achieved better, or \texttt{"Tie"} if they tied.  The score is
  evaluated according to the gold-silver-bronze order, not by the sum of total
  medals. 
\begin{minted}[mathescape,
               linenos,
               breaklines,
               numbersep=5pt,
               frame=lines,
               framesep=2mm]{python}
def better(gold1, silver1, bronze1, gold2, silver2, bronze2):
    if gold1 > gold2:
        return "First"
    # Add here!

print(better(10,4,24, 1,35,25))  # First
print(better(1,35,25, 10,4,24))  # Second
print(better(10,18,0, 10,4,24))  # First
print(better(10,4,24, 10,18,0))  # Second
print(better(10,20,5, 10,20,4))  # First
print(better(10,20,4, 10,20,5))  # Second
print(better(10,20,5, 10,20,5))  # Tie
\end{minted}

\item Write a function \texttt{area\_triangle(a, b)} that returns the area of a
  triangle formed by $y = \texttt{a}x + \texttt{b}$, $x$-axis, and $y$-axis.
  Return 0 if no triangle is formed. Assume \texttt{a} and \texttt{b} are
  either \texttt{int} or \texttt{float} type.
\end{enumerate}
\begin{minted}[mathescape,
               linenos,
               breaklines,
               numbersep=5pt,
               frame=lines,
               framesep=2mm]{python}
def area_triangle(a, b):
    # Add here!

\end{minted}
\end{document}
